\chapter{Inductors}
\label{chapInductors}

In this chapter, we will begin our study of \glossterm{inductors}.

\section{What is an Inductor}

In Chapter~\ref{chapCapacitors}, we learned that capacitors stored charge by using an electric field.
We also learned that capacitors continued to hold their charge even after it was disconnected from the rest of the circuit.
We learned in Chapter~\ref{chapSound} that capacitors blocked DC signals but allowed AC signals to pass through.

An inductor is kind of like the capacitor's evil twin.  
It does a lot of the same kinds of things as a capacitor, but in kind of an opposite way.

An inductor is a device that consists of wire wound around a core material.
Like a capacitor, an inductor stores energy.
However, while the capacitor stores energy in an essentially unmoving electric field, inductors store energy in a magnetic field.
Magnetic fields only occur when current is flowing.
Every wire with current flowing contains both a tiny electric field and a tiny magnetic field.
Not enough to notice or be bothered with, but it is there.
Capacitors work by augmenting the size of the conductors enough that the electric field becomes significant enough to do work.
Inductors work by augmenting the magnetic field produced by flowing current.
By wrapping a wire around and around in a circle, many magnetic fields from many wires are aligned in the same direction, producing a magnetic field.

The energy in an inductor is stored in the inductor's magnetic field.
Current flowing into an inductor is used to first generate the magnetic field, and then the magnetic field's energy keeps current flowing.

While a capacitor stores energy as a charge and releases it when there is a voltage change, an inductor stores energy as magnetic flux (i.e., how large/strong the magnetic field is) and releases it when there is a current change.
An inductor's inductance is measured using the \glossterm{henry} (H).

The core equations of capacitors and inductors are related as well.
The defining equation of a capacitor is $Q = C\cdot V$, where $Q$ is the charge in the capacitor.
The defining equation of an inductor is $\phi = L\cdot I$, where $\phi$ is the magnetic flux in the inductor, $L$ is the inductance (in henries), and $I$ is the current (in amperes).



