\chapter{Dealing with Units}

Before we begin our exploration of electronics, we need to talk about \glossterm{units of measurement}.
A unit of measurement is basically a standard against which we are measuring something.
For instance, when measuring the length of something, the units of measurement we usually use are feet or meters.
You can also measure length in inches, yards, centimeters, kilometers, miles, etc.
Additionally, there are some obscure units of length like furlongs, cubits, leagues, and paces.

Every type of quantity has its own types of units.
For instance, we measure time in seconds, minutes, hours, days, weeks, and years.
We measure speed in miles per hour, kilometers per hour, meters per second, etc..
We measure mass in pounds, ounces, grams, kilograms, grains, etc.
We measure temperature in Farenheit, Celsius, Kelvin, and Rankine.

Units for the same type of quantity can all be converted into each other using the proper formula.

\section{SI Units}

The scientific community has largely agreed upon a single standard of units known as the \glossterm{International System of Units}, abbreviated as \glossterm{SI Units}.
This is the modern form of the metric system.
Because of the large number of unit systems available, the goal of creating the SI standard was to create a single set of units that had a basis in physics and had a standard way of expressing larger and smaller quantities.

The imperial system of volumes illustrates the problem they were trying to solve.  
In the imperial system, there were gallons.
If you divided a gallon into four parts, you would get quarts.
If you divided quarts in half, you get pints.
If you divided a pint into twentieths, you get ounces.

The imperial system was very confusing.
Not only were there an enormous number of units, but they all were divisible by differing amounts.
The case was similar for length---twelve inches in a foot, but three feet to a yard, and 1,760 yards in a mile.
This was a lot to memorize, and doing the calculations was not easy.

The imperial system does have some benefits (the quantities used in the imperial system match the sizes normally used in human activities---few people order drinks in milliliters), but for doing work which require a lot of calculations and units, the SI system has largely won out.
Scientific quantities are almost always expressed in SI units.  
In engineering it is more of a mix, just as engineering is a mix between scientific inquiry and human usefulness.
However, the more technical fields have started with SI units and kept with them.

So, the base units for the SI system for everyday quantities are the following:

\begin{itemize}
\item Unit of length---the meter
\item Unit of time---the second
\item Unit of mass---the gram
\item Unit of force---the newton
\end{itemize}

\section{Scaling Units}

Now, sometimes you are measuring really big quantities and sometimes you are measuring very small quantities.
In the imperial system, there are different units altogether to reach a different scale of a quantity.
For instance, there are inches for small distances, yards for medium-sized distances, and miles for large distances.
There are ounces for small volumes and gallons for larger volumes.

In the SI system, however, there is a uniform standard way of expressing larger and smaller quantities.
There are a set of modifiers, known as \glossterm{unit prefixes}, which can be added to \emph{any unit} to work at a different scale.
For example, the prefix \emph{kilo} means thousand.
So, while a meter is a unit of length, a kilometer is a unit of length that is 1,000 times as large as a meter.
While a gram is a unit of mass, a kilogram is a unit of mass that is 1,000 times the mass of a gram.

It works the other way as well.
The prefix \emph{milli} means thousandth, as in $\frac{1}{1000}$.
So, while meter is a unit of length, a millimeter is a unit of length that is $\frac{1}{1000}$ of a meter.
While a gram is a unit of mass, a microgram is a unit of mass that is $\frac{1}{1000}$ the mass of a gram.

Therefore, by memorizing one single set of prefixes, you can know how to modify all of the units in the SI system.
The common prefixes occur at every power of 1,000, as you can see in Figure~\ref{figSIPrefixes}.

\begin{figure}
\caption{Common SI Prefixes}
\label{figSIPrefixes}
\begin{tabular}{|l|l|l|l|}
Conversion Factor & Prefix & Abbreviation & Examples \\
\hline
1,000,000,000,000 & tera & T & terameter, terasecond, teragram \\
1,000,000,000 & giga & G & gigameter, gigasecond, gigagram \\
1,000,000 & mega & M & megameter, megasecond, megagram \\
1,000 & kilo & K & kilometer, kilosecond, kilogram \\
1 & & & meter, second, gram \\
0.001 & milli & m & millimeter, millisecond, milligram \\
0.000001 & micro & $\si{\micro}$ or u & micrometer, microsecond, microgram \\
0.000000001 & nano & n & nanometer, nanosecond, nanogram \\
0.000000000001 & pico & p & picometer, picosecond, picogram \\
\end{tabular}
\end{figure}

To convert between a prefixed unit (i.e., kilogram) and a base unit (i.e., gram), we just apply the conversion factor.
So, if something weighs $24.32$ kilograms, then I could convert that into grams by multiplying by $1,000$---$24.32 \cdot 1000 = 24320$.
In other words, $24.32$ kilograms is the same as $24,320$ grams.

To move from the base unit to a prefixed unit, you \emph{divide} by the conversion factor.
So, if something weighs $35.2$ grams, then I could convert that into kilograms by dividing it by $1,000$---$35.2 / 1000 = 0.0352$.
In other words, $35.2$ grams is the same as $0.0352$ kilograms.

You can also convert between two prefixed units.  
You simply multiply by the starting prefix and divide by the target prefix.
So, if something weight $220$ kilograms and I want to know how many micrograms that is, then I will multiply using the kilo prefix (1,000) and divide by the micro prefix (0.000001):

$$\frac{220 \cdot 1000}{0.000001} = 220000000000$$

In other words, 220 kilograms is the same as 220,000,000,000 micrograms.

\section{Using Abbreviations}

\section{Visualizing the Prefixes}

You can do everything you need just by knowing the multipliers.
However, what usually helps me deal with these multipliers intuitively and instinctively is by simply visualizing where each one lands in a single number.
Figure~\ref{figSIUnitVisualization} shows all of the prefixes laid out in a single number.

\begin{figure}
\caption{Visualizing Common Unit Prefixes}
\label{figSIUnitVisualization}
\centering
$$\mathlarger{\mathlarger{\mathlarger{\underbrace{000}_\text{\smaller[2]{tera (T)}}\underbrace{000}_\text{\smaller[2]{giga (G)}}\underbrace{000}_\text{\smaller[2]{mega (M)}}\underbrace{000}_\text{\smaller[2]{kilo (K)}}\overbrace{000}^\text{\smaller[2]{units}}.\underbrace{000}_\text{\smaller[2]{milli (m)}}\underbrace{000}_\text{\smaller[2]{micro ($\mu$)}}\underbrace{000}_\text{\smaller[2]{nano (n)}}\underbrace{000}_\text{\smaller[2]{pico (p)}}}}}$$
\end{figure}

\applysection

\begin{enumerate}
\item \fixme{Need questions on this}
\end{enumerate}

