
Unless otherwise specified, assume that the transistor is a BJT NPN transistor and that the beta is stable.

\begin{enumerate}
\item If the base of a transistor is at $3\myvolt$ and the transistor is on, what will be the emitter voltage?
\item If the base of a transistor is at $45\myvolt$ and the emitter is on, what will be the emitter voltage?
\item If the base of a transistor is at $5\myvolt$ and the emitter, if conducting, would have to be at $4.5\myvolt$, is the transistor on or off?
\item If the base of a transistor is at $0.6\myvolt$ and the emitter is at ground, is the transistor on or off?
\item If the base of a transistor is at $0.4\myvolt$ and the emitter is at ground, is the transistor on or off?
\item If a transistor has a base current ($I_{BE}$) of $2\mymamp$ and the transistor has a beta of $55$, how much current is going through the collector ($I_{CE}$)?
\item In the previous problem, how much total current is coming out of the emitter?
\item If a transistor has a base current of $3\mymamp$ and the transistor has a beta of $200$, how much current is going through the collector?
\item If the base voltage is greater than the collector voltage, what does this mean for our transistor operation?
\item The output of your microcontroller is $3.3\myvolt$ and supports a maximum output current of $10\mymamp$.  Using Figure~\ref{figArduinoTransistorSwitchOnMotor} as a guide, design a circuit to control a motor that requires $80\mymamp$ to operate.  Assume that the transistor beta is 100.
\item Redesign the previous circuit so that it utilizes a stabilizing resistor on the emitter to prevent variations in the transistor beta.
\end{enumerate}
