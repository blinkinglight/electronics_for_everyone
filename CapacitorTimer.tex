\chapter{Capacitors as Timers}
\label{chapCapacitorTimer}

In this chapter, we are going to learn how to use measure the time it takes for a capacitor to charge.  
Once we learn this, we can use capacitors for timers---both for delaying signal as well as for creating an oscillating circuit.

\section{Time Constants}

As we learned in Chapter~\ref{chapCapacitors}, when a voltage is applied to a capacitor it will store energy by storing a charge on its plates, the amount of charge being based on the voltage supplied (see Equation~\ref{eqBasicCapacitance}).
However, if a capacitor charges through a resistor, then it takes much longer to fill the capacitor to capacity than if it were connected to the battery directly.
In fact, it never \emph{fully} reaches capacity, though it gets close enough that we say that it does.

Having a resistor in series with a capacitor is a configuration known as an \glossterm{RC circuit}.
The amount of time it takes a capacitor to charge is based on both the resistance of the resistor and the capacitance of the capacitor.
The actual equation for this is kind of complicated but there is a simple trick that suffices for nearly every situation, known as the RC time constant.

The RC time constant is merely the product of the resistance (in ohms) multiplied by the capacitance (in farads) which will yield the RC time constant in seconds.
The RC time constant can be used to determine how long it will take to charge a capacitor to a given level.
So, for instance, if I have a $100\myuf$ capacitor and a $500\myohm$ resistor, the RC time constant is $0.0001 * 500 = 0.05\textrm{ seconds}$.

This constant can then be used with the table in Figure~\ref{figTimeConstants} to determine how long it will take to charge a capacitor to a given level.

\begin{figure}
\caption{RC Time Constants}
\label{figTimeConstants}
\begin{tabular}{|l|l|l}
\textbf{\# of Time Constants} & \textbf{\% of Voltage} & \textbf{\% of Current} \\
0.5 & 39.3\% & 60.7\% \\
0.7 & 50.3\% & 49.7\% \\
1 & 63.2\% & 36.8\% \\
2 & 86.5\% & 13.5\% \\
3 & 95.0\% & 5.0\% \\
4 & 98.2\% & 1.8\% \\
5 & 99.3\% & 0.7\% \\
\end{tabular}
\end{figure}

For instance, if I wait for 2 time constants (in this case, $0.05 \cdot 2 = 0.1\textrm{ seconds}$), my capacitor will be charged to 86.5\% of the supply voltage.
The current flowing through it will be at 13.5\% of what current would be flowing if there was just a straight wire instead of a capacitor.

While a capacitor is never really ``fully charged'' (because it never fully reaches 100\%), in this book, we will use 5 time constants to consider a capacitor fully charged.

\begin{exampleprob}
I have a power supply that is $7\myvolt$ and a capacitor that is a $100\myuf$ capacitor.
I want it to take 9~seconds to charge my capacitor.
What size of resistor do I need to use to do this?

To solve this problem, we need to work backwards.
Remember, we are considering 5 time constants to be fully charged.
Therefore, the time constant we are hoping to achieve is $9 / 5 = 1.8\textrm{ seconds}$.
The capacitance is $100\myuf$, which is $0.0001\myf$.
Since the time constant is merely the product of the capacitance and resistance, we can solve for this as follows:

\begin{align*}
\textrm{RC Time Constant} &= \textrm{capacitance} \cdot \textrm{resistance} \\
1.8 &= 0.0001 \cdot R \\
0.0001 \cdot R &= 1.8 \\
R &= \frac{1.8}{0.0001} \\
R &= 18000
\end{align*}

Therefore, to make it take 9~seconds to charge the capacitor, we need to use an $18\mykohm$ resistor.
\end{exampleprob}

In the same way, if we connect the capacitor to ground through the resistor instead of to the voltage supply, the capacitor will discharge in the same way that it charged.
It will begin discharging a lot, but then, as it gets closer to zero, it will level off the amount that it is discharging.
You would then read Figure~\ref{figTimeConstants} to be the percentage \emph{discharged} that the capacitor was.

\begin{exampleprob}
Suppose a $100\myuf$ capacitor has been charged up to $7\myvolt$ and then is disconnected from the power supply, but the ground connection remains.
We then connect the positive terminal of the capacitor to a $5\mykohm$ resistor that connects to ground.
After 2 seconds, how much voltage is remaining in the capacitor?

To find this out, we first need to find the RC time constant of the circuit.

\begin{align*}
T &= 5\mykohm \cdot 100\myuf \\
  &= 5,000\myohm \cdot 0.0001\myf \\
  &= 0.5\textrm{ seconds}
\end{align*}

So the RC time constant is 0.5 seconds.

So, after 2 seconds, we have performed 4 RC time constants.
Looking at Figure~\ref{figTimeConstants}, we have discharged 98.2\% of the capacitor's voltage.
This means that the amount of voltage \emph{lost} should be $0.982 \cdot 7\myvolt = 6.874\myvolt$.
If we have lost that much voltage, that means we should subtract it, so the remaining voltage is $7\myvolt - 6.874\myvolt = 0.126\myvolt$.

So the voltage after 2 seconds will be $0.125\myvolt$.
\end{exampleprob}

\section{Constructing a Simple Timer Circuit}

Let's say that we want a circuit that, when turned on, \emph{waits} for a certain amount of time and then does something.
How might we do it?

Think of it this way.  
We can use capacitor charging to give us a time delay.
However, we need something that ``notices'' when the time delay is finished.
In other words, we need a way to trigger something when a certain threshold is crossed.

What happens to the capacitor as it charges?
The voltage across its terminals changes.
When it is first connected to the circuit, there is zero voltage across its terminals.
As it charges, the voltage across the capacitors terminals keeps going up until it matches the supply voltage.
Therefore, we know when we have hit our target time based on when the voltage is at a certain level.
But how will we know when we are at the right voltage?
Have we done a circuit so far that detects voltages?
What component did we use?

If you remember back to Chapter~\ref{chapIC}, we used the LM393 to compare voltages.
We supplied the LM393 with a \emph{reference} voltage, and then it triggered when our other voltage went above that voltage.
We can do the same thing here.

What we will construct is a circuit that waits for 5~seconds and then turns on an LED.
In order to do this, we will need to choose (a) a reference voltage to use, (b) a resistor/capacitor combination that will surpass the reference voltage after a certain amount of time, and (c) an output circuit that lights up the LED.

There are virtually infinite combinations we could choose from for our reference voltage, resistance, and capacitance.
In fact, the supply voltage doesn't matter so much since what we are looking at is \emph{percentages} of the supply voltage, which will be the same not matter what the actual voltage is.

% FIXME - should I use 5V or 9V?

For this example, we will use a basic $5\myvolt$ supply, and make the reference voltage to be half of the supply voltage.
This allows us to use any two equivalent resistors into a voltage divider to get our reference voltage.
Technically the voltage itself does not matter nearly as much as the fact that we are using half of the supply voltage as our reference voltage.

Now, since the reference voltage is half of our supply voltage, we will use the table in Figure~\ref{figTimeConstants} to determine how many time constants that needs to be.
The table says that for 50\% of voltage, it will take 0.7 time constants (it is actually 50.3\%, but we are not being that exact).

Therefore, the equation for our resistor and capacitor is 

\begin{align*}
T &= 5\textrm{ seconds} \\
5 \textrm{ seconds} &= 0.7\cdot R\cdot C \\
\frac{5}{0.7} &= R\cdot C \\
7.14 &= R\cdot C
\end{align*}

So, we can use any resistor and capacitor such that the ohms multiplied by the farads equals 7.14.
I usually use $100\myuf$ capacitors for larger time periods such as this because they are larger and because being exactly $0.0001\myf$ it makes it easier to calculate with.
Therefore, we can very simply calculate the needed resistor for this capacitor.

\begin{align*}
7.14 &= R\cdot C \\
7.14 &= R\cdot 0.0001 \\
\frac{7.14}{0.0001} = R \\
71400 = R
\end{align*}

Therefore, we need a resistor that is about $71,400\myohm$.
We could choose a combination of resistors that hits this exactly, but for our purposes, we only need to get close.
In my case, the closest resistor I have is $68,000\myohm$ (i.e., $68\mykohm$).
That is close enough (though I could get even closer by adding in a $3\mykohm$ resistor in series).

\simplegraphicsfigure{A Simple Timer for an LED}{SimpleTimerCircuit}{0.08}

Figure~\ref{figSimpleTimerCircuit} shows the full circuit.
When building this circuit, don't forget that the comparator also has to be connected to the supply voltage and ground as well!
As you can see, R2 and R3 form the voltage divider that provides the reference voltage for the comparator at half of the supply voltage.
The values for R2 and R3 are arbitrarily chosen, but they must be equal to get the reference voltage.
I chose medium-high values for these resistors so as to not waste current with the voltage divider.
The circuit made by R1 and C1 is the timing circuit.
When the circuit is first plugged in, C1 is at voltage level 0, essentially acting as a short circuit while it first begins to store charge.
At zero volts, this is obviously less than our reference voltage.
However, as the capacitor charges, less, and less current can flow into C1.
Its voltage level increases under the rules of the RC time constant.
After 0.7 time constants, the voltage will be above the voltage level set by the reference voltage, and our voltage comparator will switch on.

Remember, though, from Chapter~\ref{chapIC}, that the LM393 operates using a pull-up resistor.
That is, the comparator doesn't ever source current.  
It will sink current (voltage level 0 when the + input is less than the - input), or disconnect (when the + input is greater than the - input).
Therefore, R4 is providing a pull-up resistor to supply power to the LED when the LM393 disconnects.

\simplegraphicsfigure{The Capacitor Timer Circuit on a Breadboard}{SimpleTimerBreadboard}{1}

Figure~\ref{figSimpleTimerBreadboard} shows this circuit laid out on the breadboard.
We are using the second voltage comparator of the LM393 just because it was a little easier to show the wiring for it.
If you need to see the pinout for the LM393 again, it was back in Figure~\ref{figlm393Pinout}.

Notice the prominence of our basic circuits from Chapter~\ref{chapBasicResistorCircuits}.
On the top, we have a combination pull-up resistor which is also acting as a current-limiting resistor (as they often do).
On the bottom left, we have a voltage divider.

On the bottom right, we have our new timing circuit, which looks a lot like a voltage divider.
In fact, it acts like one, too, where the voltage varies by time!
If you think about what the capacitor is doing, when we first apply voltage it acts like a short circuit---in other words, no resistance.
This means that there are zero volts across the capacitor, and the resistor is eating up all of the voltage.
But, as the capacitor fills up, it increases voltage, which it is \emph{dividing} with the resistor!

\section{Resetting Our Timer}

The timer is great, except for one thing---how do we turn it off?
You might have noticed that, even if you disconnect power to the circuit, when you turn it back on, it doesn't do any timing anymore!
Remember, the capacitor is \emph{storing} charge.
When you turn off the circuit, it is \emph{still} storing the charge.

Now, you can very simply get rid of the charge by putting a wire between the legs of the capacitor.
However, for larger capacitors, you would need to do this with a resistor instead of a wire in order to keep there from being a dangerous spark (the resistor limits the current, which makes the discharge slower).
But how do we do this with a circuit?

What we can do is add a switch that will do the same thing as putting a wire or resistor across the legs of the capacitor.
Figure~\ref{figSimpleTimerCircuitWithReset} shows how to do this.

\simplegraphicsfigure{Adding a Reset to the Capacitor}{SimpleTimerCircuitWithReset}{0.08}

Now let's look at how this circuit works.
First of all, we added two components, the switch and the R5 resistor connected to it.
To understand what this does, pretend for a minute that the resistor isn't there.
What happens when we push the button?
That would create a direct link from the positive side of the capacitor to ground.
Since the negative side of the capacitor is also connected to ground, that means that these two points would be directly connected.
Thus, they would be at the \emph{same} voltage.
This would happen by the current suddenly rushing from one side to another.

If the capacitor were larger or the voltages greater, this might be somewhat dangerous.
You would have a large current and a large voltage for a short period of time (which would yield a large wattage), which could blow something out.
Therefore, it is good practice to use a small resistor between the button and ground.
The resistor should be \emph{much} smaller than the resistor used to charge the capacitor.
The specific size doesn't matter too much---it needs to be small enough to discharge it quickly and large enough to prevent a spark when you push the button.
You can use the same time constants for discharging the capacitor that you used for charging it.
However, keep in mind that the charging circuit is still running!
That's why the resistor has to be small---it has to discharge the capacitor \emph{while} the other resistor is trying to charge it.

Also notice R1 and R5.  
When the button is being pushed, what is happening to them?
Think back to our basic resistor circuits.  
If we have two resistors, with a wire coming out of the middle, what is that?
It's a voltage divider!
So, not only is R5 being a current-limiting resistor for the button, it is also acting as a voltage divider in concert with R1.

What this means is that the capacitor will only discharge down to the level of voltage division between these resistors. 
That's fine, as long as it is low enough.
You will find that in electronics, ``close enough'' often counts.
The trick is knowing how close is close enough and how close is still too far.

However, usually we can perform some basic calculations to figure this out.
Since R1 is $68\mykohm$ and R5 is $200\myohm$, what is the voltage at the point of division?
If we use a $9\myvolt$ supply, then using the formula from Equation~\ref{eqVoltageDivision} then we can see that:

\begin{align*}
V_{OUT} &= V_{IN} \cdot \frac{R_5}{R_1 + R_5} \\
 &= 9 \cdot \frac{220}{68000 + 220} \\
 &= 9 \cdot \frac{220}{68220} \\
 &\approx 9 \cdot 0.00322 \\
 &\approx 0.029\myvolt
\end{align*}

So, as you can see, when the button is pushed, the final voltage of the capacitor, while not \emph{absolutely} zero volts, is pretty darn close.

Also note that, if we wanted, we can reverse the action of this circuit.  
By swapping our two inputs to the voltage comparator, we can make the circuit be on for 5 seconds and then switch off.

\reviewsection

In this chapter, we learned:

\begin{enumerate}
\item A resistor and capacitor in series with each other is known as an RC circuit.
\item In an RC circuit, the amount of time it takes for a capacitor to charge to the supply voltage is based on the capacitance of the capacitor and the resistance of the resistor.
\item The RC \emph{time constant} is a convenient way to think about how long it takes for a capacitor to charge in RC circuits.
\item The RC time constant is calculated by multiplying the resistance (in ohms) by the capacitance (in farads) with the result being the number of seconds in 1 RC time constant.
\item The table in Figure~\ref{figTimeConstants} shows how long it takes to charge a capacitor to different percentages of supply voltage as a multiple of RC time constants.
\item The RC time constant chart can also be used to calculate the amount of time it takes for a capacitor to discharge to ground if it is disconnected from its source and connected through a resistor to ground.  In this case, Figure~\ref{figTimeConstants} is used to tell the percentage that the voltage has been \emph{discharged}.
\item A timer can be constructed by using a comparator and an RC circuit along with a reference voltage provided by a voltage divider.  By tweaking the RC circuit, the timing can be changed.
\item After charging a capacitor, a means needs to be provided to discharge it as well, such as a button leading to ground.
\item Such discharge methods need to have resistance to prevent sparks and other failures, but not too much resistance as they can accidentally form voltage dividers with other resistors.
\item Even as our circuits get more advanced, the basic circuits we found in Chapter~\ref{chapBasicResistorCircuits} are still dominating our circuit designs.
\end{enumerate}

\applysection


\begin{enumerate}
\item If I have an RC circuit with a resistor of $10\myohm$ and a capacitor of $2\myf$, what is the RC time constant of this circuit?
\item In the previous question, how many seconds does it take to charge my capacitor to approximately 50\% of supply voltage?
\item If I have an RC circuit with a resistor of $30,000\myohm$ and a capacitor of $0.001\myf$, what is the RC time constant of this circuit?
\item In the previous question, what percentage of the way is the capacitor charged after 60 seconds?
\item If I have an RC circuit with a resistor of $25\mykohm$ and a capacitor of $20\myuf$, what is the RC time constant of this circuit?
\item Give a resistor and capacitor combination that will yield an RC time constant of 0.25 seconds.
\item Reconfigure the circuit in Figure~\ref{figSimpleTimerCircuit} to wait for 3 seconds.  Draw the whole circuit.
\item Redraw the previous circuit, and circle each basic circuit pattern and label it.
\end{enumerate}

