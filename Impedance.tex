\chapter{Reactance and Impedance}
\label{chapImpedance}

\section{Reactance and Impedance}

We have discussed resistance quite a bit in this book.  
Resistance is specifically about the ability of a component to be a good conductor of electricity.
When a circuit encounters resistance, power is lost through the resistor.

However, another way of preventing current from flowing is known as \emph{reactance}.
With reactance, the power isn't dissipated, but rather the current is \emph{prevented from flowing altogether}.

Let's think again about what happens when a voltage is connected to a capacitor in series.
The capacitor starts to fill up.
As the capacitor gets more an more full, there is less charge that can get onto the plate of the capacitor.
This prevents the other side from filling up as well, and current cannot get through the capacitor.
This acts \emph{in a similar way to resistance}---it is preventing the flow of current.
However, it is not dissipating power because it is actually preventing the current from flowing.
This is known as reactance.

Reactance is usually frequency-dependent.
Again, going to the capacitor example, with high frequencies, the capacitor is continually charging and discharging, so it never really gets full.
Therefore, it adds very little reactance.
The lower frequencies, however, give the capacitor time to get full.
Therefore, for capacitors, lower frequencies create more reactance.

Reactance is measured in ohms, just like resistance.  
In fact, resistance and reactance are usually added together in a circuit to get a quantity known as \glossterm{impedance}, which is simply the combination of resistive and reactive quantities.

Resistance and reactance aren't added together directly, instead you can think of them acting at angles to each other.
Let's say that I start at my house and walk ten feet out my front door.
Then, I turn left and walk another ten feet.
While I have walked twenty feet, I am \emph{not} twenty feet from my door.
I am, instead, a little over 14 feet from my door.





\section{LC Circuits and Resonant Frequencies}

When a capacitor and an inductor are in series with each other, it is termed an LC series circuit.
Because the inductor inhibits high frequencies and the capacitor inhibits low frequencies, LC circuits can be used to let through a very specific frequency range.
The center of this range is known as the \glossterm{resonant frequency} of the circuit.




