\chapter{Reactance and Impedance}
\label{chapImpedance}

% FIXME - should I split this into two chapters?  Perhaps split it right before Ohm's Law for AC circuits?

\section{Reactance}

We have discussed resistance quite a bit in this book.  
Resistance is specifically about the ability of a component to be a good conductor of electricity.
When a circuit encounters resistance, power is lost through the resistor.

However, another way of preventing current from flowing is known as \emph{reactance}.
With reactance, the power isn't dissipated, but rather the current is \emph{prevented from flowing altogether}.

Let's think again about what happens when a voltage is connected to a capacitor in series.
The capacitor starts to fill up.
As the capacitor gets more an more full, there is less charge that can get onto the plate of the capacitor.
This prevents the other side from filling up as well, and current cannot get through the capacitor.
This acts \emph{in a similar way to resistance}---it is preventing (impeding) the flow of current.
However, it is not dissipating power because it is actually preventing the current from flowing.
This is known as \glossterm{reactance}.
For capacitors it is called \glossterm{capacitative reactance} and for inductors it is called \glossterm{inductive reactance}.

Reactance is usually frequency-dependent.
Again, going to the capacitor example, with high frequencies, the capacitor is continually charging and discharging, so it never really gets full, so it never impedes the current flow very much.
Therefore, capacitors add very little reactance with high-frequency AC current.
The lower frequencies, however, give the capacitor time to get full, and when they are full, they impede the current flow.
Therefore, for capacitors, lower frequencies create more reactance.

Reactance is measured in ohms, just like resistance.  
However, they cannot be simply added to resistances, so they are usually prefixed with the letter $j$.
So, 50 ohms of reactance is usually labelled as $j50\myohm$ so that it is understood as a reactance.
Reactances can be added to each other and resistances can be added to each other.
We will see how to combine them in Section~\ref{secImpedance}.
Reactances are using denotated using the letter $X$.

The reactance of a capacitor ($X_C$) is given by the following formula:

\begin{equation}
\label{eqCapReactance}
X_C = -\frac{1}{2\pi\cdot f\cdot C}
\end{equation}

In this formula, $f$ is the AC frequency of the signal (in hertz) and $C$ is the capacitance.
It might seem strange that this produces a negative value. 
The reason for this will make sense as we go forward.
However, it is \emph{not} producing negative impedance---we will see this when we combine reactance with resistance.

\begin{exampleprob}
What is the reactance of a $50\mynf$ capacitor to a signal of $200\myhz$?

To find this, we merely use the formula:
\begin{align*}
X_C &= \frac{1}{2\pi\cdot f\cdot C} \\
 &= \frac{1}{2\pi\cdot 200\cdot 0.00000005} \\
 &\approx \frac{1}{2\cdot 3.14\cdot200\cdot 0.00000005} \\
 &= \frac{1}{0.00000624} \\
 &\approx -j160256\myohm
\end{align*}
\end{exampleprob}

You can see from this formula why it is said that a capacitor blocks DC current.
DC current is, essentially, current that does not oscillate.
In other words, the frequency is zero.
Therefore, the formula will reduce to $\frac{1}{0}$, which is infinite.
Therefore, it has infinite reactance against DC current.

Also note what happens as the frequency increases.
As the frequency increases, the denominator gets larger and larger.
That means that the reactance is getting smaller and smaller---closer and closer to zero.
As the frequency goes up, the reactance is essentially heading towards zero, but will never get there because the frequency can't be infinite.

The formula for \glossterm{inductive reactance} ($X_L$) is very similar:

\begin{equation}
\label{eqReactanceInductor}
X_L = 2\pi\cdot f\cdot L
\end{equation}

In this equation, $f$ is the signal frequency and $L$ is the inductance of the inductor in Henries.

\begin{exampleprob}
Let's say that I have an $3\myhy$ inductor with a $50\myhz$ AC signal.
How much reactance does the inductor have in this circuit?

\begin{align*}
X_L &= 2\pi\cdot f\cdot L \\
    &= 2\cdot 3.14 \cdot 50 \cdot 3 \\
    &= j9420\myohm
\end{align*}

The reactance in this circuit is $j9,420\myohm$.
\end{exampleprob}

\section{Impedance}
\label{secImpedance}
In fact, resistance and reactance are usually added together in a circuit to get a quantity known as \glossterm{impedance}, which is simply the combination of resistive and reactive quantities.
Impedance is often designated using the letter $Z$.

Resistance and reactance aren't added together directly, instead you can think of them acting at angles to each other.
Let's say that I start at my house and walk ten feet out my front door.
Then, I turn 90 degree right and walk another ten feet.
While I have walked twenty feet, I am \emph{not} twenty feet from my door.
I am, instead, a little over 14 feet from my door.
I can call this my displacement.
Figure~\ref{figDistanceGraph} shows what this looks like visually.

\simplepdffigure{Total Distance Traveled vs. Total Displacement}{DistanceGraph}{0.25}

Since this is a right triangle, the distance from the start to the end is found on the hypotenuse of this triangle.
We can calculate the total displacement using the Pythagorean theorem ($A^2 + B^2 = C^2$). 
If we solve for $C$ (total displacement), we get: 

\begin{equation}
\label{eqTotalDisplacement}
C = \sqrt{A^2 + B^2}
\end{equation}

So, in our distance example, if I went forward 10 feet, turned left, and went another 10 feet, the total distance traveled would be:

\begin{align*}
C &= \sqrt{A^2 + B^2} \\
  &= \sqrt{10^2 + 10^2} \\
  &= \sqrt{100 + 100} \\
  &= \sqrt{200} \\
  &\approx 14.14
\end{align*}

The total impedance is like the total distance from your door.
Resistance and reactance are like different walking directions (at right angles to each other), and impedance is the total displacement.
That's why we use the letter $j$ to signify an impedance---it is just like resistance but in a different direction.

So how doe we calculate impedance? 
In fact, it is calculated \emph{precisely like} the displacement calculation in Equation~\ref{eqTotalDisplacement}:

\begin{equation}
\label{eqTotalImpedance}
\textrm{impedance} = \sqrt{\textrm{resistance}^2 + \textrm{reactance}^2}
\end{equation}

Or, using their common abbreviations, we can say:

\begin{equation}
\label{eqTotalImpedanceLetters}
Z = \sqrt{R^2 + X^2}
\end{equation}

Let's see how we can use Equation~\ref{eqTotalImpedance} to calculate total impedance.
If I have a circuit that has $30\myohm$ of resistance and $j20\myohm$ of reactance, then the formula for total impedance is:

\begin{align*}
\textrm{impedance} &= \sqrt{\textrm{resistance}^2 + \textrm{reactance}^2} \\
  &= \sqrt{30^2 + 20^2} \\
  &= \sqrt{900 + 400} \\
  &= \sqrt{1300} \\
  &\approx 36.1\myohm
\end{align*}

As you can see, when using this formula we are calculating a total impedance, so the $j$ drops away.

\begin{exampleprob}
If I have a $1\mykohm$ resistor in series with a $100\mynf$ capacitor with a $800\myhz$ signal, what is the total impedance to the signal that my circuit is giving?

To find out impedance, we need both resistance and reactance.
We already have resistance---$1\mykohm$.
The reactance is found by using Equation~\ref{eqCapReactance}:

\begin{align*}
X_C &= -\frac{1}{2\pi\cdot f\cdot C} \\
    &= -\frac{1}{2\pi\cdot 800 \cdot 0.0000001} \\
    &\approx -\frac{1}{0.0005024} \\
    &\approx -j1990\myohm
\end{align*}

So the reactance is about $-j1990\myohm$.

So, if the resistance is $1000\myohm$ and the reactance is $-j1990\myohm$ what is the impedance?
The impedance is found by using Equation~\ref{eqTotalImpedance}:

\begin{align*}
\textrm{impedance} &= \sqrt{\textrm{resistance}^2 + \textrm{reactance}^2} \\
 &= \sqrt{1000^2 + (-1990)^2} \\
 &= \sqrt{1000000 + 3960100} \\
 &= \sqrt{4960100} \\
 &\approx 2227\myohm
\end{align*}

\end{exampleprob}


\section{RLC Circuits}

So far we have discussed RC (resistor-capacitor) and RL (resistor-inductor) circuits.
When you combine all of these components together, you get an RLC (resistor-inductor-capacitor) circuit.

When you calculate the impedance of such circuits, you have to be sure you include the reactance of \emph{both} the capacitative and the inductive components.
While inductors and capacitors both offer reactance to certain frequencies, their reactances actually oppose each other.
That is, the reactance of one cancels out the reactance of the other.
This is why the capacitative reactance is negative and the inductive reactance is positive.

Therefore, when calculating reactances that include \emph{both} inductance and capacitance, you can add the reactances just like you would add resistances.
However, since the capacitive reactances are negative and the inductive reactances are positive, they wind up canceling each other out to some degree.

\begin{exampleprob}
If I have an inductor of $5\mymhy$ and a capacitor of $5\myuf$ in series with a $200\myohm$ resistor, what is the impedance of the circuit for a frequency of $320\myhz$?

To solve this problem, we need to first find the capacitative reactance ($X_C$) and the inductive reactance($X_L$).
To get the capacitative reactance, we use Equation~\ref{eqCapReactance}:

\begin{align*}
X_C &= -\frac{1}{2\pi\cdot f\cdot C} \\
    &= -\frac{1}{2\pi\cdot 320\cdot 0.000005} \\
    &\approx -\frac{1}{0.01} \\
    &\approx -j100\myohm
\end{align*}

The inductive reactance is found using Equation~\ref{eqReactanceInductor}:

\begin{align*}
X_L &= 2\pi\cdot f\cdot L \\
    &= 2\pi\cdot 320 \cdot 0.005 \\
    &\approx j10\myohm
\end{align*}

Now, we can just add these reactances together.

\begin{align*}
X_{total} &= X_L - X_C \\
 &= j10\myohm + -j100\myohm \\
 &= -j90\myohm
\end{align*}

The fact that this is negative is not a problem because it will be squared (which will get rid of the negative) in the next step.
Now that we know the resistance ($200\myohm$) and the reactance ($-j90\myohm$), we just need to use Equation~\ref{eqTotalImpedance} to calculate the total impedance:

\begin{align*}
\textrm{impedance} &= \sqrt{\textrm{resistance}^2 + \textrm{reactance}^2} \\ 
  &= \sqrt{200^2 + (-90)^2} \\
  &= \sqrt{40000 + 8100} \\
  &= \sqrt{48100} \\
  &\approx 219\myohm
\end{align*}

So the total impedance (opposition to current) in this circuit is $219\myohm$.
\end{exampleprob}


\section{Ohm's Law for AC Circuits}

In an AC circuit, the current and voltage are continually varying.
Therefore, using the traditional Ohm's law, you would have to calculate Ohm's law over and over again in order to find out the relationships between voltage, current, and resistance.

However, there is a form of Ohm's law that works directly on AC circuits of a given frequency.
That is, it is essentially a \emph{summary} of the voltages and currents that happen on each cycle.
Ohm's law for AC circuits is basically identical to the previous Ohm's law, but the terms are slightly different:

\begin{equation}
\label{eqOhmsLawAC}
V_{RMS} = I_{RMS} \cdot Z
\end{equation}

Here, the voltage we are referring to ($V_{RMS}$) is an \emph{average} voltage through one cycle of AC.
This average is known as the \glossterm{RMS} average.
It is a little different that the typical average you might think of.
If an AC voltage is swinging back-and-forth from positive to negative, the actual average voltage is probably around zero.
However, RMS voltage is about calculating the average amount of push in any direction---positive or negative.
Therefore, the RMS voltage will always yield a positive answer.\footnote{RMS stands for ``root mean square.'' It is obtained by (a) squaring every data point, (b) averaging the squares, and then (c) taking the square root of the average.  This is why it will be positive---it deals in squares.}

Likewise, the current refers to the RMS current ($I_{RMS}$).
Just like the RMS voltage, the RMS current will always be positive, because it is the measure of the average amount of flow in \emph{any} direction.

Finally, the impedance $Z$ is calculated as we have noted in this chapter---by combining resistances and reactances together into an impedance.

Thus, Ohm's law for AC circuits can be used to express summary relationships about average voltage, average current, and impedance in an AC signal.

\begin{exampleprob}
I have an AC circuit whose RMS voltage is $10\myvolt$.
I have calculated the impedance of this circuit to be $20\myohm$.
What is the RMS current of this circuit?

To find this out, we just rearrange Ohm's law a little bit:

\begin{align*}
V_{RMS} &= I_{RMS} \cdot Z \\
I_{RMS} &= \frac{V_{RMS}}{Z}
\end{align*}

Now I just use the values given to fill in the blanks:

\begin{align*}
I_{RMS} &= \frac{V_{RMS}}{Z}
 &= \frac{10}{20}
 &= 0.5\myamp
\end{align*}

In this circuit we would have an average of half an amp (500 milliamps) flowing through the circuit.
\end{exampleprob}

\begin{exampleprob}
I have an AC voltage source with an RMS voltage of $5\myvolt$ running at $200\myhz$.
It is connected in series with a $50\mykohm$ resistor and a $50\mynf$ capacitor.
What is the current in this circuit?

To find this out, we have to first find the total impedance of the circuit.
That means we have to use Equation~\ref{eqCapReactance} to find the reactance of the capacitor:

\begin{align*}
X_C &= -\frac{1}{2\pi\cdot f\cdot C} \\
    &= -\frac{1}{2\cdot 3.14\cdot 200 \cdot 0.00000005} \\
    &= -\frac{1}{0.00000628} \\
    &\approx -j159236\myohm
\end{align*}

Now that we have the resistance ($50\mykohm$) and the reactance ($-j159236\myohm$) we can combine them with Equation~\ref{eqTotalImpedanceLetters}:

\begin{align*}
Z &= \sqrt{R^2 + X^2} \\
  &= \sqrt{50000^2 + 159236^2} \\
  &= \sqrt{2500000000 + 25356103696} \\
  &= \sqrt{27856103696} \\
  &\approx 166901\myohm
\end{align*}

Now we can use Ohm's law for AC circuits (Equation~\ref{eqOhmsLawAC} to find the current:

\begin{align*}
I_{RMS} &= \frac{V_{RMS}}{Z} \\
        &= \frac{5}{166901} \\
        &\approx 0.00003\myamp
\end{align*}

This means that the average current ($I_{RMS}$) will be approximately $0.00003$ \myamp, or $30\myuamp$.
\end{exampleprob}

\section{Resonant Frequencies of RLC Circuits}

As we have seen, the capacitative reactance goes closer to zero when the frequency goes up.
Likewise, the inductive reactance increases when the frequency goes up.
Additionally, the capacitative reactance and the inductive reactance have opposite signs---negative for capacitative reactance and positive for inductive reactance).

What is interesting is that if you have a combination of inductors and capacitors, there is always some frequency at which their reactances exactly cancel each other out.
This point is known as the \glossterm{resonant frequency} of the circuit.

When a capacitor and an inductor are in series with each other, it is termed an LC series circuit.
Because the inductor inhibits high frequencies and the capacitor inhibits low frequencies, LC circuits can be used to let through a very specific frequency range.
The center of this range is known as the \glossterm{resonant frequency} of the circuit.

Now, if you are good with algebra, you can combine Equation~\ref{eqCapReactance} and Equation~\ref{eqReactanceInductor} to figure out the resonant frequency of a circuit (i.e., set them to add up to zero and then solve for the frequency $f$).
However, to spare you the trouble, there is a formula that you can use to find the resonant frequency of a circuit:

\begin{equation}
\label{eqResonantFrequency}
f = \frac{1}{2\pi\sqrt{L\cdot C}}
\end{equation}

At this frequency, there is no total reactance to the circuit---the only impedance comes from the resistance.

\begin{exampleprob}
Let's say that you have a $20\myuf$ capacitor in series with a $10\mymhy$ inductor.
What is the resonant frequency of this circuit?

To find the resonant frequency, we only need to employ Equation~\ref{eqResonantFrequency}:

\begin{align*}
f &= \frac{1}{2\pi\sqrt{L\cdot C}} \\
  &= \frac{1}{2\pi\sqrt{0.01 \cdot 0.00002}} \\
  &= \frac{1}{2\pi\sqrt{0.0000002}} \\
  &\approx \frac{1}{2\pi\cdot 0.000447} \\
  &\approx \frac{1}{0.00279} \\
  &\approx 358\myhz
\end{align*}

Therefore, this circuit has a resonant frequency of $358\myhz$.  
This means that, at this frequency, this circuit has no reactive impedance.
\end{exampleprob}

Resonance frequencies are important in signal processing.
They can be used in audio equipment to boost the sound of a specific frequency (since all other frequencies will have resistance).
They can be used to select radio stations in radio equipment (since it will be the only frequency allowed through without resistance).
You can also remove a specific frequency by taking a resonant frequency circuit to ground, thereby having a specific frequency short-circuited to ground with no resistance.

\reviewsection

In this chapter, we learned:

\begin{enumerate}
\item Reactance ($X$) is a property of some electronics components that is similar to resistance, but it \emph{prevents} the flow of current instead of \emph{dissipating} the flow (i.e., converting it to heat).
\item Reactance, like resistance, is measured in ohms.  A $j$ is placed in front of the reactance to specify that it is a reactance value.
\item Reactance is frequency-dependent---the amount of reactance depends on the frequency of the signal.
\item Capacitors and inductors each have formulas that can be used to calculate the reactance of the components.  
\item Capacitors year negative reactance and inductors yield positive reactance.  This means that their reactances will oppose and cancel each other out to some degree.
\item Impedance ($Z$) is the total inhibition of the flow of current, combining both resistive and reactive elements.
\item Reactance and resistance are combined into impedance in the same way that walking two different directions can be combined into a total distance from your originating point---using the Pythagorean theorem.
\item RMS voltage ($V_{RMS}$) is the average voltage of an AC circuit, regardless of the direction (positive or negative) of the voltage.  This can be used to summarize the effects of an AC voltage.
\item RMS current ($I_{RMS}$) is the average current of an AC circuit, regardless of the direction (positive or negative) of the voltage.  This can be used to summarize the effects of an AC current.
\item Ohm's law for AC circuits yields the summary relationship between RMS voltage, RMS current, and impedance in an AC circuit.  It is identical to the previous Ohm's law, but uses the summary values for the circuit at a particular frequency, rather than the values at a particular point in time.
\item The resonant frequency of a circuit is the frequency at which inductive reactance and capacitative reactance cancel each other out.
\item Resonant frequencies can be used in any application where isolating a frequency is important, because the resonant frequency will be the only frequency not encountering resistance.
\end{enumerate}

\exercisesection

\begin{enumerate}
\item As the frequency of a signal goes up, how does that affect the reactance from a capacitor?  What about with an inductor?
\item As the frequency of a signal goes down, how does that affect the reactance from a capacitor?  What about with an inductor?
\item What is true about the relationship between the capacitative reactance and the inductive reactance at the resonant frequency?
\item Why is power not used up with reactance?
\item How are reactance and resistance combined to yield impedance?
\item Calculate the capacitative reactance of a $3\myf$ capacitor at $5\myhz$.
\item Calculate the capacitative reactance of a $20\myuf$ capacitor at $200\myhz$.
\item Calculate the inductive reactance of a $7\myhy$ inductor at $10\myhz$.
\item Calculate the inductive reactance of a $8\mymhy$ inductor at $152\myhz$.
\item Calculate the impedance of a circuit with a $200\myohm$ resistor in series with a $75\myuf$ capacitor with a signal of $345\myhz$.
\item Calculate the impedance of a circuit with a $310\myohm$ resistor in series with a $90\mynf$ capacitor with a signal of $800\myhz$.
\item Calculate the impedance of a circuit with no resistor and a $60\mymhy$ inductor with a signal of $89\myhz$.
\item Calculate the impedance of a circuit with a $50\myohm$ resistor in series with a $75\myuhy$ inductor with a signal of $255\myhz$.
\item If I have an AC circuit with an RMS voltage of $6\myvolt$ and an impedance of $1\mykohm$, what is the average (RMS) current of this circuit?
\item If I have an AC circuit, and I measure the AC voltage as $10\myvolt$ RMS, and I measure the AC current at $2\mymamp$ RMS, what is the impedance of this circuit?
\item If I have an $80\myhz$ AC circuit that has an $8\myvolt$ RMS voltage source in series with a $500\myohm$ resistor, a $5\myhy$ inductor, and a $200\mynf$ capacitor, what is the RMS current flowing in this circuit?
\item Calculate the impedance of a circuit with a $250\myohm$ resistor in series with a $87\myuhy$ inductor and a $104\myuf$ capacitor with a signal of $745\myhz$.
\item What is the resonant frequency of the circuit in the previous question?
\item What is the reactance of a circuit at its resonant frequency?
\item If I have a $10\myuf$ capacitor, what size inductor do I need to have a resonant frequency of $250\myhz$?
\end{enumerate}
