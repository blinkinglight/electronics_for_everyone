\chapter{Analog Input and Output on an Arduino}

In Chapter~\ref{chapArduinoProjects}, we learned how to do basic digital input and output with an Arduino using its I/O pins.
In this chapter, we will cover how to do analog input and output as well.

\section{Reading Analog Inputs}

So far we have been focusing on digital input and output---HIGH and LOW states.
The Arduino Uno also supports some amount of analog input (through its analog pins) and a sort of ``faked'' analog output (through its PWM pins, which will be covered in the next section).

On the Arduino Uno, the analog input pins are grouped together in a section labelled ``Analog In.''
These pins are voltage sensors similar to the digital I/O pins, but they can detect a range of values between $0\myvolt$ and $5\myvolt$ (you should never exceed $5\myvolt$ on any Arduino pin).
The \icode{analogRead()} function is similar to the \icode{digitalRead()} function in that it takes a pin number and returns an output value.
The difference is that the pin number given to \icode{analogRead()} corresponds to an analog input pin number, not a digital pin number, and the output, rather than being LOW or HIGH, is a number from 0 to 1023 (10 bits of resolution).
$0\myvolt$ will give you a 0, and $5\myvolt$ will give you 1023.

Because of this, we can rework the darkness sensor we developed in Chapter~\ref{chapIC} to use the Arduino.
Since the photoresistor is just a resistor, we still need to use a voltage divider to convert the resistance value to a voltage.
However, we no longer need the voltage comparator to get it to work---we can just directly connect it to an analog port on the Arduino!

\simplegraphicsfigure{The Darkness Sensor Rebuilt for Arduino}{DarknessSensorArduino}{1}

Figure~\ref{figDarknessSensorArduino} shows the darkness sensor rebuilt for the Arduino.
Notice that we have a lot fewer custom components, because the control has been moved from hardware to the Arduino's software.
We don't need a reference voltage and we don't need a voltage comparator.
We just have one voltage divider to convert the photoresistor's resistance to a voltage (with a wire going to the Arduino's analog input pin~1), and an LED with a current limiting resistor for the output (fed from digital pin~2).
Everything else comes from software.

\begin{typingwithlabel}{Code for the Arduino Darkness Sensor}{figDarknessSensorArduinoCode}
\lstinputlisting{DarknessSensorArduino.cpp}
\end{typingwithlabel}

Figure~\ref{figDarknessSensorArduinoCode} shows the code that will run the sensor.
Note that in the \icode{loop()} function, we are now using \icode{analogRead()} rather than \icode{digitalRead()}. 
Now, instead of it returning HIGH or LOW, it is returning a number.
We can then compare that number to a baseline number to tell us whether we should turn the LED on or off.

Now, you may wonder where I got the value to compare against (i.e., 450). 
What I did was to test the sensor in a variety of conditions, and see which value turned off the light when I wanted it to!

However, you may be wondering what exactly are the values that it is reading.
Thankfully, the Arduino environment allows a way for us to get feedback from the device while it is running, if it is connected to the computer.
To do this, we use the what is known as the \glossterm{serial} interface to the Arduino.
This interface communicates over USB so that we can let our computer know how things are going in the program.

To use the USB serial interface, in your setup function you add the following line:

\icode{Serial.begin(9600);}

This tells the chip to initialize its serial interface at 9600 baud (\glossterm{baud} is an old term meaning ``bits per second''), which allows us to talk back to the computer.
However, it is important to note that if you use the \icode{Serial} functions, you should not have anything connected to Digital Pin~0 or Pin~1 of the Arduino.

Now, in your program, you can do \icode{Serial.println()} to output any value you want.
We will do

\icode{Serial.println(analogRead(1));}

to let us know what the current value of the analog input is reading at.
The new program, with the added feedback, is shown in Figure~\ref{figSimpleArduinoButtonLEDWithSerialCode}.
After uploading this code to the Arduino, to see your output, click on the magnifying glass on the top right of the screen.  
You can also go to the ``Tools'' menu and select ``Serial Monitor.''
Either way gets you to the same screen.
When the code is running, it should be spewing out pages and pages of numbers.
Each of these numbers is the current value of \icode{analogRead()} when it is encountered in the code, which happens hundreds or thousands of times each second (it is slowed down a bit by the USB communication).

\begin{typingwithlabel}{Darkness Sensor with Serial Feedback}{figSimpleArduinoButtonLEDWithSerialCode}
\lstinputlisting{DarknessSensorArduinoSerial.cpp}
\end{typingwithlabel}

\section{Analog Output with PWM}

So, we have discussed analog \emph{input}, but what about analog \emph{output}.
Truthfully, the Arduino does not support analog output as such.
However, analog output is \emph{faked} on an Arduino using a technique known as \glossterm{pulse-width modulation}, abbreviated as \glossterm{PWM}.
The Arduino only outputs $5\myvolt$ on its output pins.
But, let's say we wanted to fake a $2.5\myvolt$ signal.
What might we do?
Well, if we turned the pin on and off rapidly so that it was only on for half the time, that would give us about the same amount of electricity as $5\myvolt$.
That's what PWM does---it fakes lower voltages by just flipping the power to the pin on and off very rapidly so that it ``looks'' like a lower voltage.

Arduino programs use the function \icode{analogWrite()} to use a pin for PWM.
This function is a little confusing because, (a) it uses digital pins, not analog pins, and (b) the value is between 0 and 255, not 0 and 1023 like \icode{analogWrite()}.
Other than that, it basically does what you might expect.
\icode{analogWrite(3, 0);} will turn off Digital Pin~3, \icode{analogWrite(3, 255);} will turn it all the way on, \icode{analogWrite(3, 127);} will flick it on and off pretty evenly, and \icode{analogWrite(3, 25);} will keep Pin~3 on only a short time relative to how long the pin stays off.

\simplegraphicsfigure{A Simple Analog Dimmer}{AnalogDimmer}{1}

To get a flavor for PWM, we will do a very simple PWM project---a dimmed LED.
Figure~\ref{figAnalogDimmer} shows what the connection will look like.
Just an LED with a current limiting resistor attached to Digital Pin~3 (which is marked with an \icode{~} to indicate that it is capable of PWM).
Figure~\ref{figAnalogDimmerCode} shows the code to dim the output.

\begin{typingwithlabel}{Code for the Analog Dimmer}{figAnalogDimmerCode}
\lstinputlisting{AnalogDimmer.cpp}
\end{typingwithlabel}

This code is a little more complicated.
It's alright if you don't understand it fully.
In short, it creates a \glossterm{variable}, which is a named temporary storage location for a value (we are calling it \icode{i} for a short name).
It is declared an \icode{int}, which means it will hold an integer, and we set it with a starting value of zero.

The \icode{while} loop executes everything within the block of code between \icode{\{} and \icode{\}} over and over again, as long as \icode{i} is less than \icode{255}.
Within this block, we write the value of \icode{i} to pin \icode{3} using \icode{analogWrite()}.
Then, we delay for 10~milliseconds to make sure it is visible.
Then, we increase \icode{i} by one to go to the next value.

The next \icode{while} loop does the same thing but goes the other way.
It starts at \icode{255} and progresses down to~\icode{0}.
Then, when it is all the way to zero, it runs the \icode{loop} function over again.

Even though this code is running by switching the pin on and off at different rates, it \emph{looks like} the LED is dimming on and off.
It is flickering so fast that we merely perceive it as a lower-energy light than as a pulsing light.
In fact, when it gets to about 180 (about 70\% on and 30\% off), there is not a lot of difference between that and full brightness.
