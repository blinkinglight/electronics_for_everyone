\chapter{Voltage and Resistance}
\label{chapVoltageResistance}

In the previous chapter we learned about current, which is the rate of flow of charge.
In this chapter we are going to learn about two other fundamental electrical quantities---\glossterm{voltage} and \glossterm{resistance}.
These two quantities are the ones that are usually the most critical to building effective circuits.

Current is important because limiting current allows us to preserve battery life and protect precision components.
Voltage, however, is usually the quantity that has to be present to do any work within a circuit.

\section{Picturing Voltage}

What is voltage?
Voltage is the amount of power each coulomb of electricity can deliver.
If you have a one coulomb of electricity at 5 volts and I have one coulomb of electricity at 10 volts, that means that my coulomb can deliver twice as much power as yours.

A good analogy to electronics is the flow of water.
When comparing water to electricity, \emph{coulombs} are a similar unit to \emph{liters}---coulombs measure the amount of electrical charge present just like a liter is the amount of water stuff present.
Both charge and water both flow.
In water, we can measure the flow of a current of a stream in liters-per-second.
Likewise, in electronics, we measure the flow of charge through a wire in coulombs-per-second, called amperes.

Now, I want you to image the end of a hose containing water.
Normally, the water just falls out of the hose, especially if the hose is just sitting on the ground.
That hose just sitting on the ground is like a current with zero volts---each unit of water or charge is just not doing that much.

Let's pretend we added a spray nozzle to the hose.
What happens now?
Water shoots out of the nozzle.
We haven't added any more water---it is actually the same amount of current flowing.
Instead, we increased the pressure on the water, which is just like increasing the voltage on an electric charge.
By increasing the pressure, we changed the amount of work that each liter of water is available to perform.
Likewise, when we increase voltage, we change the amount of work that each coulomb of electricity can do.

One way we might measure the pressure of water coming out of a hose is to measure how far up it can shoot out of the hose.
By doubling the pressure of the water, we can double how far out of the hose it can shoot.
Similarly, with voltages, large enough voltages can actually jump air gaps across circuits.
However, to do this, it takes a lot of voltage---about 30,000 volts per inch of gap.
If you have been shocked by static electricity, though, this is what is happening!
The power of the charge was extreme (thousands of volts), but the amount of charge in those shocks are so small that it doesn't harm you (about 0.00000001 coulombs).

\section{Volts are Relative}

While charge and current are fairly concrete ideas, voltage is a much more relative idea.
You can actually never measure voltage absolutely.
All voltage measurements are actually relative to other voltages.
That is, I can't actually say that my electric charge has exactly 1, 2, 3, or whatever volts.
Instead, what I have to do is say that one charge is however many volts more or less than another charge.
So, let's take a 9-volt battery.
What that means is not that the battery is 9 volts in any absolute sense, but rather that there is a 9-volt \emph{difference} between the charge at the positive terminal and the charge at the negative terminal.
That is, the pressure with which charge is trying to move from the positive terminal to the negative terminal is 9 volts.

\section{Relative Voltages and Ground Potential}

When we get to actually measuring voltages on a circuit, we will only be measuring voltage \emph{differences} on the circuit.
So, I can't just put a probe on one place on the circuit, I have to put my probe on two different places on the circuit and measure the voltage difference (also called the \glossterm{voltage drop}) between those two points.

However, to simplify calculations and discussions, we usually choose some point on the circuit to represent ``zero volts.''
This gives us a way to standardize voltage measurements on a circuit, since they are all given relative to the same point.
In theory this could be any point on the circuit, but, usually, we choose the negative terminal on the battery to represent zero volts.

This ``zero point'' goes by several names, the most popular of which is \glossterm{ground} (often abbreviated as \glossterm{GND}).
It is called the ground because, historically, the physical ground has often been used as a reference voltage for circuits.
Using the physical ground as the zero point allows you to also compare voltages between circuits with different power supplies.
However, in our circuits, when we refer to the ground, we are referring to the negative terminal on the battery, which we are designating as zero volts.
If we designate any other part of the circuit as a ground, we will let you know.  % FIXME - need to pull or clarify this depending on what is in the final version of the book

Another, lesser-used term for this designated zero volt reference is the \glossterm{common} point.
Many multimeters label one of their electrodes as \glossterm{COM}, for the common electrode.
When analyzing a circuit's voltage, this electrode would be connected to whatever your zero-volt point is.

This ``ground'' analogy also makes sense with our water hose analogy.
Remember that a voltage is the potential for a charge to do work.
What happens to water after it lands on the ground?
By the time the water from my hose lands on the ground, it has lost all its energy.
It is just sitting there.
Sure, it may seep or flow around a bit, but nothing of consequence.
All of its ability to do work---to move quickly or to knock something over---has been drained.
It is just on the ground.
Likewise, when our electric charge is all puttered out, we say that it has reached ``ground potential.''

So, even though we could designate any point as being zero, we usually designate the negative terminal of the battery as the zero point, indicating that by the time electricity reaches that point, it has used up all of its potential energy---it now has zero volts.

\section{Resistance}

Resistance is how much a circuit or device resists the flow of current.
Resistance is measured in \glossterm{ohms}, and is usually represented by the symbol \si{\ohm}.
Going back to our water hose analogy, \glossterm{resistance} is how small the hose is.
Think about a 2-liter bottle of pop.
The bottle has a wide base, but the opening is small.
If I turn the bottle upside down, the small opening limits the amount of liquid that flows out at one time.
That small opening is giving \emph{resistance} to the flow of liquid, making it flow more slowly.
If you cut off the small opening, leaving a large opening, the liquid will come out much faster because there is less resistance.

Ohm's law, which we will use throughout this book, tells us about the relationship between resistance, voltage, and current flow.
The equation is very simple.
It says:

\begin{equation}
\label{ohmequationv}
V = I * R
\end{equation}

In this equation, V stands for voltage, I stands for current (in \emph{amperes}, not milliamperes), and R stands for resistance (in ohms).
To understand what this equation means, let's think again about water hoses.
The water the comes out of the fawcet of your house has essentially a constant current.
Therefore, according to the equation, if we add resistance, it will increase our voltage.

We know this to be true from experience. 
If we have a hose and just point it forward, water usually comes out about a foot or two.
Remember, voltage is how much push the water has, which determines how far the water will go when it leaves the hose.
However, if my children are on the other side of the yard, and I want to hit them with a water spray, what do I do?
I put my thumb over the opening.
This increases the resistance, and, since the current is constant, the voltage (the distance the water will travel after it leaves the hose) will increase.

However, in circuits, we usually don't have a constant current source.
Instead, batteries provide a constant voltage source.
A 9-volt battery will provide 9-volts in nearly every condition.
Therefore, for electronics work, we usually rearrange the equation a little bit.  
Using a little bit of algebra, we can solve our equation for either current or resistance, like this:

\begin{equation}
\label{ohmequationi}
I = V / R
\end{equation}

\begin{equation}
\label{ohmequationr}
R = V / I
\end{equation}

Equation~\ref{ohmequationi} is the one that is usually most useful.
To understand this equation, think back to the example of the bottle turned upside down.
There, the liquid has a constant amount of push/voltage (from gravity), but we had different resistances.
With the small opening, we had a large resistance, so the water came out slower.
With the large opening, we had almost no resistance, so the water came out all at once.

\begin{exampleprob}
Let's put Ohm's law to use.
If I have a 5-volt voltage source with 10 ohms of resistance, how much current will flow?
Since we are solving for current, we should use equation~\ref{ohmequationi}.
This says $I = V / R$.
Therefore, plugging in our voltage and resistance, we have $I = 5 / 10$, which is $I = 0.5 \textrm{amperes}$ (remember, Ohm's law always uses amperes for current).
Note that, in this book, we will never use fractions when we solve problems, we will only use decimals.
\end{exampleprob}

\begin{exampleprob}
Now let's say that we have a 10 volt source, and we want to have 2 amps worth of current flowing.
How much resistance do we need in order to make this happen?
Since we are now solving for resistance, we will use equation~\ref{ohmequationr}, which says $R = V / I$.
Plugging in our values, we see that $R = 10 / 2 = 5 \si{\ohm}$.
Therefore, we would need $5 \si{\ohm}$ of resistance.
\end{exampleprob}

\begin{exampleprob}
Now let's say that I have a 9-volt source and I want to limit my current to 10 \emph{milliamps}.  
This uses the same equation, but the problem I have is that my units are in milliamps, but my equation uses amps.
Therefore, before using the equation, I have to convert my current from milliamps to amps. 
Remember, to convert milliamps to amps, we just divide by 1000.
Therefore, we take 10 milliamps and divide by 1000, we get $0.010$ amps.
Now we can use equation~\ref{ohmequationr} to find the resistance we need.
$R = V / I = 9 / 0.010 = 900 \si{\ohm}$.
Therefore, with 900 \si{\ohm} of resistance, we will limit our current to 10 milliamps.
\end{exampleprob}

\reviewsection

In this chapter, we learned:

\begin{enumerate}
\item Voltage is the amount of power that each unit of electricity delivers.
\item The volt is the electrical unit that we use to measure voltage.
\item Voltage is always given relative to other voltages---it is not an absolute value.
\item The ground of a circuit is a location on the circuit where we have chosen to use as a universal reference point---we define that point as having zero voltage for our circuit to make measuring other points on our circuit easier.
\item In DC electronics, the chosen ground is usually the negative terminal of the battery.
\item Other terms and abbreviations for the ground include common, GDN, and COM.
\item Resistance is how much a circuit resists the flow of current and is measured in ohms (\si{\ohm}).
\item Ohm's law tells us the relationship between voltage, current, and resistance: $V = I * R$.
\item Using basic algebra, we can rearrange ohm's law in two other ways, depending on what we want to know.  It can be solved for current, $I = V /R$, or it can be solved for resistance, $R = V / I$.
\end{enumerate}

\applysection


\begin{enumerate}
\item 
\question{If I have a 4-volt battery, how many volts are between the positive and negative terminals of this battery?}
\solution{$4\myvolt$}
\explanation{The definition of a battery's voltage is the number of volts between the positive and negative terminals.}
\item 
\question{If I choose the \emph{negative} terminal of this battery as my ground, how many volts are at the \emph{negative} terminal?}
\solution{$0\myvolt$}
\explanation{Because volts are relative units, a point must be chosen as the ``zero-volt'' level.  This is known as the ground.  Therefore, if the negative terminal is the ground, it is, by definition, zero volts.}
\item 
\question{If I choose the \emph{negative} terminal of this battery as my ground, how many volts are at the \emph{positive} terminal?}
\solution{$4\myvolt$}
\explanation{On a $4\myvolt$ battery, the positive terminal is by definition 4 volts above the negative terminal.  If the negative terminal is the ground (i.e., the zero point), then 4 volts above that will be $4\myvolt$.}
\item 
\question{If I choose the \emph{positive} terminal of this battery as my ground, how many volts are at the \emph{negative} terminal?}
\solution{$-4\myvolt$}
\explanation{On a $4\myvolt$ battery, the positive terminal is by definition 4 volts above the negative terminal. If the positive terminal is the ground (i.e., the zero point), then that is 4 volts above the negative terminal.  That must mean that the negative terminal is 4 volts below zero, or $-4\myvolt$.}
\item 
\question{Given a constant voltage, what effect does increasing the resistance have on current?}
\solution{The current will decrease.}
\explanation{Because $V = I \times R$, with constant voltage increasing the resistance will reduce the current.}
\item 
\question{Given a constant current, what effect does increasing the resistance have on voltage?}
\solution{The voltage will increase.}
\explanation{Because $V = I \times R$, if the current is kept constant, increasing the resistance will increase the voltage.}
\item 
\question{If I have a $10\myvolt$ battery, how much resistance would I need to have a current flow of 10 amps?}
\solution{$1\myohm$}
\explanation{Because we are looking for resistance, we can use Equation~\ref{ohmequationr}.
\begin{align*}
R &= V / I \\
  &= 10 / 10 \\
  &= 1
\end{align*}
We would need a $1\myohm$ resistance.
}
\item 
\question{If I have a 3-volt battery, how much resistance would I need to have a current flow of 15 amps?}
\solution{$0.2\myohm$}
\explanation{We can use Equation~\ref{ohmequationr} to find how much resistance we need:
\begin{align*}
R &= V / I \\
  &= 3 / 15 \\
  &= 0.2
\end{align*}
We would need a $0.2\myohm$ resistance.
}
\item 
\question{Given 4 amps of current flow across 200 ohms of resistance, how much voltage is there in my circuit?}
\solution{$800\myvolt$}
\explanation{This can be solved using Equation~\ref{ohmequationv}:
\begin{align*}
V &= I \times R \\
  &= 4 \times 200 \\
  &= 800
\end{align*}
This circuit has 800 volts.
}
\item 
\question{If I am wanting to limit current flow to 2 amps, how much resistance would I need to add to a 40-volt source?}
\solution{$20\myohm$}
\explanation{This problem can be solved using Equation~\ref{ohmequationr}:
\begin{align*}
R &= V / I \\
  &= 40 / 2 \\
  &= 20
\end{align*}
You would need to add $20\myohm$ of resistance to that source to limit the current flow.
}
\item 
\question{If I am wanting to limit current flow to 2 milliamps, how much resistance would I need to add to a 9-volt source?}
\solution{$4,500\myohm$}
\explanation{Since Ohm's law only works for amps, we need to first convert milliamps to amps:
$$ 0.001 \frac{\myamp}{\mymamp} \times 2 \mymamp = 0.002 \myamp $$
Now we can use Equation~\ref{ohmequationr} to find the resistance we need:
\begin{align*}
R &= V / I \\
  &= 9 / 0.002 \\
  &= 4500
\end{align*}
We would need to add 4500 ohms of resistance to this source to limit the current flow.
}
\item
\question{If I am wanting to limit current flow to 20 milliamps, how much resistance would I need to add to a 5-volt source?}
\solution{$250\myohm$}
\explanation{Since Ohm's law only works for amps, we need to first convert milliamps to amps:
$$ 0.001 \frac{\myamp}{\mymamp} \times 20 \mymamp = 0.02 \myamp $$
Now we can use Equation~\ref{ohmequationr} to find the resistance we need:
\begin{align*}
R &= V / I \\
  &= 5 / 0.02 \\
  &= 250
\end{align*}
We need to add 250 ohms of resistance to this voltage source to limit the current.
}
\end{enumerate}

