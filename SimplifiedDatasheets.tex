\chapter{Simplified Datasheets for Common Devices}
\label{appSimplifiedDatasheets}

Datasheets are what electronics manufacturers use to communicate the specifications of their components to the engineers who will use them in projects and products.
However, datasheets tend to be horrendously complex.
Typical datasheets run about 10--30 pages, and most of that is completely useless for someone just trying to get a circuit to work.
Sometimes the first pages of the datasheet are all of the different packages the chip is available in, and the pages detailing what the component \emph{actually does} are almost in the very back. 
I've even seen many sheets that never speak of what a device is for or why you would want to use it.

So, to simplify your life, I have created some simplified datasheets.
All of these are single-page sheets, and focus on the typical component styles used in solderless breadboards.
These datasheets are also oriented towards \emph{generic} parts. 
Each manufacturer has their own parts with their own specs and their own benefits and drawbacks.
Some manufacturers may have higher or lower current specifications, faster or slower switching times, or other variations.
These datasheets should be used as a starting point, but not as a final authority.

Use these datasheets to find the component you want, find its general characteristics, and what each of the pins are for and what they should be hooked up to.
However, the manufacturer's datasheet should be consulted for final specifications.
