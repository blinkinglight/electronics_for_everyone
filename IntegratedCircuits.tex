\chapter{Integrated Circuits and Resistive Sensors}
\label{chapIC}

So far, the components we have studied are simple, basic components---batteries, resistors, diodes, etc.
In this chapter, we are going to start to look at \glossterm{integrated circuits}, also called \glossterm{chips}, \glossterm{microchips}, or \glossterm{IC}s.
An IC is a miniaturized circuit placed on silicon.
It is a whole collection of parts geared around a specific function.
These functions may be small, such as comparing voltages or amplifying voltages, or they may be complex, such as processing video or even complete computers.
A single chip may hold just a few components, or it may hold billions.

Miniaturized circuits have several advantages---they are cheaper to produce in mass, they use less power, and they take up less space in your overall circuit---all because they have a reduced area and use fewer materials.
These miniaturized circuits are what allowed for the computer revolution over the last century.

\section{The Parts of an Integrated Circuit}

Integrated circuits, as we have noted, are basically miniaturized circuits placed on a siliconplate, called the die.
This die is where all of the action of the integrated circuit takes place.

The die is then placed into a \glossterm{package}, which then provides connection points for circuit designers to interface with the IC.
These connection points are often called \glossterm{pins} or \glossterm{pads}.
Each pin on an IC is numbered, starting with pin 1 (we will show you how to find pin 1 shortly).
Knowing which pin is which is important, because most of pins on a chip each have their own purposes, so if you attach a wire to the wrong pin your circuit won't work, or you will destroy the chip.
Most packages are marked with the chip's manufacturer and part number.

There are many different types of packaging available, but there are two general types that are often encountered:

\begin{description}
\item[Through-Hole] In this packaging type, the connection points are long pins which can be used on a breadboard.  This type of packaging is easiest for amateur usage.
\item[Surface Mount]  In this packaging type, the connection points are small pads which are meant to be soldered to a circuit board.  This packages are much smaller (and therefore less expensive), and can be more easily managed by automated systems.  These are also referred to as SMD (surface mount devices) or SMT (surface mount technology).
\end{description}

\simplegraphicsfigure{Comparison of the Same IC in SMD (left) and DIP (right) Packages}{DIPAndSMD}{0.125}

Since we are only using breadboards in this book and not doing any soldering, we will only concern ourselves with through-hole packaging.
However, through-hole packaging itself comes in a variety of styles.
The main one we will concern ourselves with is called a \glossterm{dual in-line package}, or \glossterm{DIP}.  

\simplegraphicsfigure{Pin 1 is Immediately Counterclockwise of the Notch}{FindPin1}{0.125}

An Integrated Circuit in a DIP package has two rows of pins coming out of the package.
Most chips mark either the top of the chip with a notch or indentation (where pin 1 is immediately counterclockwise of the notch), or mark pin 1 with an indentation, or both. 
See Figure~\ref{figFindPin1} to see how to use the notch to find pin 1.
The rest of the pins are numbered counterclockwise around the chip.

\simplegraphicsfigure{A DIP IC Inserted Into a Breadboard}{ChipInBreadboard}{0.125}

The beauty of a DIP packaged IC is that it fits perfectly onto most breadboards.
Figure~\ref{figChipInBreadboard} shows how you can place your IC across the breadboard's bridge and each pin on the chip will have its own terminal strip to connect to.

Be careful, though, when inserting ICs into breadboards.
The pins on an IC are often slightly wider than the breadboard.
If you just jam the IC into the breadboard, you will likely accidentally crush one or more of the pins that aren't exactly aligned on the hole.
Instead, compare the width of the pins to the width it has to fit in on your breadboard.
If it doesn't match up, \emph{very gently} bend the pins with your fingers or with pliers to get them to match up.

Usually, the ICs that I purchase are just a little wide, and I will squeeze the pins on each side slightly between my thumb and finger until they move close enough together.
However, you adjust the pins, make sure they line up before pushing them into their connection points on the breadboard.
Also, with larger ICs, you may need to adjust the IC back and forth as you gently insert it into place on the breadboard.

\section{The LM393 Voltage Comparator}

There are thousands and thousands of available chips which do a dizzying array of functions.
In this chapter, we are going to focus on a very simple chip---the LM393 Voltage Comparator.
This chip does one simple task.
The LM393 compares two input voltages, and then outputs either a high-voltage signal or a low-voltage signal depending on which input voltage is greater.
The LM393 is actually a \emph{dual} voltage comparator, which means that it will do two separate comparisons on the same chip.
Like most chips, the LM393 is an \emph{active} device, which means that it additionally requires a voltage source and a ground connection to provide power to the device.

\simplegraphicsfigure{The Pin Configuration of an LM393}{lm393Pinout}{0.25}

Figure~\ref{figlm393Pinout} shows the pin configuration (also called the \glossterm{pinout}) of the LM393.
The first thing to note on any pinout is where the voltage and ground connections are.
In this case, the voltage is marked as $V_{CC}$ and the ground is marked as $GND$.
Even though the LM393 has \emph{two} voltage comparators on the chip, they both share the power ($V_{CC}$) and ground ($GND$) pins.
The left side of the chip diagram shows the inputs and output for the first voltage comparator ($1IN+$, $1IN-$, and $1OUT$), and the inputs and output for the second voltage comparator is on the right ($2IN+$, $2IN-$, and $2OUT$).
In your projects, you can use whichever one is more convenient for you, or even both at the same time.

So, the \icode{1IN+} pin (pin 3) and the \icode{1IN-} pins are where the two voltages are being fed that are being compared by the first comparator.
The \icode{1OUT} is the pin which will contain the output.
If the voltage at \icode{1IN+} is less than the voltage on \icode{1IN-}, the output pin will be at a low (i.e., zero/ground) voltage.
If the voltage at \icode{1IN+} is greater than the voltage on \icode{1IN-}, the output pin \emph{will not conduct at all}, but this will be considered a ``high'' (positive-voltage) state.
This sounds counterintuitive, but, as we will see, this lets us set our own output voltage to whatever we want without causing too much complexity.
This configuration where high-voltage outputs don't conduct is called an \glossterm{open collector} configuration.
Don't worry if this is a little confusing, we will discuss it more in-depth later in the chapter.

\begin{sidebar}[Voltage Sources on Integrated Circuits]
Note that the voltage pins on integrated circuits can be marked in a number of different ways.
The positive voltage source is often labelled as $V_CC$, $V_DD$, or $V_+$.
The ground connect is often labelled as GND, $V_EE$, $V_SS$, or $V_-$.
There are additional ways that these are labelled as well.
Finding the positive and ground connections for an IC should always be the first thing you do with them.
\end{sidebar}

\section{The Importance and Problems of Datasheets}

Every IC (and, usually, any other part as well) has a datasheet supplied by the manufacturer which tells you important details about how you should use their chip in your circuit.
Reading datasheets is one of the worst parts of electronics, in my opinion. 
For me, datasheets rarely have the information I am actually looking for in a way that is easy to find.

In fact, most datasheets assume that you already know how to use the device, and the datasheets are just there to supply additional details about the limitations of the device.
For instance, looking through the LM393 datasheet from Texas Instruments, the actual operation of the device isn't even listed until page 11, and there it is buried within a sub-subsection, almost as a side-note.

These datasheets are written by people who have spent a lot of time being electrical engineers, and they are written for people who have spent a lot of time being electrical engineers, so when mere mortals read the datasheets, the important pieces are often shrouded in unintelligible gibberish.
For instance, the fact that the ``high'' output state of the device doesn't conduct isn't mentioned explicitly anywhere at all in the datasheet.  
Instead, it is implied by the configuration.

The reason for this is that the datasheets are usually read by professionals familiar with the type of device, and just need to know the electrical details so they don't accidentally bend the device beyond the breaking point.
Thus, the datasheets oftentimes spend more time just showing and describing the layout of the circuit on the chip and graphs of different chip properties, and then you are left to interpret what that means for your circuit.
For advanced circuit designers, this is great.
For students and hobbyists, however, this is oftentimes more frustrating than helpful.

However, datasheets do often provide a few basic details that are helpful to everyone.
They will often tell you:
\begin{itemize}
\item What each pin does
\item What the power requirements are
\item What the outer limits of the chip's operation are
\item An example circuit that you can build with the device
\end{itemize}

For all of these reasons, Appendix~\ref{appSimplifiedDatasheets} contains simplified datasheets for a number of common devices that are easier to read than the standard ones. 

For the LM393, the important points are:

\begin{enumerate}
\item The input voltage on $V_{CC}$ can be anywhere between $2\myvolt$ and $36\myvolt$.
\item When sensing voltage, the LM393 doesn't really draw any (or at least much) current, so there are no parallel resistances we need to worry about.
\item The output is high when \icode{IN+} is greater than \icode{IN-}, and low (i.e., ground) when \icode{IN+} is less than \icode{IN-}, with an error range of about 2 millivolts.
\item When the output is low, the output pin will conduct current into itself (since it is at ground, positive charge will naturally flow into it), but if sink more than $6\mymamp$ into it, you will destroy it.
\item When the output is high, the device will not conduct any current.
\end{enumerate}

That isn't to say that the datasheets aren't important, but for a beginner the datasheets usually aren't what you need to get started.

\section{A Simple Circuit with the LM393}

In this section, I am going to show a simple circuit using the LM393 chip.
In doing so, we are going to be using several of the resistor circuit patterns that we learned in Chapter~\ref{chapBasicResistorCircuits}.

\simplegraphicsfigure{A Simple Comparator Circuit}{SimpleComparatorCircuit}{0.08}

The circuit we are discussing is shown in Figure~\ref{figSimpleComparatorCircuit}.
Can you identify the resistor circuit patterns?  
Take a minute and see if you can find some.
Note that the wire coming out of \icode{1IN-} crosses two wires that it is \emph{not} joined with.

The first thing to notice is that we have \emph{two} voltage dividers.
The first voltage divider is between R1 and R2.
Since R1 and R2 are the same resistance and are connected to both $9\myvolt$ and $0\myvolt$, that means that they divide the voltage in half, giving a $4.5\myvolt$ output.
The second voltage divider is between R3 and R4.
Since R3 is half of the resistance of R4, that means that it only uses up half as much voltage as R4.  
Thus, since R3 eats up $3\myvolt$ and R4 eats up $6\myvolt$, the wire coming out from the middle is at $6\myvolt$.

Then, to the right of the circuit, you can see that we have a current-limiting resistor in front of the LED.
That is not its only function, though.
It also functions, as we will see shortly, as a pull-up resistor.

So what is the big triangle?
Comparators (and several other circuits commonly placed on ICs) are represented as triangles in the schematic (we could have also placed the chip itself there).
Each of the connections are labelled the same as they are labelled in the pinout diagram in Figure~\ref{figlm393Pinout} so they would be easy to locate.

The way that the circuit works is very simple.
The voltage coming in to \icode{1IN+} is $6\myvolt$ and the voltage coming in to \icode{1IN-} is $4.5\myvolt$.
Since \icode{1IN+} is greater than \icode{1IN-} then that will turn \icode{1OUT} to high (positive voltage).
However, remember that we said that \icode{1OUT} \emph{does not conduct} when it is high.
It acts like an open switch.
Therefore, R5 acts like a pull-up resistor and supplies the positive voltage for us to our LED to turn it on.

Now let's say that the input voltages were reversed.
What would happen?
If \icode{1IN-} is greater than \icode{1IN+}, then \icode{1OUT} will go low (zero volts) and also conduct.
It will act like a closed switch going to ground.

Therefore, current will go the easy route - it will go through \icode{1OUT} (directly to ground) instead of through the LED ($1.8\myvolt$ or more).
This works just like the switch in the circuit in Section~\ref{secPullUpResistor}.
When \icode{1OUT} is low, it acts like a closed switch to ground.
When \icode{1OUT} is high, it acts like an open switch (and is whatever positive voltage you supply yourself).

The resistor R5 does several jobs.
The first job is to act as a pull-up resistor. 
Remember that a pull-up resistor prevents the load to ground from going too high when the switch is closed.
Without the pull-up resistor, when the switch is closed (\icode{1OUT} goes low), we would have a short-circuit from the voltage source to ground.
This would not only waste a large amount of electricity, it would break the LM393, because it can only sink a maximum of $6\mymamp$ of current.
Having a $3\mykohm$ resistor, we limit the current for the closed switch to $I = V / R = 9 / 3,000 = 0.003A = 3\mymamp $.

When the switch is open, the current flows through the resistor to the LED, and then the resistor acts as a current-limiting resistor for the LED.
The amount of current to the LED will be calculated as $I = V / R = (9 - 1.8) / 3,000 = 7.2 / 3,000 = 0.0024\myamp = 2.4\mymamp$.

\section{Resistive Sensors and Voltages}

One of the more practical uses of the voltage comparator circuit is to measure the values of sensors which act as variable resistors.
Many different materials in the world act as resistors.
What's really interesting is that many of these materials \emph{change their resistance} depending on external factors.
Some of them change their resistance based on temperature, pressure, light, humidity, and any number of other environmental factors.

Now, changing resistance doesn't tell us much by itself.
If we put a resistor between a voltage source and ground, it will always eat up that voltage source.
However, if you use it in concert with a fixed resistor to make a voltage divider, you can then get the output voltage to vary based on the changes in resistance.

\simplegraphicsfigure{A Simple Resistor Sensor Circuit}{SimpleResistiveSensorCircuit}{0.08}

Figure~\ref{figSimpleResistiveSensorCircuit} illustrates this principle.
It is a simple voltage divider, where the top resistor is actually a photoresistor (a resistor that varies based on light) and the bottom resistor has a fixed resistance.
Thus, as the light varies, the top resistance will vary.
This will change the ratio between the top and bottom resistor, which will affect the output voltage.

To use this circuit, you will need to know the resistances of your photoresistor on the different conditions you are interested in.
I usually use the GL5528, which ranges from $10\mykohm$ in bright light to $1\myMohm$ in complete darkness.
However, depending on your specific photoresistor as well as the light conditions that you think of as ``light'' and ``dark,'' the resistance values that are relevant for light and dark will be different for you.
So, whatever photoresistor you use, it is worthwhile to measure the resistance using your multimeter in the different conditions you think of as light and dark.

\section{Sensing and Reacting to Darkness}

So far in the book, we have focused entirely on example circuits that didn't really do anything.
They lit up, they had voltage and current, but there wasn't much interesting that they were doing.
However, now, we finally have enough knowledge to start building circuits that \emph{do} something.

We have:
\begin{enumerate}
\item A way to generate a fixed voltage (using a voltage divider)
\item A way to generate resistances from real-world events (photoresistors and other resistance sensors)
\item A way to convert changes in resistances to changes in voltage (using a voltage divider with one fixed resistor)
\item A way to compare our varying voltage to our fixed voltage (using the LM393 comparator)
\item A way to utilize the output signal from the LM393 to do work (using the pull-up resistor and the LED)
\end{enumerate}

There are a lot of pieces to put together this simple circuit, which is why it has taken so long to do anything worthwhile. 
However, if you have followed along carefully, now that you are here you should be able to see how all of this fits together.

What we will do is to take the circuit given in Figure~\ref{figSimpleComparatorCircuit} and modify R4 to be our photoresistor and R3 to be a fixed resistor.
In my own testing, I discovered that the light/dark switchover point for my photoresistor was about $15\mykohm$.
Therefore, I am going to use a $15\mykohm$ resistor as the fixed resistor for R3.
Yours may need to vary based on your experimentation with your photoresistor.

When the light is on, my photoresistor will have a lower resistance than $15\mykohm$, which will make the fixed resistor R3 use up more of the voltage.  
Thus, the voltage at the divider will be less than $4.5\myvolt$, which will turn \icode{1OUT} to low (which closes the switch and makes a path to ground on the output before it gets to the LED, which turns the LED off).

In low-light conditions, the resistance will jump way up above the resistance of the fixed resistor.
If the upper, fixed resistor has less resistance than the bottom resistor, then the voltage at the divider will be larger than $4.5\myvolt$, activating the comparator and turning \icode{1OUT} to high (i.e., opening the switch and allowing power to flow through the LED).

The final circuit is given in Figure~\ref{figDarknessSensorCircuit}.
You can see a way to lay it out on the breadboard in Figure~\ref{figDarknessSensorBreadboard}.

\simplegraphicsfigure{Darkness Sensor Schematic}{DarknessSensorCircuit}{0.08}

\simplegraphicsfigure{Darkness Sensor Breadboard Layout}{DarknessSensorBreadboard}{1}

\section{Sources and Sinks}

Two terms that often come up when dealing with circuits are the concepts of a current \glossterm{source} and a current \glossterm{sink}.
A source is a component whose pins might provide current to other parts of the circuit.
A sink is a component whose pins might pull current from other parts of the circuit.

For the LM393, its input pins neither source nor sink current (at least not any significant amount).
The input pins more-or-less just sense the voltage without pulling any measurable current.
Therefore, they are neither sources nor sinks of current.
Technically, they probably sink a few nanoamps (billionths of an amp), but not nearly enough to affect our circuit analysis.

The output pin, even though it is called an \emph{output}, doesn't source any current.  
Instead, it acts either as a sink (when low) or as a disconnected circuit (when high).
This is known as an \glossterm{open collector} output.

Anytime an IC sources or sinks current, be sure to read the datasheets on the maximum amount of current it can source or sink.
These are usually quantities that \emph{you} have to limit---they are merely telling you at what point their circuit will physically break.
Therefore, you must use resistors to limit the currents to make sure that they are within limits.

However, be aware that many (but certainly not all) ICs do not source current, using open collectors for their output operations.
This has the disadvantage that you have to supply your own voltage and pull-up resistor to the output pin, but it also has the advantage that the output is set to \emph{whatever voltage level you choose}.
In other words, you don't need to pick a new comparator IC to get a different output voltage.

\reviewsection

In this chapter, we learned:

\begin{enumerate}
\item Integrated Circuits (called ICs or chips) are miniaturized circuits packaged up into an a single chip that can be added to other circuits.
\item ICs can have a few or several billion components on them, depending on the function.
\item ICs have different types of packages, including through-hole (optimized for breadboards) and surface mount (optimized for soldering and machine placement).
\item Dual In-line Packages (DIP) are the most common through-hole packaging type used for students, hobbyists, and prototype-builders.
\item DIP chips should be placed in the breadboard saddling the bridge, so that each IC pin is attached to its own terminal strip.
\item On most chips, pin 1 is located immediately counterclockwise of the notch in the chip, and remaining pins are numbered counterclockwise.
\item Most ICs are active devices, meaning that they have a direct connection to a power supply in addition to their input and output pins.a
\item An ICs Datasheet is a document that tells about the electrical characteristics of an IC.  However, most of them are difficult to read and assume you are already familiar with the part.  However, they are very useful for getting a pinout for the chip as well as telling the maximum ratings for voltages and currents.
\item The LM393 is a dual voltage comparator IC---it compares two voltages and alters its output based on which is larger.
\item The LM393's inputs do not consume any significant current on them when sensing the input voltages.
\item The LM393's outputs are open collectors---which means that they act as a switch to ground.  When the output is ``low'' the pin acts as a closed switch to ground.  When the output is ``high'' the pin acts as a disconnected circuit.
\item Because the LM393 acts as a disconnected circuit when high, a pull-up resistor circuit is required to get an output voltage.
\item Many sensors are based on the fact that the resistance of many materials will change with environmental factors.  Therefore, the sensor acts as a variable resistor, with the resistance telling you about the environment.
\item A resistive sensor can be used with a fixed resistor to make a variable voltage divider, essentially converting the resistance to a voltage.
\item By putting the resistive voltage divider in comparison with a fixed reference, we can use the LM393 comparator to trigger an output when the sensor crosses some threshold of resistance.
\end{enumerate}

\applysection


\begin{enumerate}
\item  Calculate the amount of current flowing through each element of the circuit in Figure~\ref{figSimpleComparatorCircuit}.  You can presume that the LM393 uses about $1\mymamp$ for its own operation, and that the LED is a red, $1.8\myvolt$ LED.  What is the total amount of current used by the circuit?
\item Take the circuit in Figure~\ref{figSimpleComparatorCircuit} and swap which voltage divider is attached to \icode{1IN+} and \icode{1IN-}.  Now calculate the total amount of current used by this circuit.
\item The Spectra Flex Sensor is a resistive sensor that changes its resistance when bent.  When it is straight, it has a resistance of $10\mykohm$.  When it is bent, it has resistances of $60\mykohm$ and above.  Draw a circuit that turns on an LED when the resistor is bent.  You may invent your own symbol for the flex sensor.
\item Build the circuit in Figures~\ref{figDarknessSensorCircuit} and~\ref{figDarknessSensorBreadboard}. 
\item If you wanted to wait until the room was even darker before the LED went on, how would you change the circuit?
\end{enumerate}

