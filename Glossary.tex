\chapter{Glossary}
\label{chapGlossary}

\begin{description}
\item[AC current] See \emph{alternating current}.
\item[AC mains current] This is the type of current that is supplied to your house by the public utility companies.  This is usually 120 volts AC and cycles back and forth 50--60 times per second.
\item[AC signal current] This is the type of current usually picked up by a microphone or antenna.  It has very low current and usually must be amplified before processing.
\item[alternating current]
\item[amp] A shorthand way of saying ampere.  See \emph{ampere}.
\item[ampere] An ampere is a measurement of the movement of charge.  It is equivalent to one coulomb of charge per second moving past a given point in a circuit.
\item[charge] Charge is a fundamental quantity in physics.  A particle can be positively charged (like a proton), negatively charged (like an electron), or neutrally charged (like a neutron).  Charge is measured in coulombs.
\item[conventional current flow]
\item[coulomb] A coulomb is a quantity of electric charge.  One coulomb is roughly equivalent to the charge of $6.242×10^18$ protons.  The same number of electrons produces a charge of $-1$ coulomb.  Coulombs are represented by the symbol C.
\item[DC current] See \emph{direct current}.
\item[direct current]
\item[electron current flow]
\item[electron] A negatively-charged particle that is usually on the outside of an atom.
\item[milliamp] A short way of saying milliampere.  See \emph{milliampere}.
\item[milliampere] One thousandth of an ampere.  See \emph{ampere}.
\item[neutron] An uncharged particle in the nucleus of an atom.
\item[nucleus] The nucleus is the part of the atom where protons and neutrons reside. 
\item[proton] A positively-charged particle in the nucleus of an atom.

\end{description}
