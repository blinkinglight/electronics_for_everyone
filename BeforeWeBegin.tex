\chapter{Before We Begin}

I put this chapter at the beginning because it is important and I wanted it easy to find, but you may not know enough to understand all of it.
You can skip over this section and come back to it when you start to do projects in chapter~FIXME.

\section{General Safety Note}
This book deals almost entirely with DC current from small battery sources.  
This current is inherently fairly safe, as small batteries are not capable of delivering the amount of current needed to injure or harm.  
For these projects, you can freely touch wires and work with active circuits without any protection, because the current is incapable of harming you.  

However, please note that if you ever deal with AC current or large batteries (such as a car battery), you must exercise many more precautions than described in this book, because those devices can and will harm or kill you if mishandled.

\section{Safety Guidelines}

Using small-battery DC current is very safe.  Nonetheless, you should employ these safety guidelines, both for your safety and for the safety of your circuit.  The biggest potential problem is with the battery itself, not the electricity.  Batteries are made from potentially toxic chemicals.

Please follow these guidelines, as they will both keep you safe as well as help prevent you from accidentally damaging your own equipment.

\begin{enumerate}
\item If you have any cuts or other open areas on your skin, please cover them. Your skin is where most of your electric protection exists in your body.
\item Before applying power to your circuit, check to be sure you have not accidentally wired in a short circuit between your positive and negative poles of your battery.
\item If your circuit does not behave like you expect it to when you plug in the battery, unplug it immediately and check for problems.
\item If your battery or any component becomes warm, disconnect power immediately.
\item If you smell any burning or smoky smells, disconnect power immediately.
\item Dispose of all batteries in accordance with local regulations.
\item For rechargeable batteries, follow the instructions on the battery for proper charging procedures.
\end{enumerate}

If you follow these common sense rules you should have a fun and safe experience!

\section{Electrostatic Discharge}

If you have ever touched a doorknob and received a small shock, you have experienced electrostatic discharge (ESD).  
ESD is not dangerous to you, but it can be dangerous to your equipment.  
Even shocks that you can't feel may damage your equipment.  
With modern components, ESD is rarely a problem, but nonetheless it is important to know how to avoid it.  
You can skip these precautions if you wish, just know that occasionally you might wind up shorting out a chip or transistor because you weren't careful.  
ESD is also more problematic if you have carpet floors, as those tend to build up static electricity.

Here are some simple rules you can follow to prevent ESD problems:

\begin{enumerate}
\item When storing IC components, store them with the leads enmeshed in conductive foam.  This will prevent any voltage differentials from building up in storage.
\item Wear natural 100\% cotton fabrics.
\item Use a specialized ESD floor mat and/or wrist strap to keep you and your workspace at ground potential.
\item If you don't use an ESD strap or mat, touch a large metal object before starting work.  Do so again any time after moving around.
\end{enumerate}

\section{Using Your Multimeter Correctly}
In order to keep your multimeter functioning, it is important to take some basic precautions.  Multimeters, especially cheap ones, can be easily broken through mishandling.  Use the following steps to keep you from damaging your multimeter, or damaging your circuit with your multimeter:

\begin{enumerate}
\item Do not try to measure resistance on an active circuit.  Take the resistor all the way out of the circuit before trying to measure it.
\item Choose the appropriate setting on your multimeter before you hook it up.
\item Always err on the side of choosing high values first, especially for current and voltage.  Use the high value settings for current and voltage give your multimeter the maximum protection.  If they are too large, it is easy enough to turn them lower.  If you had it set too low, you may have to buy a new multimeter!
\end{enumerate}

%% FIXME - if I add soldering to the book, I also need soldering safety tips
