
\begin{enumerate}
\item Take a look at the circuit in Figure~\ref{figSimple555Oscillator} (this will be used as the basis for the problems in this section).   Copy this circuit to a piece of paper.  Draw a line in one color showing how the current flows in the main circuit as it charges the capacitor.  With a different color, draw a line showing how current flows as the capacitor discharges.  Use arrows to indicate current direction.
\item Why is R1 important?  What would happen if we just replaced it with a wire?
\item Why are there two different pins on the NE555 connected to the capacitor?  What type of circuit (that we have discussed in this book) do you think they are connected to inside the chip?
\item Why is the charging time of the NE555 always at least a little longer than the discharging time?
\item Why does the NE555 stay in the on state a little longer when it first turns on?
\item Let's say that we wanted our circuit to be on for $2\mysec$ and off for $1\mysec$.  Keeping the same capacitor, what values should we use for R1 and R2 to accomplish that?
\item Let's say that we wanted our circuit to be on for $10\mysec$ and off for $3\mysec$.  Keeping the same capacitor, what values should we use for R1 and R2 to accomplish that?
\item The factory called and said that they were out of the capacitor we wanted for the circuit, and instead only had a $23\myuf$ capacitor that we could use.  Recalculate the previous problem using this new capacitor value.
\item How much current is our output sourcing from the chip?
\item When the chip first turns on (and thus the capacitor is empty and at $0\myvolt$) how much current is the RC circuit using?
\end{enumerate}

% FIXME - might have a problem where people draw the pinout of the 555
% FIXME - might have a problem where they draw a circuit and then circle different part
