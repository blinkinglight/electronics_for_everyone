\chapter{Introduction to Microcontrollers}

In Chapter~\ref{chapLogicICs}, we learned the basics of digital logic.
However, I think we can all agree that those chips wound up taking up a lot of space on our breadboard.
If we wanted to do a lot of complicated tasks, we would wind up needing a lot of chips, and our breadboard would get unwieldy very quickly.
Additionally, as the number of chips increased, it would get very expensive to build such projects.

Additionally, when all of the logic of a circuit is hardwired into the circuit through logic chips, it is very difficult to change.
If you need to add buttons, or remove buttons, or do anything else, you wind up needing to wade through masses of circuitry to make the change you want.
Then, if you are mass-producing the circuit, you have to setup for mass production all over again.

To solve all of these problems (and more), the microcontroller was introduced.
A microcontroller is essentially a low-power, single-chip computer.
A ``real'' computer chip usually relies on a whole slew of other chips (memory chips, input/output chips, etc.) to operate.
A microcontroller contains all of these (though usually on a smaller scale) in a single chip that can be added to an electronics project.

Unlike your typical computer, most microcontrollers can't be connected to a keyboard or other typical input, and can't be connected to a monitor, disk drive, or other typical output.
Instead, microcontrollers usually communicate entirely through digital (true/false) electrical signals on their pins.

So, instead of wiring complex logic onto their boards, many people opt to have microcontrollers provide the bulk of their digital logic.
One complication that this adds, however, is that, since the microcontroller is essentially a computer, then just like a computer, it has to be programmed.
This means that not only must circuit designers be familiar with electronics, they must also be familiar with computer programming.

\section{The ATmega328/P Chip}

The microcontroller we will be focusing on is the ATmega328/P.
Actually, we will focus less on this specific chip than the overall environment surrounding it, known as Arduino.
However, it is good to have a quick introduction to this chip and how it works.

Figure~\ref{figATmegaPinout} shows a simplified pin configuration of the chip, focused on how it is used in the Arduino environment.
The VCC pin and the GND pins are the primary power pins.
The chip can run on a range of voltages, but $5\myvolt$ is a very common and safe setting.
AVCC and AGND power the chip's analog-to-digital converter unit.

\simplegraphicsfigure{A Simplified Pinout of the ATmega328/P}{ATmegaPinout}{0.08}

All of the pins labelled ``D'' are digital input/output pins.
They can be configured as inputs for buttons or other signals, or as outputs for driving LEDs or other output devices.
The pins labelled ``A'' are analog input pins.
While the digital input pins can only read that a value is true/false, the analog pins can read voltages and convert them to numbers.
AREF is a ``reference voltage'' used for setting the maximum voltage for analog inputs, but is usually unconnected (it should also not be higher than AVCC).

Microcontrollers, like most processors, control their operation by using a ``clock.''  
This is not a clock like you normally think of.
A better way to think of this is as a heartbeat.
Basically, there is a continuous signal of pulses that are provided through the clock, and the pulses allow the chip to synchronize all of its activities.
The ATmega328/P has an internal clock, but it can also be more efficiently operated by connecting an external clock (quartz crystals, for instance, provide a \emph{very} steady pulse).

You might wonder, what does the chip \emph{do} with its input and output pins?
That is entirely up to you.  
It does \emph{whatever you program it to do}.
The ATmega328/P has flash memory on the chip which can store a computer program.
You have to upload your program to the chip, and then after that it will do whatever you like with its inputs and outputs.
The D0 and D1 pins, in addition to providing input and output, can also be used to reprogram the chip.
We will learn how to program the chip in Section~\ref{secProgrammingArduino}.

\section{The Arduino Environment}

The chip itself is just one piece of the puzzle.
In order to use the chip, you have to be able to program it.
Programming requires the use of programming tools on your main computer.
In addition, you also need some way to take the program that you built on your computer and load it onto the chip.
That takes both software and hardware.

Then, once the program is on the chip, you have to build a circuit to properly power the chip.
This requires voltage regulation for the VCC pin, and several other recommendations from the manufacturer about how to setup the other pins.  
All of this can be quite a lot of work, and a lot of pieces that need to be brought together.

Thankfully, most chips have what is called a \glossterm{development board} that can be purchased.
A development board is a pre-built circuit that has a microcontroller chip pre-connected in its recommended manner.
It is made to simplify the work of developing circuits.
Likewise, most chips have a recommended \glossterm{programming environment} as well.
A programming environment is a set of tools for your computer that allow you to create programs for your microcontroller.
Additionally, a device called an \glossterm{in-system programmer} connects your computer to your chip or development board and will transmit the program from your computer to the chip.

In 2005, a complete, simplified system for doing all of these tasks called \glossterm{Arduino} was created.
Arduino consists of (a) a simplified development environment for your computer to write software for microcontrollers, (b) a simplified development board to make it very easy to build electronics projects, and (c) integrating the in-system programmer into the development board so that all that is required is a USB cable.

The Arduino environment supports a number of different microcontroller chips.
Because it is a simplified environment, many of the special features of individual chips are not directly supported.
However, for getting started and doing basic projects, the Arduino environment is excellent.

\section{The Arduino Uno}

This book focuses on the Arduino Uno development board.
Figure~\ref{figUnoBoard}

\section{Programming the Arduino}
\label{secProgrammingArduino}
