\chapter{Introduction to Microcontrollers}

In Chapter~\ref{chapLogicICs}, we learned the basics of digital logic.
However, I think we can all agree that those chips wound up taking up a lot of space on our breadboard.
If we wanted to do a lot of complicated tasks, we would wind up needing a lot of chips, and our breadboard would get unwieldy very quickly.
Additionally, as the number of chips increased, it would get very expensive to build such projects.

Additionally, when all of the logic of a circuit is hardwired into the circuit through logic chips, it is very difficult to change.
If you need to add buttons, or remove buttons, or do anything else, you wind up needing to wade through masses of circuitry to make the change you want.
Then, if you are mass-producing the circuit, you have to setup for mass production all over again.

To solve all of these problems (and more), the microcontroller was introduced.
A microcontroller is essentially a low-power, single-chip computer.
A ``real'' computer chip usually relies on a whole slew of other chips (memory chips, input/output chips, etc.) to operate.
A microcontroller contains all of these (though usually on a smaller scale) in a single chip that can be added to an electronics project.

Unlike your typical computer, most microcontrollers can't be connected to a keyboard or other typical input, and can't be connected to a monitor, disk drive, or other typical output.
Instead, microcontrollers usually communicate entirely through electrical signals on their pins.

Instead of using complex logic on your 


