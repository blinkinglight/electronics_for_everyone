\chapter{What is Electricity?}
\label{electricitybasics}

The first thing to tackle in the road to understanding electronics is to wrap our minds around what electricity is and how it works.
The way that electricity works is very peculiar and unintuitive.
We are used to dealing with the world in terms of physical objects---desks, chairs, baseballs, etc.
Even if we never took a class in physics, we know the basic properties of such objects from everyday experience.
If I drop a rock on my foot, it will hurt.
If I drop a heavier rock, it will hurt more.
If I remove an important wall from a house, it will fall down.

However, for electricity, the only real experience we have is that we have been told to stay away from it.
Sure, we have experience with computers and phones and all sorts of devices, but they give us the result of processing electricity a million times over.
But how does electricity itself work?

\section{Charge}

To answer this question, we need to answer another question first: what \emph{is} electricity?
Electricity is the flow of \glossterm{charge}.
So what is charge?

Charge is a fundamental quantity in physics---it is not a combination (that we know of) of any other quantity.
A particle can be charged in one of three ways---it can be positively charged (represented by a \icode{+} sign), negatively charged (represented by a \icode{-} sign), or be neutrally charged (i.e., have no charge).
Figure~\ref{figChargedParticles} shows what an atom looks like.  
In the center of the atom are larger, heavier particles called \glossterm{protons} and \glossterm{neutrons}.
Protons are positively charged particles, and neutrons are neutrally charged particles.
Together these form the \glossterm{nucleus} of the atom, and determine \emph{which} atom we are talking about.
If you look on a periodic table, the big number of the element refers to how many protons it has in its nucleus, and the smaller number is usually the total number of protons and neutrons (this smaller number is sometimes a decimal because the number of neutrons can change, so it is an average).

\begin{figure}
\caption{Charged Particles in an Atom}
\label{figChargedParticles}
FIXME - need drawing
\end{figure}

Circling around the nucleus are \glossterm{electrons}.
Electrons are negatively charged particles. 
Even though electrons are much smaller and lighter than protons, the amount of negative charge of one electron is equal to the amount of positive charge of one proton.
Positive and negative charges attract each other, which is what keeps electrons contained within the atom.
Electrons are arranged in shells surrounding the nucleus.
The outermost shell, however, is the most important one when thinking about how atoms work.

When we think about individual atoms, we think about them when they are isolated and alone.
In these situations, the number of electrons and protons are equal, making the atom as a whole electrically neutral.
However, especially when atoms interact with other atoms, the configuration of their electrons can change.
If the atoms gain electrons, then they are negatively charged.
If the atoms lose electrons, then they are positively charged.
Free electrons are all negatively charged.

If there are both positively and negatively charged particles moving around, their opposite charges attract one another.
If there is a great imbalance of positive and negative charges, usually you will have a \emph{movement} of some of the charged particles towards the particles of the opposite charge.
This is a \emph{flow} of charge, and is what is referred to when we speak of electricity.

The movement of charge can be either positively-charged particles moving towards negatively-charged ones, or the reverse.
Usually, in electronics, it is the electrons which are moving through a wire, but this is not the only way which charge can move.

Electricity can be generated by a variety of means.
The way that electricity is generated in a battery is that a chemical reaction takes place, but the reactants (the substances that react together) are separated from each other by some sort of medium.  
The positive charges for the reaction move easiest through the medium, but the negative charges for the reaction move easiest through the wire.
Therefore, when the wire is connected, electricity moves through the wire to help the chemical reaction complete on the other side of the battery.

This flow of electric charge through the wire is what we normally think of as electricity.

\begin{sidebar}[Making Your Own Battery]
You can make a simple battery of your own out of three materials: thick copper wire or tubing, a galvanized nail (it \emph{must} be galvanized), and a potato or a lemon.
This battery operates from a reaction between the copper on the wire and the zinc on the outside of the galvanized nail.
The electrons will flow from the zinc to the copper through the wire, while the positive charge will flow within the potato.

To build the battery, you must insert the thick copper and the nail into the potato.  
They should be near each other, but \emph{not touching}.
This is a battery that will produce about 1.2 volts of electricity.
This is not quite enough to light up an LED, but it should register on a multimeter.  
See chapter~FIXME for how to measure voltage with a multimeter.

Different plants will yield different voltages.
You might experiment on this with lemons, strawberries, and other produce items to see what voltages each one produces.

Remember, however, that the potato is not actually supplying the current.
What the potato is doing is creating a barrier so that only the positive charges can flow freely in the potato, and the negative charges have to use the wire.
Note that this can be made even more efficient by boiling the potato first.
\end{sidebar}

\section{Measuring Charge and Current}

Atoms are very, very tiny.
Only in the last few years have scientists even developed microscopes that can see atoms directly.
Electrons are even tinier.
Additionally, it takes a \emph{lot} of electrons moving to have a worthwhile flow of charge.
Individual electrons do not do much on their own---it is only when there are a very large number of them moving that they can power our electronics projects.

Therefore, scientists and engineers usually measure charge on a much larger scale.
The \glossterm{coulomb} is the standard measure of electric charge.
One coulomb is equivalent to the electric charge of about $6,242,000,000,000,000,000$ protons.
If you have that many electrons, you would have $-1$ coulomb.
That's a lot of electrons and protons, and it takes that many to do very much electrical work.
Thankfully, protons and electrons are very, very small.
A typical 9-volt battery can provide about $2,000$ coulombs of charge, which is over $10,000,000,000,000,000,000,000$ electrons (ten thousand billion billion electrons).

However, electricity and electronics are not about electric charge sitting around doing nothing.
Electricity deals with the \emph{flow} of charge.
Therefore, when dealing with electricity, we rarely deal with coulombs.
Instead, we talk about how fast the electrical charge is flowing.
For that, we use \glossterm{amperes}, often called amps, and abbreviated as A.
1 ampere is equal to the movement of 1 coulomb of charge out of the battery each second.

For the type of electronics we will be doing, an ampere is actually a lot of current.
In fact, a full ampere of current can do a lot of physical harm to you, but we don't usually deal with full amperes when creating electronic devices.
Power-hungry devices like lamps, washers, dryers, printers, stereos, and battery-chargers need a lot of current---that's why we plug them into the wall.
Small electronic devices don't usually need so much current.
Therefore, for electronic devices, we usually measure current in \glossterm{milliamperes}, usually called just milliamps, and abbreviated as mA.
The prefix \emph{milli-} means one thousandth of (i.e., $\frac{1}{1000}$ or $0.001$).
Therefore, a milliamp is one thousandth of an amp.
If someone says that there is 20 milliamps of current, that means that there is 0.020 amps of current.
This is important, because the equations that we use for electricity are based on amps, but we are going to be mainly concerned with milliamps.

So, to go from amps to milliamps, multiply the value by $1,000$.
To go from milliamps to amps, divide the value by $1,000$ (or multiply by $0.001$) and give the answer in decimal (electronics always uses decimals instead of fractions).

\begin{exampleprob}
If I were to have $2.3$ amps of electricity, how many milliamps is that?
To go from amps to milliamps, we multiply by $1,000$.  
$2.3 * 1,000 = 2,300$.  
Therefore, $2.3$ amps is the same as $2,300$ milliamps.
\end{exampleprob}

\begin{exampleprob}
If I were to have $5.7$ milliamps of electricity, how many amps is that?
To go from milliamps to amps, we divide by $1,000$.
$5.7 / 1,000 = 0.0057$
Therefore, $5.7$ milliamps is the same as $0.0057$ amps.
\end{exampleprob}

\begin{exampleprob}
Now, let's try something harder---if I say that I am using 37 milliamps of current, how many coulombs of charge has moved after 1 minute?
Well, first, let's convert from milliamps to amps.  
To convert from milliamps to amps, we divide by $1,000$.
$37 / 1000 = 0.037$
Therefore, we have $0.037$ amps.
What is an amp?
An amp is 1 coulomb of charge moving per second.
Therefore, we can restate our answer as being $0.037$ coulombs of charge moving each second.

However, our question asked about how much has moved after 1 \emph{minute}.
Since there are $60$ seconds in each minute, we can multiply $0.037$ by $60$ for our answer.
$0.037 * 60 = 2.22$
So, after 1 minute, 37 milliamps of current moves $2.22$ coulombs of charge.
\end{exampleprob}

\section{AC vs. DC Current}

You may have heard the terms AC or DC when people talk about electricity.
What do those terms mean?
In short, DC stands for \glossterm{direct current} and AC stands for \glossterm{alternating current}.
So far, our descriptions of electricity have dealt mostly with DC current.
With DC current, electricity makes a route from the positive terminal to the negative.
It is the way most people envision electricity.
It is ``direct.''

However, DC current, while great for electronics projects, very quickly loses power over long distances.
If we were to transmit current that simple flows from the positive to the negative throughout the city, we would have to have power stations every mile or so.

So, instead of sending current in through one terminal and other through another, with alternating current, the positive and negative sides make a complete switch (both back and forth) 50--60 times per second.
So, the electrons switch back and forth, over and over again, which direction they are moving.
It is like someone is pushing and pulling current back-and-forth.
In fact, at the generator station, that is exactly what is going on!
This may seem strange, but this push and pull action allows much easier power generation and allows much more power to be delivered over much longer distances.

AC current such as the current that comes out of a wall socket is much more powerful than we require for our projects here.
In fact, converting high-power AC current to low-power DC voltage used in electronic devices is an art in itself.
This is why companies charge so much money for battery chargers---it takes a lot of work to get one right!

Now, not all AC current is like this.  
We call this current AC ``mains'' current, because it comes from the power mains from the power stations.
It is supposed to operate at about 120 volts and the circuits are usually rated for about 15--30 amps (that's 15,000--30,000 milliamps).
That's a lot of electricity!

In addition to AC mains current, there are also AC currents which we will call AC ``signal'' current.
These currents come from devices like microphones.
They are AC because they do alternate.
When you speak, your voice vibrates the air back-and-forth.
A microphone converts these air vibrations into small vibrations of electricity---pushing and pulling a small electric current back and forth.
However, these AC currents are so low-powered as to be almost undetectable.
They are so small, we have to actually amplify these currents just to work with them using our DC power!

So, in short, while we will do some work with AC voltages later in the book, all of our projects will be safe, low-power projects.
We will often touch wires with our projects active, or use multimeters to measure currents and voltages in active circuits.
This is perfectly safe for battery-operated projects.
But \emph{do not} attempt these same maneuvers for anything connected to your wall outlet unless you are properly trained.

\section{Which Way Does Current Flow?}

One issue that really bungles people up when they start working with electronics is figuring out which way that electrical current flows.
You hear first that electrical current is the movement of electrons, and then you hear that electrons move from negative to positive.
So, one would naturally assume that current flows from negative to positive, right?

Good guess, but no.
Current is not the flow of physical stuff like electrons, but the flow of \emph{charge}.
So, when the chemical reaction happens in the battery, the positive side gets positively charged.
The electrons are a negative charge that moves toward the positive charge.
The positive charge is just as real as the electron charge, even though physical stuff isn't moving.

Think about it this way.
Have you ever used a vacuum cleaner?
Let's say we are building a vacuum cleaner.
Where do you start?
Usually, you start at the inside where the suction happens and then trace the flow of suction through the tubes.
Then, at the end of the tube, the dust comes into tube.

Engineers don't trace their systems from the dust to the inside, they trace their systems from the suction on the inside out to the dust particles on the outside.
Even though it is the dust that moves, it is the suction that is interesting.

Likewise, for electricity, we usually trace current from positive to negative even though the electrons are moving the other way.
The positive charge is like the suction of a vacuum, pulling the electrons in.
Therefore, we want to trace the flow of the vacuum from positive to negative, even though the dust is moving the other way.

The idea that we trace current from positive to negative is often called \glossterm{conventional current flow}.
It is called that way because we conventionally think about circuits as going from the positive to the negative.
If you are tracing it the other way, that is called \glossterm{electron current flow}, but it is rarely used.

\reviewsection

In this chapter, we learned:

\begin{enumerate}
\item Electric current is the flow of charge.
\item Charge is measured in coulombs.
\item Electric current flow is measured in coulombs per second, called amperes or amps.
\item A milliampere is one thousandth of an ampere.
\item In an atom, protons are positively charged, electrons are negatively charged, and neutrons are neutrally charged.
\item Batteries work by having a chemical reaction which causes electricity to flow through wires.
\item In DC current, electricity flows continuously from positive to negative.
\item In AC current, electricity flows back and forth, changing flow direction  many times every second.
\item Even though electrons flow from negative to positive, in electronics we usually think about circuits and draw circuit charges as flowing from positive to negative.
\item AC mains current (the kind in your wall outlet) is dangerous, but battery current is relatively safe.
\item Small signal AC current (like that generated by a microphone) is not dangerous, either.
\end{enumerate}

\applysection


\begin{enumerate}
\item 
\question{If I have 56 milliamps of current flowing, how many amps of current do I have flowing?}
\solution{$0.056\myamp$}
\explanation{A milliamp is $0.001\myamp$, so, to convert, we multiply milliamps by $0.001$.
$$ 0.001 \frac{\myamp}{\mymamp} \times 56 \mymamp = 0.056\myamp $$
}
\item 
\question{If I have 1,450 milliamps of current flowing, how many amps of current do I have flowing?}
\solution{$1.45\myamp$}
\explanation{A milliamp is $0.001\myamp$, so, to convert, we multiply milliamps by $0.001$.
$$ 0.001 \frac{\myamp}{\mymamp} \times 1450\mymamp = 1.45\myamp $$
}
\item 
\question{If I have 12 amps of current flowing, how many milliamps of current do I have flowing?}
\solution{$12,000\mymamp$}
\explanation{There are $1,000$ milliamps in each amp.  So, to convert, we multiplay amps by $1,000$.
$$ 1,000 \frac{\mymamp}{\myamp} \times 12\myamp = 12,000\mymamp $$
}
\item 
\question{If I have 0.013 amps of current flowing, how many milliamps of current do I have flowing?}
\explanation{There are $1,000$ milliamps in each amp.  So, to convert, we multiplay amps by $1,000$.
$$ 1,000 \frac{\mymamp}{\myamp} \times 0.013\myamp = 12\mymamp $$
}
\item 
\question{If I have 125 milliamps of current flowing for one hour, how many coulombs of charge have I used up?}
\solution{$450\text{ coulombs}$}
\explanation{An amp is a flow of 1 coulomb per second.  First we have to convert milliamps to amps:
$$ 0.001\frac{\myamp}{\mymamp} \times 125 \mymamp = 0.125 \myamp $$
Because an amp is 1 coulomb per second, this is equal to $0.125\frac{\text{coulomb}}{\mysec}$.
Since it is flowing for an hour, we have to convert hours into seconds.
$$ 1 \textrm{ hour} \times \frac{60\text{ min}}{\text{hour}} \times \frac{60\text{ seconds}}{\text{minute}} = 3,600 \text{ seconds} $$
So, we have $0.125$ coulombs per second for $3,600$ seconds.  So the total number of coulombs is:
$$3,600 \frac{\text{coulombs}}{\text{second}} \times 0.125 \text{ seconds} = 450\text{ coulombs} $$
}
\item 
\question{What is the difference between AC and DC current?}
\solution{AC stands for ``alternative current.'' In AC current moves back and forth, continually changing the direction that it is moving.  DC stands for ``direct current.''  In DC current moves in essentially one direction.}
\item 
\question{In AC mains current, how often does the direction of current go back and forth?}
\solution{The direction of current goes back and forth $50--60$ times per second.}
\item 
\question{Why is AC used instead of DC to deliver electricity within a city?}
\solution{There is much less loss over large distances using AC}
\item 
\question{In working with electronic devices, do we normally work in amps or milliamps?}
\solution{With small electronic devices, we usually measure our currents in milliamps.}
\end{enumerate}

