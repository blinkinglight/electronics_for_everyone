\chapter{Electronics Equations and Where They Come From}
\label{appendixElectronicsEquations}

% Thevenin Equivalent Equation
% Gain resistor equation

This appendix is a catalog of equations in electronics and where they came from for those who are curious.
This book is meant more for an introductory approach, but nonetheless many people are curious.
This chapter isn't for the faint of heart, and it may involve lots of math you haven't taken.
That's why it is stuck in an appendix.

However, if you are curious, these are the mathematical answers to your questions.

\section{Basic Formulas}

\subsection{Charge and Current Quantities}

\begin{itemize}
\item $1\textrm{ coulomb}  = 6.241509×10^18\textrm{ electrons}$ (how many electrons in a coulomb)
\item $I = \frac{dC}{dt}$ (current is the derivative of charge with respect to time)
\item $1A = 1\frac{C}{s}$ ($A$ = ampere; $C$ = coulomb; $s$ = second)
\item $3.6C = 1 mAh$ ($C$ = coulomb; $mAh$ is milliamp-hour,  a common unit for batteries)
\end{itemize}

\subsection{Volt Quantities}

Volts are basically measures of energy per unit of charge. Volts are also known as electromotive force (EMF), or $\epsilon$.  Volts can be expressed in a number of ways:

\begin{itemize}
\item $V = \frac{J}{C}$ ($J$ = joules; $C$ = coulombs)
\item $V = \frac{\textrm{potential energy}}{\textrm{charge}}$
\item $V = \frac{N\cdot m}{C}$ ($N$ = newtons; $m$ = meters; $C$ = coulombs)
\item $V = \frac{kg\cdot m^2}{A\cdot s^3}$ ($kg$ = kilograms; $m$ = meters; $A$ = amperes; $s$ = seconds)
\item $V = \frac{d\phi}{dt}$ (Faraday's law of induction---voltage is the derivative of the flux of the magnetic field with respect to time)
\end{itemize}

\subsection{Resistance and Conductance Quantities}

Resistance is in ohm's.  The inverse of resistance is conductance (the ability of current to flow through a wire) and is measured in Siemens (S).  The Seimens unit is also called a mho (ohm spelled backwards), and is sometimes marked by an upside down omega (℧).

\begin{itemize}
\item $G = \frac{1}{R}$ ($G$ = conductance in siemens, $R$ = resistance)
\item $G = \frac{I}{V}$ ($G$ = conductance; $I$ is current; $V$ is voltage)
\item $R = \frac{V}{I}$ (Ohm's law)
\end{itemize}

Individual materials have a resistivity ($\rho$).

\begin{equation}
R = \rho \cdot \frac{\textrm{length}}{\textrm{cross-sectional area}}
\end{equation}

In other words, from beginning to end, resistance decreases with cross-sectional area, and increases with length.

\subsection{Ohm's Law}

$V$ is voltage (in volts), $I$ is current (in amperes), and $R$ is resistance (in ohms).

\begin{equation}
V = I\cdot R
\end{equation}

\subsection{Power}

$P$ is in Watts.  The following hold true for DC circuits.  For AC circuits, they hold true if the resistance is actually an impedance.

\begin{itemize}
\item $P = V\cdot A$
\item $P = I^2R$
\item $P = \frac{V^2}{R}$
\end{itemize}

\subsection{Capacitance}

Capacitance is the ability to store charge.

The fundamental equation for a capacitor:

\begin{equation}
Q = V\cdot C
\end{equation}

$Q$ is the amount of charge stored, $V$ is the voltage across the terminals, and $C$ is the capacitance in farads.

The derivative of this equation with respect to time is:

\begin{equation}
\frac{dQ}{dt} = \frac{dV}{dt}\cdot C
\end{equation}

Because current is the derivative of charge, we can then say:

\begin{equation}
I = C\frac{dV}{dT}
\end{equation}

The capacitance of capacitors is given by the equation:

\begin{equation}
C = \epsilon_r\epsilon_0\frac{A}{d}
\end{equation}

Here $C$ is capacitance, $\epsilon_r$ is the dielectric constant of whatever separates the capacitor's plates, $\epsilon_0$ is the dielectric constant of free space, $A$ is the area of the plates in square meters, and $d$ is the distance between the plates in meters.

\subsection{Inductance}

The fundamental equation for an inductor is:

\begin{equation}
\phi = L\cdot I
\end{equation}

Here, $\phi$ is the flux of the magnetic field in Webers, $L$ is inductance in henries, and $I$ is current in amperes.
The derivative gives you voltage:

\begin{align}
\frac{d\phi}{dt} = L\frac{dI}{dt} \\
V = L\frac{dI}{dt}
\end{align}

In other words, the voltage produced is proportional to the change in current.

The inductance of a coil of wire can be calculated by:

\begin{equation}
L = \frac{\mu \cdot N^2 \cdot A}{l}
\end{equation}

Where $N$ is the number of turns of wire, $A$ is the area of the coil, $l$ is the length of the coil, and $\mu$ depends on the core being used.

\section{Semiconductors}

Components made from silicon are known as semiconductors, and have very useful non-linear properties.

\subsection{Diodes}

Diodes do not have a fixed voltage drop like we assume in this book.  
It is an exponential function, but is steep enough to act like a fixed $0.6\myvolt$ voltage drop for most purposes.
The actual equation is:

\begin{equation}
I = I_S (e^{\frac{V}{\eta V_T}} - 1)
\end{equation}

$I_S$ is the saturation current (depends on the construction of the diode), $V$ is the voltage, $\eta$ is either 1 for germanium or 2 for silicon, and $V_T$ is known as the thermal voltage (the amount of voltage created just by particles moving around at a given temperature, usually about $0.026\myvolt$ at room temperature).

\subsection{NPN BJT Transistors}

While we discussed general rules about BJT transistors, the technical model used to model them is known as the Ebers-Moll model.
This model is much more complex to use, which is why we don't discuss it much in the chapter.

There are also several different Ebers-Moll models, depending on the level of detail required.
The basic Ebers-Moll model for a conducting but unsaturated transistor is as follows:

\begin{equation}
I_E = I_S(e^{\frac{V_BE}{V_T}} - 1)
\end{equation}

In this $I_E$ is the emitter current (you can also use it for the collector current, since they are approximately equal).
$I_S$ is the saturation current of the base-emitter diode, and $V_T$ is the thermal voltage, just like for diodes.

% http://people.seas.harvard.edu/~jones/es154/lectures/lecture_3/bjt_models/ebers_moll/ebers_moll.html
% http://inderjitsingh87.weebly.com/uploads/2/1/1/4/21144104/the__ebers-moll_bjt_model.pdf
% http://ecetutorials.com/analog-electronics/ebers-moll-model-of-transistor/

\section{DC Motor Calculations}
\label{appDCMotorCalculations}

The voltage drop across a motor ($V_m$) is defined by the following equation:
\begin{align*}
V_m = V_b + R_m I_m
\end{align*}
where $V_b$ is the back EMF of the motor, $R_m$ is the internal resistance of the motor's wiring, and $I_m$ is the current flowing through the motor.
So, basically, just Ohm's law plus the back EMF generated by the spinning of the motor.

So how much back EMF is created?  We can determine that like this:
\begin{align*}
V_b = K_e \omega
\end{align*}
In this equation, $K_e$ varies by the motor, and is usually given in volts per RPM or volts per radians per second.
$\omega$ is merely the rotational speed in the units given.

Likewise, the torque generated can be determined from this equation:
\begin{align*}
T = K_T I_m
\end{align*}
In this equation, $K_T$ is the torque constant for the motor (be careful of the units), and $I_m$ is the current going through the motor.
Knowing the peak (stall) current, you can find the maximum torque available for the motor.

Interestingly, you can see that increasing the torque will actually affect, to some degree, the RPM of the motor.
The full equation for the voltage across the motor is:
\begin{align*}
V_m = K_e \omega + R_m I_m
\end{align*}

The torque will affect $I_m$.
That, in effect, will increase the voltage drop given by $R_m I_m$.
Given a fixed voltage source, this will leave less voltage available for the $K_e \omega$ part of the equation.
Since $K_e$ is a constant, that means that $\omega$ will be reduced to some extent.


\section{555 Timer Oscillator Frequency Equation}
\label{apOscillatorFreq}

In Chapter~\ref{chapOscillators} we learned to make oscillators using the 555 timer chip.
In the actual chapter, I wanted you to focus on actually learning what was happening with the 555 timer rather than using a formula.
However, there is a nice, simple formula that allows you to relate the resistor/capacitor network of the 555 timer to the final output frequency.

The formula is as follows:

\begin{equation}
f = \frac{1.44}{C(R_1 + 2 R_2)}
\end{equation}

In this equation, $f$ is the frequency, $R_1$ is the resistor coming from the supply voltage, $R_2$ is the resistor next to the capacitor, and $C$ is the timer capacitor.

To understand where this equation comes from, remember that frequency is just $\frac{1}{\textrm{period}}$.
We can use time constant formulas to find the period, and then just flip it to find the frequency.

If you recall, the period is just the total time it takes to complete a charge/discharge cycle.  
The 555 charges through \emph{both} $R_1$ and $R_2$, but only discharges through $R_2$.
Additionally, since it is just bouncing back-and-forth between $\frac{1}{3}$ and $\frac{2}{3}$ full, it only uses $0.693$ time constants.

Therefore, we can have two formulas, one for the time charging and one for the time discharging:

\begin{align*}
T_{\text{CHARGE}} &= 0.693 C (R_1 + R_2) \\
T_{\text{DISCHARGE}} &= 0.693 C R_2
\end{align*}

The total period is just these two time periods added together.
Therefore, you get:

\begin{align*}
T_{\text{PERIOD}} &= 0.693 C (R_1 + R_2) + 0.693 C R_2 \\
 &= 0.693 C((R_1 + R_2) + R_2) && \textrm{Factoring out $0.693 C$} \\
 &= 0.693 C(R_1 + 2 R_2) && \textrm{Regrouping}
\end{align*}

Since $f = \frac{1}{T_{\text{PERIOD}}}$, we can flip the above equation and get:

\begin{align*}
f &= \frac{1}{0.693 C(R_1 + 2 R_2)} \\
  &= \frac{1}{0.693}\frac{1}{C(R_1 + 2 R_2)} && \textrm{Regrouping} \\
  &\approx 1.44 \frac{1}{C(R_1 + 2 R_2)} \\
  &= \frac{1.44}{C(R_1 + 2 R_2)}
\end{align*}

At the end of the day, this is exactly what you did when you solved those problems, you just did it by hand instead of using a nice little formula.
All a formula does is encapsulate the things that you normally do anyway, but simplifies it down to a set of pre-defined steps.

I have a love/hate relationship with formulas.  
Formulas are nice because they are easy to use.
However, when you use them, it makes it easy to forget the basic facts behind them.
The basic facts are more important than the formula, because you can rearrange the basic facts and develop all sorts of formulas depending on your needs.
In fact, if you know the basic facts, and you know how to make formulas, if you ever forget a formula it is easy to determine one from the basic facts.
Therefore, while memorizing formulas is important, knowing \emph{why} formulas work is just as important, as it allows you to think more deeply and broadly and adapt your knowledge to new situations.

% Need to think more on this
\iffalse
\subsection{Current Gain Stabilization in BJT Common Emitter Applications}
\label{eqGainStabilizingEmitterResistorDCCommonEmitter}

% FIXME - should probably show two circuits

This section will show how we got the equation for the gain stabilizing emitter resistor you encountered in Chapter~\ref{chapTransistorIntro}.
We will need to imagine two circuits.
One before we add the emitter resistor (we will call this the \emph{nominal} circuit) and one after (we will call this the \emph{actual} circuit).
Since the goal is to do our calculations for base current on the circuit \emph{without} the emitter resistor, we will call this current the \emph{nominal base current}, and give it the symbol $I_N$.
All other values and currents will be determined from the \emph{actual} circuit.
The final gain between the nominal base current we calculated and the final collector current we will designate as $K$.
The values that are shared between the nominal and the actual circuit are $V_S$ (source voltage) and $V_B$ (base resistance).

\begin{align*}
K &= \frac{I_C}{I_N} && \textrm{this is our final gain} \\
I_N &= \frac{V_S - 0.6}{R_B} && \textrm{Ohm's Law for nominal base} \\
I_B &= \frac{V_S - V_B}{R_B} && \textrm{Ohm's Law for actual base} \\
V_B &= V_E + 0.6 && \textrm{transistor rules} \\
I_B &= \frac{V_S - (V_E + 0.6)}{R_B} && \textrm{substituting} \\
I_B &= \frac{V_S - V_E - 0.6)}{R_B} && \textrm{simplifying} \\
V_E &= R_E I_E && \textrm{Ohm's Law} \\
V_E &= R_E(I_B + I_C) && \textrm{substituting} \\
V_E &=  R_E I_B + R_E I_C && \textrm{distributing} \\
I_B &= \frac{V_S - (R_E I_B + R_E I_C) - 0.6}{R_B} && \textrm{substituting} \\
I_B &= \frac{V_S - R_E I_B - R_E I_C - 0.6}{R_B} && \textrm{simplifying} \\
I_C &= \beta I_B && \textrm{transistor beta equation} \\
I_C &= \beta \frac{V_S - R_E I_B - R_E I_C - 0.6}{R_B} && \textrm{substituting} \\
I_C &= \frac{\beta V_S - \beta R_E I_B - \beta R_E I_C - \beta 0.6}{R_B} \\
I_N &= \frac{V_S - 0.6}{R_B} && \textrm{Copied from Earlier} \\
K &= \frac{I_C}{I_N} && \textrm{This is the value we are looking for} \\
\frac{I_C}{I_N} &= \frac{\frac{\beta V_S - \beta R_E I_B - \beta R_E I_C - \beta 0.6}{R_B}}{\frac{V_S - 0.6}{R_B}} && \textrm{substituting} \\
\frac{I_C}{I_N} &= \frac{\beta V_S - \beta R_E I_B - \beta R_E I_C - \beta 0.6}{R_B} \frac{R_B}{V_S - 0.6} && \textrm{simplifying} \\
\frac{I_C}{I_N} &= \frac{\beta V_S - \beta R_E I_B - \beta R_E I_C - \beta 0.6}{V_S - 0.6} && \textrm{simplifying} \\
I_B &= \frac{I_C}{\beta} && \textrm{transistor beta definition} \\
\frac{I_C}{I_N} &= \frac{\beta V_S - \beta R_E \frac{I_C}{\beta} - \beta R_E I_C - \beta 0.6}{V_S - 0.6} && \textrm{substituting} \\
\frac{I_C}{I_N} &= \frac{\beta V_S - R_E I_C - \beta R_E I_C - \beta 0.6}{V_S - 0.6} && \textrm{simplifying} \\
\end{align*}
\begin{align*}
I_C(V_S - 0.6) = I_N(\beta V_S - R_E I_C - \beta R_E I_C - \beta 0.6) && \textrm{cross-multiplying} \\
I_C V_S - 0.6 I_C = I_N \beta V_S - I_N R_E I_C - I_N \beta R_E IC - I_N \beta 0.6 && \textrm{distributing} \\
I_C V_S - 0.6 I_C + I_N R_E I_C + I_N \beta R_E I_C =  I_N \beta V_S  -  I_N \beta 0.6 && \textrm{collecting $I_C$ terms} \\
I_C V_S - 0.6 I_C + \frac{V_S - 0.6}{R_B} R_E I_C + \frac{V_S - 0.6}{R_B} \beta R_E I_C =  I_N \beta V_S  -  I_N \beta 0.6 && \textrm{substituting some $I_N$ terms} \\
I_C V_S - 0.6 I_C R_B + V_S R_E IC - 0.6 R_E I_C + V_S \beta R_E I_C - 0.6 \beta R_E I_C =  R_B I_N \beta V_S  -  R_B I_N \beta 0.6 && \textrm{getting rid of fraction} \\
I_C(V_S - 0.6 R_B + V_S R_E - 0.6 R_E + V_S \beta R_E - 0.6 \beta R_E) =  I_N(R_B \beta V_S  -  R_B \beta 0.6) && \textrm{factoring} \\
\frac{I_C}{I_N} = \frac{R_B \beta V_S  -  R_B \beta 0.6}{V_S - 0.6 R_B + V_S R_E - 0.6 R_E + V_S \beta R_E - 0.6 \beta R_E}
\end{align*}

Now, this looks like a huge mess, and it is.
However, lets look at what happens with reasonable values.
Let's say our base resistor was $1\mykohm$ and the emitter resistor was $300\mykohm$.  
Let's also say that the source voltage if $5\myvolt$ and the transistor $\beta$ is 100.
What does this look like?

\begin{align*}
\frac{IC}{IN} &= \frac{R_B \beta V_S  -  R_B \beta 0.6}{V_S - 0.6 R_B + V_S R_E - 0.6 R_E + V_S \beta R_E - 0.6 \beta R_E} \\
\frac{I_C}{I_N} &= \frac{1000 \cdot 100 \cdot 5  -  1000\cdot 100 \cdot 0.6}{5 - 0.6 \cdot 1000 + 5 \cdot 300 - 0.6 \cdot 300 + 5\cdot 100 \cdot 300 - 0.6 \cdot 100 \cdot 300}
\end{align*}

This then becomes

\begin{align*}
\frac{I_C}{I_N} = \frac{500000 - 60000}{5 - 600 + 150000 - 180 + 1500 - 18000}
\end{align*}

Now, on the numerator, the dominating term is $500,000$, and on the bottom it is $150,000$.  
The other terms pale in comparison.
This will be true for most ``typical'' situations.
So, what makes up these two terms?

On the top, the $500,000$ term comes from $R_B \beta V_S$.
On the bottom, the $150,000$ term comes from $V_S \beta R_E$.
Therefore, a simplified look at this equation is just:

\begin{align*}
\frac{I_C}{I_N} = \frac{R_B \beta V_S}{V_S \beta R_E} && \textrm{these are the dominant factors} \\
\frac{I_C}{I_N} = \frac{R_B}{R_E} && \textrm{cancelling out factors} \\
\end{align*}

So, even though it isn't an exact result, you can see that the ratio between the actual current in the real circuit and the nominal base current that we calculated will be given by $\frac{R_B}{R_E}$.
\fi

\section{Voltage Gain Stabilization in BJT Common Emitter Applications}
\label{apTransistorVoltageGain}

The previous section told you how to stabilize the current gain from an emitter resistor, with the final gain being $\frac{R_B}{R_E}$.
For \emph{voltage} gain, the final gain is limited by $\frac{R_C}{R_E}$, and we will show that here using similar reasoning.

The voltage at the emitter and the base will be:

\begin{align*}
V_E &= I_E R_E \\
V_B &= V_E + 0.6 \\
V_B &= I_E R_E + 0.6 \\
\end{align*}

The voltage at the collector will be:

\begin{align*}
V_C &= I_C R_C
\end{align*}

However, the collector current and the emitter current are very close.
Therefore, we can simplify this to say:

\begin{align*}
V_C &= I_E R_C
\end{align*}

We can then divide our equation for the collector voltage by our equation for the base voltage, and get:

\begin{align*}
\frac{V_C}{V_B} &= \frac{I_E R_C}{I_E R_E + 0.6}
\end{align*}

If we remove the diode voltage (which has relatively little influence overall), this simply becomes:

\begin{align*}
\frac{V_C}{V_B} &= \frac{I_E R_C}{I_E R_E} \\
 &= \frac{R_C}{R_E}
\end{align*}

Therefore, the ratio of base voltage to output voltage is approximately equal to the ratio of collector to emitter.
However, also keep in mind that this is the voltage \emph{drop} in the collector.  
What we actually send to the output is actually our supply voltage with $V_C$ subtracted from it.

As you can see, there is a lot of simplification involved.
However, such simplifications allow us to think more clearly about our circuits.
Since there are so many moving parts, looking for and finding the dominating factors is very important.

Note that this is important in life, too.
Sometimes we get so enmeshed in the details that we forget what factors dominate our quality of life.
Finding out what the important factors are help us to focus on the things that matter most, and let the details sort themselves out.

\section{The \thev Formula}

In Chapter~\ref{chapPartialCircuits}, we used two formulas which allowed us to calculate the \thev Equivalent circuit for circuits experimentally.
Equations~\ref{eqThevEqVoltExp} and~\ref{eqThevEqResExp} seem strange and complicated, but they are actually directly deducible from Ohm's law and the concept of an equivalent circuit.

The \thev Theorem states that any combinations of voltage sources and resistances can be replaced by a single voltage source and a single resistance.  
We will call this our \thev voltage source ($V_T$) and our \thev impedance ($R_T$).
If we hook up a load (i.e., a fixed resistance) across the output terminals of this circuit, we will know the resistance that was added (because \emph{we} added it), and we can measure the voltage drop across the resistor easily enough with a multimeter or oscilloscope.

We will need to measure this using two different loads because we have two unknowns---$V_T$ and $R_T$.
Using two different loads will give us two different equations using Ohm's law that will allow us to solve for two variables.
We will call our lower-resistance load $R_L$ and the voltage drop across the $R_L$ resistor will be $V_L$.
Likewise, our higher-resistance load we will call $R_H$ and the voltage drop across it will be $V_H$.
The current running through each of these loads ($I_L$ and $I_H$) can be given by:

\begin{align*}
V_L &= I_L \cdot R_L \\
V_H &= I_H \cdot R_H
\end{align*}

That is just simply Ohm's law.
We can also use Ohm's law to develop equations for the whole circuit, including the \thev Equivalent voltage and impedance.
Remember, because of the current rules, whatever current is flowing through our resistor must also be flowing in our \thev Equivalent impedance.
Therefore, the \thev Equivalent voltage will be the current multiplied by the two impedances together.
Therefore, this yields the following equations:

\begin{align*}
V_T &= I_L (R_L + R_T) \\
V_T &= I_H (R_H + R_T)
\end{align*}

Both of these equations solve for $V_T$, given an unknown of $R_T$.  
We can also rearrange either of these to solve for $R_T$.  
Let's rearrange the first one to do that:

\begin{align*}
V_T &= I_L (R_L + R_T) && \textrm{Original equation} \\
\frac{V_T}{I_L} &= R_L + R_T  && \textrm{Divide both sides} \\
\frac{V_T}{I_L} - R_L &= R_T && \textrm{Subtract $R_L$} \\
R_T &= \frac{V_T}{I_L} - R_L && \textrm{Solved for $R_T$} \\
\end{align*}

This is the same as Equation~\ref{eqThevEqResExp}.
However, it requires $V_T$ to work.
Now that we have an equation for $R_T$, we can substitute that back in and get an equation for $V_T$ without using $R_T$.
Using basic algebra manipulations, we can do the following:

\begin{align*}
V_T &= I_H (R_T + R_H) && \textrm{Original equation} \\
V_T &= I_H R_T + I_H R_H && \textrm{Distributive rule} \\
V_T &= I_H (\frac{V_T}{I_L} - R_L) + I_H R_H && \textrm{Substituting for $R_T$} \\
V_T &= I_H \frac{V_T}{I_L} - I_H R_L + I_H R_H && \textrm{Distributing} \\
V_T - I_H \frac{V_T}{I_L} &= - I_H R_L + I_H R_H && \textrm{Get the $V_T$s together} \\
V_T(1 - \frac{I_H}{I_L}) &= I_H (R_H - R_L) && \textrm{Factor both sides} \\
V_T &= \frac{I_H (R_H - R_L)}{1 - \frac{I_H}{I_L}} && \textrm{Divide both sides} \\
V_T &= \frac{\frac{V_H}{R_H} (R_H - R_L)}{1 - \frac{V_H R_L}{R_H V_L}} && \textrm{Replace currents with Ohm's law equivalents ($\frac{V}{R}$)}
\end{align*}

As you can see, this is Equation~\ref{eqThevEqVoltExp}.

% Basic algebra is easier but gives a different (slightly more complex) form
%I_H V_T &= I_H I_L R_L + I_H I_L R_T && \textrm{Multiply both sides by $I_H$} \\
%I_L V_T &= I_H I_L R_H + I_H I_L R_T && \textrm{Multiply both sides by $I_L$} \\
%I_H V_T - I_L V_T &= I_H I_L R_L + I_H I_L R_T - (I_H I_L R_H + I_H I_H R_T) && \textrm{Subtract both equations} \\
%I_H V_T - I_L V_T &= I_H I_L R_L + I_H I_L R_T - I_H I_L R_H - I_H I_H R_T && \textrm{Distribute the subtraction} \\
%I_H V_T - I_L V_T &= I_H I_L R_L - I_H I_L R_H  && \textrm{Simplify} \\
%V_T (I_H - I_L) &= I_H I_L R_L - I_H I_L R_H && \textrm{Regroup the left-hand side} \\
%V_T &= \frac{I_H I_L R_L - I_H I_L R_H}{I_H - I_L} && \textrm{Divide both sides by $I_H - I_L$} \\
%V_T &= \frac{I_H I_L (R_L - R_H)}{I_H - I_L} && \textrm{Regroup the top} \\
%V_T &= \frac{\frac{V_H}{R_H} \frac{V_L}{R_L} (R_L - R_H)}{\frac{V_H}{R_H} - \frac{V_L}{R_L} && \textrm{Replace currents with their Ohm's law equivalents}

\section{Electronics and Calculus}

Calculus is a favorite subject of mine, and many parts of electronics make a lot of sense in the light of Calculus.

\subsection{Current and Voltage}
First of all, recognize that electronics includes both static and dynamic quantities. 
Charge, for instance, is a static quantity.
Current, though, is the \emph{movement} of charge, and thus is a dynamic quantity.
Current entering or leaving a point can be written as a differential:
\begin{equation}
I = \frac{\dQ}{\dt}
\end{equation}
Like current, voltage is a dynamic quantity, which is the change in the magnetic flux ($\phi$):
\begin{equation}
V = \frac{\deriv{\phi}}{\dt}
\end{equation}

\subsection{Capacitors and Inductors}

The static equation governing a capacitor is
\begin{equation}
Q = V \cdot C
\end{equation}
where $Q$ is the charge, $V$ is the voltage, and $C$ is the capacitance.
Taking the derivative of both sides, this can be converted into a dynamic equation. 
\begin{equation}
\frac{\dQ}{\dt} = C\frac{\dV}{\dt}
\end{equation}
Since $\frac{\dQ}{\dt} = I$ we can rewrite this as
\begin{equation}
I = C\frac{\dV}{\dt}.
\end{equation}
What this means is that the current is proportional to the \emph{change} in voltage.

The static equation governing an inductor is similar:
\begin{equation}
\phi = L \cdot I
\end{equation}
Taking the derivative of both sides yields
\begin{equation}
\frac{\deriv{\phi}}{\dt} = L \cdot \frac{\deriv{I}}{\dt}
\end{equation}
Since $V = \frac{\deriv{\phi}}{\dt}$, this can be rewritten as
\begin{equation}
V = L \cdot \frac{\deriv{I}}{\dt}.
\end{equation}
In other words, on inductors, the voltage is proportional to the change in current.

\subsection{Time Constants}

If an ideal capacitor were connected to an ideal voltage source, then the capacitor would charge instantaneously.

