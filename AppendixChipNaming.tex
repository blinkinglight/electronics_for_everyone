\chapter{Integrated Circuit Naming Conventions}
\label{appICNaming}

% Source information
% 4000 vs 74xx - http://www.elecdude.com/2014/07/differences-in-cmos-4000-series-74ls-74hc-74hct.html
% microwatts
% Simple logic gates - http://www.eng.utah.edu/~cs6710/handouts/AppendixB/appendixB.doc2.html
% http://www.avrfreaks.net/forum/does-hct-vs-hc-still-matter-5v
% http://www.righto.com/2014/09/reverse-engineering-counterfeit-7805.html
% Simple digital computer - http://www.simpledigitalcomputer.com/
% Open collector info - http://denethor.wlu.ca/pc200/lectures/lgcocbeam.pdf
% Logic Level - https://en.wikipedia.org/wiki/Logic_level / http://www.allaboutcircuits.com/textbook/digital/chpt-3/logic-signal-voltage-levels/
% TTL - https://en.wikipedia.org/wiki/Transistor%E2%80%93transistor_logic
% TTL/CMOS comparison - http://design-technology-in-stem.x10.mx/07%20Digital%20Electronics/02%20TTL%20and%20CMOS%20ICs/digital%20IC%20introduction/digtal%20IC's%20background%20introduction.html
% TTL/CMOS level shifting - http://www.instructables.com/id/Level-Shifting-Between-TTL-and-CMOS/?ALLSTEPS
% 7400 series history - https://en.wikipedia.org/wiki/7400_series
% 7400 series partlist -  https://en.wikipedia.org/wiki/List_of_7400_series_integrated_circuits
% 4000 series history - https://en.wikipedia.org/wiki/4000_series
% 4000 series partlist - https://en.wikipedia.org/wiki/List_of_4000_series_integrated_circuits
% Manufacturer prefixes - http://www.interfacebus.com/logic_prefix.html / https://en.wikibooks.org/wiki/Practical_Electronics/Manufacturers_Prefix / http://www.logwell.com/tech/components/ic_id.html / https://dmohankumar.wordpress.com/2012/04/24/know-the-meaning-of-transistor-and-ic-codes/
% General logic numbering - http://www.radio-electronics.com/info/data/semicond/logic-ic-families-technologies/ic-numbering.php 
% 74 families - http://www.chip1stop.com/web/AUS/en/tutorialContents.do?page=051 / http://www.chip1stop.com/web/AUS/en/tutorialContents.do?page=050
% Suffix table - http://www.chip1stop.com/web/AUS/en/tutorialContents.do?page=052
% CMOS/TTL discussion - https://www.physicsforums.com/threads/high-low-impedances-by-ttl-logic.57886/
% CMOS/TTL long - http://www.lns.cornell.edu/~ib38/teaching/p360/lectures/wk09/l26/EE2301Exp3F10.pdf
% Making chip numbers more visible - http://electronics.stackexchange.com/questions/5186/how-to-read-the-text-printed-on-top-of-every-ic
% General identification guide - http://how-to.wikia.com/wiki/How_to_identify_computer_chips_or_integrated_circuits_on_circuit_boards
%
% Note that on ebay it is best just to look at the part and make sure the picture is DIP!
% 

The naming conventions for ICs can be bewildering at first.
In truth, there is no official standard for chip names, but there are some conventions that are often followed.
When a chip is invented, the company that invented it assigns it a part number.
However, the courts have ruled that part numbers cannot be copyrighted.
Therefore, if another manufacturer makes a similar or identical chip with the same pinout, they will often use the same part number.

\section{Logic Chip Basic Conventions}

Logic chips are often broken up into two families based on the voltage levels that they expect and produce.
\glossterm{TTL} (which stands for transistor-to-transistor logic) and an old standard for logic chips which usually operate at $5\myvolt$.
TTL chips consider a signal to be ``false'' when it is below $0.8\myvolt$ and ``true'' when it is above $2.2\myvolt$, and can break if they receive voltages significantly higher than $5\myvolt$.
Between $0.8\myvolt$ and $2.2\myvolt$ is a region where the resulting value is unpredictable.
TTL originally referred to \emph{how} the logic chips were constructed, but now it usually refers to the expected input/output levels of the chip.

\glossterm{CMOS} is a newer technology, and with it came a newer standard for how logic levels are interpreted.  
CMOS chips support a much wider supply voltage range than TTL, but their logic levels are a little different.
For CMOS, false is from $0\myvolt$ to one-third of the supply voltage (whatever it is---CMOS can usually operate from $3\myvolt$ to $15\myvolt$), and true is from two-thirds of the supply voltage to the full supply voltage.

Chips are often made in a series---a whole set of chips which all perform complementary functions.
The most common series of logic chips is the 7400 series originally designed by Texas Instruments.
The 7400 series started as a set of TTL chips.
Some common chips in this series is the 7400 itself (a quad NAND gate), the 7408 (a quad AND gate), and the 7432 (a quad NOR gate).

Chips names will often have a prefix that relates to either their manufacturer or the company that originally designed them.
As some examples, National Semiconductor chips are usually prefixed with LM, Texas Instruments chips have a whole slew of prefixes, including SN and TI, Motorola chips usually have an MC prefix, and Signetics usually has an NE prefix.
There are many others, but this is just to give you an example.
The 7400 series usually has part numbers starting with SN74 because TI built them.
So, a SN7408 is an AND gate based on designs by TI.

Now, the series is 7400.  
The last two digits refers to the function and pinout of the chips.
That is, in the 7400 series, ``08'' will refer to quad AND gates which all have the same pin configuration.
However, sometimes they will insert letters in-between ``74'' and ``08.''
This usually refers to some modification to the electrical characteristics of the chip.
For instance, a low-power version of the 7400 series have a ``74LS'' prefix.
So, the ``SN74LS08'' chip is a version of the 7408 (i.e., has the same pinout and function) that was originally designed by TI (the SN prefix) but is built for lower power consumption (LS).

Then, part numbers often have suffixes as well.
Suffixes often refer to some external characteristic of the chip.
For instance, in Chapter~\ref{chapIC}, we mentioned different chip packages, such as DIP and SMD.
These different packages will have different suffixes.
For the 7400 series, the DIP package is usually suffixed with ``P,'' so an SN74LS08P is the DIP version of the SN74LS08, and the NS74LS08NSR is an SMD version.
You may also have suffixes which are based on temperature, hardiness, and even occasionally electrical output characteristics.

Sometimes, if a different manufacturer builds the same chip, they may change the manufacturer code and use different suffixes.  
For instance, Texas Instruments sells a SN74HC08P, which is a DIP 7408 which uses CMOS levels up to $6\myvolt$ (that's what the HC is for).
Essentially the same chip is available from Fairchild, which calls it the MM74HC08N.
The MM prefix is one that Fairchild uses, it is the same 74HC08 chip, and, for Fairchild, they use an ``N'' suffix to designate a DIP chip.

As I mentioned, these are only conventions, not standards, so they don't always apply.
However, they can be helpful, so that you know that if someone specifies a 7432 chip, and you see part numbers that say SN74LS32P, you might be able to determine that this at least has some relationship to the chip you are looking for.

% FIXME - table of common chip prefixes by family
%       - table of device numbers by family
%
