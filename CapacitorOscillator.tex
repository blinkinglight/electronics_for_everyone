\chapter{Capacitors as Timers}
\label{chapCapacitorTimer}

In this chapter, we are going to learn how to use measure the time it takes for a capacitor to charge.  
Once we learn this, we can use capacitors for timers---both for delaying signal as well as for creating an oscillating circuit.

\section{Time Constants}

As we learned in Chapter~\ref{chapCapacitors}, when a voltage is applied to a capacitor it will store energy by storing a charge on its plates, the amount of charge being based on the voltage supplied (see Equation~\ref{eqBasicCapacitance}).
However, if a capacitor charges through a resistor, then it takes much longer to fill the capacitor to capacity than if it were connected to the battery directly.
In fact, it never \emph{fully} reaches capacity, though it gets close enough that we say that it does.

The amount of time it takes a capacitor to charge is based on both the resistance of the resistor and the capacitance of the capacitor.
The actual equation for this is kind of complicated but there is a simple trick that suffices for nearly every situation, known as the RC time constant.

The RC time constant is merely the product of the resistance (in ohms) multiplied by the capacitance (in farads) which will yield the RC time constant in seconds.
The RC time constant can be used to determine how long it will take to charge a capacitor to a given level.
So, for instance, if I have a $100\myuf$ capacitor and a $500\myohm$ resistor, the RC time constant is $0.0001 * 500 = 0.05\textrm{ seconds}$.

This constant can then be used with the table in Figure~\ref{figTimeConstantTable} to determine how long it will take to charge a capacitor to a given level.

\begin{figure}
\caption{RC Time Constants}
\begin{tabular}{|l|l|l}
\textbf{\# of Time Constants} & \textbf{\% of Voltage} & \textbf{\% of Current} \\
0.5 & 39.3\% & 60.7\% \\
0.7 & 50.3\% & 49.7\% \\
1 & 63.2\% & 36.8\% \\
2 & 86.5\% & 13.5\% \\
3 & 95.0\% & 5.0\% \\
4 & 98.2\% & 1.8\% \\
5 & 99.3\% & 0.7\% \\
\end{tabular}
\end{figure}

For instance, if I wait for 2 time constants (in this case, $0.05 \cdot 2 = 0.1\textrm{ seconds}$), my capacitor will be charged to 86.5\% of the supply voltage.
The current flowing through it will be at 13.5\% of what current would be flowing if there was just a straight wire instead of a capacitor.

Different people consider different amounts as ``fully charged,'' but in this book, we will use 5 time constants to consider a capacitor fully charged.

\begin{exampleprob}
I have a power supply that is $7\myvolt$ and a capacitor that is a $100\myuf$ capacitor.
I want it to take 9 seconds to charge my capacitor.
What size of resistor do I need to use to do this?

To solve this problem, we need to work backwards.
Remember, we are considering 5 time constants to be fully charged.
Therefore, the time constant we are hoping to achieve is $9 / 5 = 1.8\textrm{ seconds}$.
The capacitance is $100\myuf$, which is $0.0001\myf$.
Since the time constant is merely the product of the capacitance and resistance, we can solve for this as follows:

\begin{align*}
\textrm{RC Time Constant} &= \textrm{capacitance} \cdot \textrm{resistance} \\
1.8 &= 0.0001 \cdot R \\
0.0001 \cdot R &= 1.8 \\
R &= \frac{1.8}{0.0001} \\
R &= 18000
\end{align*}

Therefore, to make it take 9 seconds to charge the capacitor, we need to use an $18\mykohm$ resistor.
\end{exampleprob}
