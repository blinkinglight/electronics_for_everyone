\section{Oscillator Circuits}
\label{chapOscillators}

\section{Using the NE555 Timer Chip}

%% FIXME - 
%% after showing initial timer circuit, modify it so that it shows the light for 5 seconds.  Then modify it so that there is a reset button.

In addition to using a voltage comparator with your RC timing circuit, there are other options as well.
A common chip that is built specifically for working with RC timing circuits is the NE555 Timer (also just called a 555 Timer or just a 555).
This timer has a number of features which we will explore in this chapter.

To see how the 555 Timer works, take a look at Figure~\ref{figSimple555Timer}.
In this circuit, pushing the button will turn on the LED for 10~seconds.


\simplegraphicsfigure{Simple NE555 Timer Example}{Simple555Timer}{0.08}


So far, all of the circuits we have built have been essentially linear.
Some input causes some output and then we are done.
However, other types of circuits rely on \glossterm{oscillations}.
Oscillate means to move back and forth.
In our case, an oscillator will go back and forth from zero to positive voltages.

Oscillators have all kinds of uses, usually to drive ongoing processes.
For instance, in your computer, there is a ``clock.''
This isn't a clock like your wall clock, but instead it is an oscillator.
This oscillator coordinates all of the actions in your computer.
From a simplistic perspective, every time the clock oscillates from a low voltage to a high voltage, the computer processes one instruction.
The oscillator keeps things happening, and keeps things happening in a well-ordered fashion.

Oscillators are also used in audio signals.  
Any sound that you hear is merely the vibrations of your eardrum.
In other words, your eardrum \emph{oscillates} back and forth.
What makes this happen?
Well, oscillations in the air.
What makes those happen?
Oscillations in the sound speaker.
When the speaker moves back and forth, it moves the air which produces sound that you hear.
But what moves the speaker back and forth?
Oscillations in the voltage supplied to the speaker.

So, as you can see, oscillators are very powerful tools.
But what does this have to do with capacitors?
Well, if you think about it, an oscillator needs to know \emph{how long} it should take to oscillate back and forth.
And, if something needs to know how long to do something, we can say how long using a timer circuit with capacitors.

\section{Building an Oscillator with the NE555}

A great chip for doing oscillations is the NE555 timer chip.
