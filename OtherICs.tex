\chapter{Common Integrated Circuits}

In Chapter~\ref{chapLogicICs} we studied basic logic circuits.
This chapter will introduce you to a few more complexi integrated circuits that are generally useful in a variety of applications.
Even more types of ICs can be found in Appendix~\ref{appSimplifiedDatasheets}.

\section{Holding a State}

Up until now, all of our circuits have fluctuated with their inputs. 
That is, the output is always determined by the \emph{present} state of the inputs.
But what happens if we want a circuit to \emph{remember} that something happened?
For that, we will need a way to set a state in the circuit and then hold it.
That is, we need something that we can tell it to remember a true/false value until we tell it not to.

Holding on to past states is called \glossterm{memory}.
In your computer you have huge memory chips that have billions of switches that can hold a state.
For our purposes, we will content ourselves to using chips that only hold a few states.

The simplest kind of memory in electronics is known as a \glossterm{latch}.
Latches hold a single \glossterm{bit} of memory.
A bit is simply a single true/false signal (i.e., positive voltage/zero voltage).
Latches allow you to store a true/false value and continue to retrieve it as often as you want until you choose to reset the value.
There are many different latches that are based on how they are set (and how complicated the circuitry is to implement them).

The easiest type of latch to learn about and use is the \glossterm{SR Latch}.
The SR Latch holds a value that it continually outputs on its $Q$ pin.
Additionally, some latches have an additional output which outputs the opposite of the $Q$ pin, known as $\bar{Q}$.
The SR Latch has two inputs---set ($S$) and reset ($R$).  
When the set ($S$) input alone is true, $Q$ gets set to true.
When the reset ($R$) input alone is true, then $Q$ gets reset back to false.
If both $S$ and $R$ are false, then, the latch continues to hold its previous state.
This allows for both very easily setting a state, as well as a very easy implementation.

Unfortunately, most latches start out in an \emph{unknown} state when they are first powered on.
This means that if you need it to be in a particular state when it turns on, \emph{you} have to provide circuitry to give the latch its starting value.
For the circuits in this book, we will not be overly concerned with the starting value of the latch.

\simplegraphicsfigure{Simple Circuit Using an SR Latch}{SRExample}{0.08}

To see the latch in action, Figure~\ref{figSRExample} shows a simple circuit using an SR Latch.
Notice that both button inputs have pull-down resistors going to ground in order to be sure that the inputs always have a valid state.
When the upper button is pressed, then the latch sets $Q$ to true and holds it, even after the button is no longer pressed.
When the lower button is pressed, the latch resets $Q$ to false and holds it, even after the button is no longer pressed.
Note that in a real circuit, the SR Latch would also need to be powered as well.  

The CD4043 chip is a quad SR Latch integrated circuit and can be used to build the project.
Figure~\ref{figSRChipExample} shows how to connect the circuit.
It is just like the circuit in Figure~\ref{figSRExample}, except that we are also showing the connections for the supply voltage.
Additionally, there is a pin labelled ``ENABLED'' that we are connecting to the power supply.
On many chips, you have to specifically connect certain pins to ground or to the supply voltage to get them to work like you want them to.
On this chip, it has a feature that allows you to connect or disconnect all of the outputs using the ENABLED pin (when they are disconnected, they act like there is no circuit there at all---neither true nor false).
We always want our outputs connected, so we just attach the ENABLED pin to the supply voltage.

\simplegraphicsfigure{An SR Latch Circuit Implemented Using a CD4043}{SRChipExample}{0.08}

Interestingly, SR Latches are fairly easy to implement using the standard gates discussed in Chapter~\ref{chapLogicICs}.
Figure~\ref{figSRLatch} shows how this is done.
It is simply two NOR gates wired so that their outputs feed into one another.

To understand how this works, think about what happens when $S$ is true and $R$ is false.
Since the $S$ is on a NOR gate, the output of the NOR gate will go to false.
This provides the output signal $\bar{Q}$ as well as one of the inputs to the top NOR gate.
As long as the $R$ stays at false, the output of the top NOR gate will always stay true, since neither input is true.
This ``true'' output then becomes $Q$, as well as the input to the bottom NOR gate, which keeps it on even when the $S$ input goes to false.
It isn't critically important to understand how this latch works, but sometimes knowing how something works can help you solve similar problems, or even just have a deeper intuition about how these work.

\simplegraphicsfigure{An Implementation of an SR Latch}{SRLatch}{0.08}

There are several other types of latches as well.
An $\bar{SR}$ Latch is just like the SR latch except that the inputs are inverted.
A JK Latch is similar to an SR Latch, except that $S$ is renamed $J$, $R$ is renamed $K$, and the latch defines a new behavior if \emph{both} $J$ and $K$ are true (the SR Latch usually doesn't define what happens in this case).
A D Latch is a little bit of a different beast.
It has an Enable ($E$) pin and a data ($D$) pin.
When the $E$ pin is true, then whatever is in the $D$ pin becomes $Q$.
When the $E$ pin goes false, then whatever was the last state of $Q$ stays put there, no matter what $D$ does.
It remains in that state until $E$ goes positive to allow $D$ to influence $Q$ again.

A flip-flop is similar to a latch, except that it uses an external clock pulse to synchronize changes between states at given intervals.  
We will not discuss clocked chips here. 
% FIXME - will we discuss clocked chips anywhere in the book?


