\chapter{Understanding Power}
\label{chappower}

So far we have covered the basic ideas of voltage, current, and resistance.
This is good for lighting up LEDs, but for doing work in the real world, what is really needed is \glossterm{power}.
This chapter on its own adds very little to your capabilties as a circuit designer, but it is absolutely critical background information for the chapters that follow.
Knowing about power, power conversions, and power dissipation will be critical to taking your electronics abilities into the real world.

\section{Important Terms Related to Power}

To understand what power is, we need to go through a few terms from physics (don't worry---they are all easy terms):

\begin{enumerate}
\item \glossterm{Work} happens when you move stuff.  
\item Work is measured in \glossterm{joules}. A joule is the amount of work performed when a 1 kilogram object is moved 1 meter.
\item The capacity to perform work is called \glossterm{energy}.  Energy is also measured in joules.
\item \glossterm{Power} is the sustained delivery of energy to a process.  
\item Power is measured \glossterm{watts} (abbreviated W).  Watts are just the number of joules per second.
\item Another measurement of power is horsepower (abbreviated hp).  1 horsepower is equivalent to 746 watts.  Horsepower is not important for electronics, but I wanted to mention this because horsepower is a term you have probably heard, and I wanted you to be able to connect its importance.
\end{enumerate}

One of the interesting things about work, energy, and power is that they can take on a number of forms that are all equivalent.
For instance, we can have mechanical energy, chemical energy, and electrical energy (as well as others).
We can also perform mechanical work, chemical work, and electrical work.
All these types of energy and work can be converted to each other.
They are all also measured in joules.
Therefore, we have a common unit of energy for any sort of task we want to accomplish.

Now, when we actually apply energy to perform work, we do not get a 100\% conversion rate.
That's because the process of conversion is \glossterm{inefficient}---not all of the energy gets directed to the task we want to perform.
There is no perfectly efficient process of converting energy to work.
Additionally, there is no way to create energy from nothing---any time you need additional energy you will need a source for it.

When energy is converted to work, \emph{all} of the energy does something, even if it isn't work on the task you want.
Usually, the inefficiencies get converted to \glossterm{heat}.
So, if I have a process that is only 10\% efficient, and I give that process 80 joules of energy, then that process will do only 8 joules of work, leaving 72 joules of energy that is converted to heat.

Work and energy are usually used for systems that do one thing and then stop at the end.
In electronics, we are usually building systems that stay on for long periods of time.
Therefore, instead of measuring energy, we measure power, which is the continuous delivery of power (or usage of power in doing work).

As we mentioned, power is measured in watts, which is 1 joule per second.
So, if you have a $100\mywatt$ light bulb, that bulb uses 100 joules of energy each second.
$100\mywatt$ light bulbs are very inefficient, which is why they get so hot---the energy that is not converted to light gets converted to heat instead.

\section{Power in Electronics}
\label{secpowerelec}

So, we have a basic idea about what power is in general.
In electronics, there are a few equivalent ways of calculating power.

The first is to multiply the number of volts being consumed by the number of amps of current going through a device:

\begin{equation}
P = V\cdot I
\end{equation}

Here, $P$ indicates power measured in watts, $V$ indicates volts, and $I$ indicates current measured in amps.
So, if my circuit is on a 9-volt battery, and I measure that the battery is delivering $20\mymamp$ to the circuit, then that means I can calculate the amount of power that my circuit is using (don't forget to convert milliamps to amps first!):

\begin{align*}
W &= V\cdot I \\
  &= 9\myvolt\cdot 20\mymamp  \\
  &= 9\myvolt\cdot 0.02\myamp \\
  &= 0.18\mywatt
\end{align*}

So, our circuit uses 0.18 watts of power.

You can also measure the amount of power that individual components use.
For instance, let's say that a resistor has a $3\myvolt$ voltage drop and has $12\mymamp$ of current running through it.
Therefore, the resistor uses up $3 * 0.12 = 0.36$ watts of power.

The second way of calculating power comes from applying Ohm's law.
Ohm's law say:

\begin{equation}
V = I\cdot R
\end{equation}

So, if we have the equation $P = V\cdot I$, Ohm's law allows us to \emph{replace} $V$ with $I\cdot R$.
Therefore, our new equation becomes:

\begin{equation}
P = (I\cdot R)\cdot I
\end{equation}

Or, we can simplify it further and say that

\begin{equation}
\label{eqpowercurrent}
P = I^2\cdot R
\end{equation}

We can also substitute $I = V / R$ and wind up with a third equation for power:

\begin{equation}
P = V^2 / R
\end{equation}

So, if we have $15\mymamp$ running through a $200\myohm$ resistor, then we can calculate the amount of power being used using Equation~\ref{eqpowercurrent}:

\begin{align*}
W &= I^2\cdot R \\
  &= (15\mymamp)^2\cdot 200\myohm \\
  &= (0.015\myamp)^2\cdot 200\myohm \\
  &= 0.000225\cdot 200 \\
  &= 0.045\mywatt
\end{align*}

Now, if you think about it, the resistor isn't actually \emph{doing} anything.
It is just sitting there.
Therefore, since we are not accomplishing any \emph{work} by going through the resistor, the energy gets converted to heat.
Electronics components are usually rated for how much power they can \glossterm{dissipate}, or easily get rid of.
Most common resistors, for instance, are rated between $1/16\mywatt$ and $1/2\mywatt$ (most that I've seen for sale are $1/4\mywatt$).  
This means that they will continue to work as long as their power consumption stays under their limit.
If the power consumption goes too high, they will not be able to handle the increased heat, and will break (and possibly catch fire!).

So far, our projects have dealt with low enough power that this isn't a concern.  
In fact, using $9\myvolt$ batteries, it is hard to generate more than $1/4\mywatt$ of power---you would have to have less than $350\myohm$ of resistance on the \emph{whole} circuit, and have the entire voltage drop occur on the resistor.

\section{Handling Power Dissipation with Heat Sinks}

As we mentioned earlier, when power dissipates without doing any work, it is converted to heat.
Some devices need to dissipate large quantities of heat under regular workloads.
One common device that often needs to dissipate heat is the voltage regulator, like the 7805 regulator we encountered in Chapter~\ref{chapLogicICs}.

The way that this regulator performs its job is essentially by dissipating power until the voltage is at the right level.
When used with any serious amount of current, this can actually get very, very hot.
As such, the back side of these contains a metal plate which is used to dissipate heat.
Additionally, it has a metal tab with a hole that can be used to attach a \glossterm{heatsink}.

A heatsink is a metal structure with a large surface area that helps an electronic component dissipate heat.
By being made of metal, it quickly moves the heat into itself.
By having a large surface area, it can transfer the heat to the air, where it will then disperse into the environment.

\simplegraphicsfigure{}{heatsink7805}{.25}

Figure~\ref{figheatsink7805} shows a 7805 chip next to its heatsink.
To attach the heatsink, just screw it in to the 7805.
On the 7800 series of regulators, the tab is electrically connected to ground, so it should not produce a voltage.
However, other types of chips in the same TO-220 package may actually have a voltage on the tab. 
In such a case, it would be wise to buy an isolation kit to electrically isolate the chip from the heatsink, otherwise incidental contact with the heatsink could cause a short circuit.

\section{Transforming Power}

As we have discussed, energy (and therefore power) can be transformed among a variety of forms---mechanical, electrical, chemical, etc., as long as weremember that energy is only reduced or lost, never gained.
The essence of energy transformation is at the heart of what makes batteries work.
A battery contains energy stored in a chemical form.
Chemical reactions in the battery allow electrons to move.
By drawing the electrons through a specific path (drawing them to the positive from the negative), this reaction generates electrical energy.
So we have a conversion from chemical energy (the reaction of the chemicals in the battery) into electrical energy (the pull of the electrons through the circuit).

This can also go the other way.
Electrical energy can be used to stimulate chemical reactions.
A common one is the separation of water into hydrogen and oxygen.

It is the same with mechanical energy.
In an electric motor, electrical energy is converted into mechanical energy in the motor.
But the reverse also can occur.
A power generator is made by converting mechanical energy into electrical energy.

We won't go into details on how each of these transformations work (you would need to take courses in chemistry, mechanics, etc. to know more), but the essential ideas are that

\begin{enumerate}
\item energy and power can be transformed between a variety of forms,
\item these forms of energy can all be measured with the same measuring stick (joules), and
\item every energy transformation will lose (never gain) some amount of energy through inefficiencies.
\end{enumerate}

Power and energy are known as \glossterm{conserved quantities}, because, although they are transformed, they are never created or lost.
When we talk about power lost through inefficiencies, the power actually isn't lost in total, it is merely converted to heat.
You can think of heat as power that is applied in a nonspecific direction.

In Section~\ref{secpowerelec}, we noted that in electronics, the power (measured in watts) is determined by \emph{both} the voltage and the current---by multiplying them together.
Therefore, what will be conserved in electronics will not be the voltage or the current individually, but their product.
What this means is that we can, at least in theory, increase the voltage without needing a power gain, but at the cost of current.
Likewise, we can, at least in theory, increase the current without needing a power gain, at the cost of voltage.
In both of these cases, rather than transforming electrical power to another form of power altogether, we are transforming it into a different configuration of electrical power.

Devices that convert electrical power between different voltage/current configurations are known as \glossterm{transformers}.
A step-up transformer is one that converts a low voltage to a higher voltage (at the cost of current), and a step-down transformer is one that converts a high voltage to a lower voltage (but can supply additional current).

Technically, for DC circuits, these are usually known as \glossterm{DC-DC power converters} or \glossterm{boost converters} instead of being called transformers, but the same rules apply---the total number of watts delivered can never increase but the voltage can be converted up and down at the expense of the current.
Note that if you tried to wire a regular AC transformer to a DC power supply, it would not deliver any power at all---DC-DC converters work on very different principles than AC converters.

So, if I had a source of $12\myvolt$ and $2\myamp$, then I would have a $12\cdot 2=24\mywatt$.
Therefore, it may be possible to convert that to $24\myvolt$, but I would only be able to get $1\myamp$ of current ($24\cdot 1=24$).
However, I could drop the voltage to get more current.  
If I needed $4\myamp$, I could reduce the voltage to $6\myvolt$.
Also remember that in doing these transformations there is always some amount of power loss as well, but these calculations will give you what the maximum possiblities are.

The actual mechanisms that these devices employ for doing power conversions are outside the scope of this book.

\section{Amplifying Low-Power Signals}

Microcontrollers (like the ATmega328/P) are only capable of processing and generating low-power signals.
The ATmega328/P can only source up to $40\mymamp$ per pin, and only about $200\mymamp$ total across all pins simultaneously.
At $5\myvolt$, $40\mymamp$ would yield a maximum of $0.2\mywatt$.
Therefore, if you want to turn on a device that requires more power than that, you will need to \glossterm{amplify} your signal.

Now, as we discussed previously, you can't actually create more power out of nothing.
What you can do is, instead of trying to \emph{create} power, you can instead \emph{control} power.
We will discuss several specific techniques on how to do this over the next few chapters, but the essential idea is that you can amplify a signal by using a small signal to control a larger one.

Think about your car.
The way that you control your car is by taking a low-power signal, such as the gas pedal, and using it to control a high-power signal, such as the engine.
My foot is not directly powering the car.
My foot is merely using the pedals to tell another power source---the engine---how much of its energy it should move.
My foot doesn't actually interact directly with the engine at all, except as a valve to unleash or not unleash the power of the engine.

In the same way, since the output signals from the microcontrollers are low-power, instead of using these signals directly we will use the signals to control larger sources of power.
Devices which can do this include devices such as relays, optocouplers, transistors, op-amps, and darlington arrays.
We will cover more about amplification starting in Chapter~\ref{chapTransistorIntro}.

\reviewsection

In this chapter, we learned:

\begin{enumerate}
\item Work is what happens when you move stuff, and is measured in joules.
\item Energy is the capacity to do work and is also measured in joules.
\item Power is the sustained delivery of energy, and is measured in joules per second, also called watts.
\item Power can be converted to a number of different forms.
\item Converting power to another form or using it to do work always has inefficiencies, and these inefficiencies result in heat.
\item In electronics, power (in watts) is calculated by multiplying the voltage by the current ($P = V\cdot I$).  It can also be calculated as $P = I^2\cdot R$ or $P = V^2 / R$.
\item To calculate the power consumption of an individual component, use the voltage drop \emph{of that component} and multiply the current flowing through it.
\item Most components have a maximum rating for the amount of power they can safely consume or dissipate.  Be sure your components stay under that limit.
\item Some components can handle additional power dissipation by adding a heatsink, which will more effectively dissipate the excess heat into the air.
\item Power can be transformed into other types of power (mechanical, chemical, etc.), but can never go beyond the original amount of power.
\item Electrical power can also be transformed into different combinations of voltage and current, as long as the total power remains the same.
\item Components that do this transformation are called transformers for AC power, and DC-DC converters for DC power.
\item Because power cannot be created, the way that signals are amplified is by using a small power signal to control a larger power source.
\end{enumerate}

\applysection


\begin{enumerate}
\item 
\question{If I have 50 joules of energy, what is the maximum amount of work I could possibly do with that amount of energy?}
\solution{50 joules of work}
\explanation{You cannot do more work than you have energy, therefore, with 50 joules of energy you cannot do more than 50 joules of work.  Likely you can do much less.}
\item 
\question{If I am using up 10 joules of energy each second, how many watts am I using up?}
\solution{$10\mywatt$}
\explanation{Since a watt \emph{means} a joule of energy each second, 10 joules per second is the same as 10 watts.}
\item 
\question{If I convert 30 watts of mechanical power into electrical power with 50\% efficiency, how many watts of electrical power are delivered?}
\solution{$15\mywatt$}
\explanation{The units of power are the same for both (watts), and, since there is only a 50\% efficiency, that means that we only retain half (50\%) of the power.  Half of 30 is 15.}
\item
\question{If I have a circuit powered by a $9\myvolt$ battery that uses $0.125\myamp$, how many watts does that circuit use?}
\solution{$1.125\mywatt$}
\explanation{Power is voltage multiplied by current. Therefore:
\begin{align*}
P &= V * I \\
  &= 9 * 0.125 \\
  &= 1.125\mywatt
\end{align*}
}
\item 
\question{If a resistor has a $2\myvolt$ drop with a $0.03\myamp$ current, how much power is the resistor dissipating?}
\solution{$0.06\mywatt$ or $60\mymwatt$}
\explanation{The amount of power consumed (and therefore dissipated) by the resistor is given by:
\begin{align*}
P &= V * I \\
  &= 2 * 0.03 \\
  &= 0.06\mywatt = 60\mymwatt
\end{align*}
}
\item 
\question{If a resistor has a $3\myvolt$ drop with a $12\mymamp$ current, how much power is the resistor dissipating?}
\solution{$0.036\mywatt$ or $36\mymwatt$}
\explanation{To solve, first we have to convert $12\mymamp$ to $0.012\myamp$.  Then we just use the power equation:
\begin{align*}
P &= V * I \\
  &= 3 * 0.012 \\
  &= 0.036\mywatt = 36\mymwatt
\end{align*}
}
\item 
\question{If a $700\myohm$ resistor has a $5\myvolt$ drop, how much power is the resistor dissipating?}
\solution{$0.0357 \mywatt$ or $35.7\mymwatt$}
\explanation{Since we are given the voltage and the resistance, we can use Equation~\ref{eqpowervr} to solve:
\begin{align*}
P &= V^2 / R \\
  &= 5^2 / 700 \\
  &= 25 / 700 \\
  &= 0.0357 \mywatt = 35.7\mymwatt
\end{align*}
Therefore, this resistor is dissipating $35.7\mymwatt$.
}
\item 
\question{If a $500\myohm$ resistor has $20\mymamp$ flowing through it, how much power is the resistor dissipating?  If the resistor was rated for $1/8$ of a watt, are we within the rated usage for the resistor?}
\solution{The resistor is dissipating $0.2\mywatt$.  $1/8$ of a watt is $0.125\mywatt$.  Therefore, this resistor is being used beyond its rated usage, which is hazardous.}
\explanation{Since we are given resistance and current, we can use Equation~\ref{eqpowercurrent}:
\begin{align*}
P &= I^2 \times R \\
  &= 0.02^2 \times 500 \\
  &= 0.0004 \times 500 \\
  &= 0.2\mywatt
\end{align*}

Now, is this above or below $1/8$ of a watt?
$1 / 8 = 0.125$.
Also, $0.2 > 0.125$.
Therefore, we have exceeded the power rating for this resistor, which is a hazardous situation, and can actually cause the resistor to catch fire if left in that state for too long.
}
\item 
\question{In the circuit below, calculate the voltage drop, current, and power dissipation of every component (except the battery).  If the resistors are all rated for $1/8$ of a watt, are any of the resistors out of spec? \\ \includegraphics[scale=0.08]{ExampleForPowerDissipation.png}}
\solution{\begin{description}
\item[$500\myohm$ resistor] $5.35\myvolt$ voltage drop, $0.0107\myamp$ current, $0.0572\mywatt$ power dissipation
\item[$800\myohm$ resistor] $3.65\myvolt$ voltage drop, $0.00456\myamp$ current, $0.0155\mywatt$ power dissipation
\item[$600\myohm$ resistor] $3.65\myvolt$ voltage drop, $0.00608\myamp$ current, $0.0222\mywatt$ power disspiation
\end{description}
All power dissipations were less than $1/8$ of a watt ($0.125\mywatt$), so all of the resistors are within specification.
}
\explanation{To solve this, first we need to solve for all of the voltage drops and currents along each segment of the circuit.  We can start by adding up the parallel resistances at the bottom of the circuit:
\begin{align*}
R_T &= \frac{1}{\frac{1}{800} + \frac{1}{600}} \\
    &\approx \frac{1}{0.002917} \\
    &\approx 342.82\myohm
\end{align*}
Now, those parallel resistances together are in series with the $500\myohm$ resistor at the top of the circuit.  Therefore they can be added together to get $342.82 + 500 = 842.82\myohm$ total resistance.

To calculate the total current, we just use Ohm's Law:
\begin{align*}
I &= V / R \\
  &= 9 / 842.82 \\
  &= 0.0107\myamp
\end{align*}
We can use this value to calculate the voltage drop of the first resistor:
\begin{align*}
V &= I \times R \\
  &= 0.0107 \times 500 \\
  &= 5.35\myvolt
\end{align*}
Therefore, the parallel circuit drops the rest: $9 - 5.35 = 3.65\myvolt$.
The current on the $800\myohm$ resistor is:
\begin{align*}
I &= V / R \\
  &= 3.65 / 800 \\
  &= 0.00456 \myamp
\end{align*}
The current on the $600\myohm$ resistor is:
\begin{align*}
I &= V / R \\
  &= 3.65 / 600 \\
  &= 0.00608 \myamp
\end{align*}
Now the power for each resistor is simply voltage times current.
For the $500\myohm$ resistor:
$$ P = 5.35 * 0.0107 = 0.0572 \mywatt $$
For the $800\myohm$ resistor:
$$ P = 3.65 * 0.00456 = 0.0166 \mywatt $$
For the $600\myohm$ resistor:
$$ P = 3.65 * 0.00608 = 0.0222 \mywatt $$
All of these power dissipations are less than $1/8$ of a watt ($0.125\mywatt$), so they are all within spec.
}
\end{enumerate}

