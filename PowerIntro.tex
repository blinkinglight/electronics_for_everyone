\chapter{An Introduction to Power}

So far we have covered the basic ideas of voltage, current, and resistance.
This is good for lighting up LEDs, but for doing work in the real world, what is really needed is \glossterm{power}.

\section{Important Terms Related to Power}

To understand what power is, we need to go through a few terms from physics (don't worry---they are all easy terms):

\begin{enumerate}
\item \glossterm{Work} happens when you move stuff.  
\item Work is measured in \glossterm{joules}. A joule is the amount of work performed when a 1 kilogram object is moved 1 meter.
\item The capacity to perform work is called \glossterm{energy}.  Energy is also measured in joules.
\item \glossterm{Power} is the sustained delivery of energy to a process.  
\item Power is measured \glossterm{watts} (abbreviated W).  Watts are just the number of joules per second.
\end{enumerate}

One of the interesting things about work, energy, and power is that they can take on a number of forms that are all equivalent.
For instance, we can have mechanical energy, chemical energy, and electrical energy (as well as others).
We can also perform mechanical work, chemical work, and electrical work.
All these types of energy and work can be converted to each other.
They are all also measured in joules.
Therefore, we have a common unit of energy for any sort of task we want to accomplish.

Now, when we actually apply energy to perform work, we do not get a 100\% conversion rate.
That's because the process of conversion is \glossterm{inefficient}---not all of the energy gets directed to the task we want to perform.
There is no perfectly efficient process of converting energy to work.
Additionally, there is no way to create energy from nothing---any time you need additional energy you will need a source for it.

When energy is converted to work, \emph{all} of the energy does something, even if it isn't work on the task you want.
Usually, the inefficiencies get converted to \glossterm{heat}.
So, if I have a process that is only 10\% efficient, and I give that process 80 joules of energy, then that process will do only 8 joules of work, leaving 72 joules of energy that is converted to heat.

Work and energy are usually used for systems that do one thing and then stop at the end.
In electronics, we are usually building systems that stay on for long periods of time.
Therefore, instead of measuring energy, we measure power, which is the continuous delivery of power (or usage of power in doing work).

As we mentioned, power is measured in watts, which is 1 joule per second.
So, if you have a $100\mywatt$ light bulb, that bulb uses 100 joules of energy each second.
$100\mywatt$ light bulbs are very inefficient, which is why they get so hot---the energy that is not converted to light gets converted to heat instead.

\section{Power in Electronics}

So, we have a basic idea about what power is in general.
In electronics, there are two equivalent ways of calculating power.

The first is to multiply the number of volts being consumed by the number of amps of current going through a device:

\begin{equation}
W = V\cdot I
\end{equation}

Here, $W$ indicates watts, $V$ indicates volts, and $I$ indicates current measured in amps.
So, if my circuit is on a 9-volt battery, and I measure that the battery is delivering $20\mymamp$ to the circuit, then that means I can calculate the amount of power that my circuit is using (don't forget to convert milliamps to amps first!):

$$
W &= V\cdot I \\
  &= 9\myvolt\cdot 20\mymamp  \\
  &= 9\myvolt\cdot 0.02\myamp \\
  &= 0.18\mywatt
$$

So, our circuit uses 0.18 watts of power.

You can also measure the amount of power that individual components use.
For instance, let's say that a resistor has a $3\myvolt$ voltage drop and has $12\mymamp$ of current running through it.
Therefore, the resistor uses up $3 * 0.12 = 0.36$ watts of power.

The second way of calculating power comes from applying Ohm's law.
Ohm's law say:

\begin{equation}
V = I\cdot R
\end{equation}

So, if we have the equation $W = V\cdot I$, Ohm's law allows us to \emph{replace} $V$ with $I\cdot R$.
Therefore, our new equation becomes:

\begin{equation}
W = (I\cdot R)\cdot I
\end{equation}

Or, we can simplify it further and say that

\begin{equation}
W = I^2\cdot R
\end{equation}

So, if we have $15\mymamp$ running through a $200\myohm$ resistor, then we can calculate the amount of power being used:

$$
W &= I^2\cdot R \\
  &= (15\mymamp)^2\cdot 200\myohm \\
  &= (0.015\myamp)^2\cdot 200\myohm \\
  &= 0.000225\cdot 200 \\
  &= 0.045\mywatt
$$

Now, if you think about it, the resistor isn't actually \emph{doing} anything.
It is just sitting there.
Therefore, since we are not accomplishing any \emph{work} by going through the resistor, the energy gets converted to heat.
Electronics components are usually rated for how much power they can \glossterm{dissipate}, or easily get rid of.
Most common resistors, for instance, are rated between $1/16\mywatt$ and $1/4\mywatt$.  
This means that they will continue to work as long as their power consumption stays under their limit.
If the power consumption goes too high, they will not be able to handle the increased heat, and will break (and possibly catch fire!).

So far, our projects have dealt with low enough power that this isn't a concern.  
In fact, using $9\myvolt$ batteries, it is hard to generate more than $1/4\mywatt$ of power.

\section{Transforming Voltage}
