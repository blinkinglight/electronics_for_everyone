\chapter{Introduction}

Welcome to the world of electronics!  
In the modern world, electronic devices are everywhere, but fewer and fewer people seem to understand how they work or how to put them together.
At the same time, it has never been easier to do so as an individual.
The availability of training tools, parts, instructions, videos, and tutorials available for the home experimenter has grown enormously, and the costs for equipment has dropped to almost nothing.

However, what has been lacking is a good guide to bring students from \emph{wanting} to know how electronic circuits work to actually understanding them and being able to develop their own.
For the hobbyist, there are many guides that show you how to do individual projects, but they often fail to provide enough information for their readers to be able to build projects of their own.
There is plenty of information on the physics of electricity in physics books, but they fail to make the information practical.
There are also books like \booktitle{The Art of Electronics} which are great describing how to put together circuits---but only if you are studying to be an electrical engineer, and also only if you can shell out large amounts of cash.

What has been needed for a long time is a book that takes you from knowing nothing about electronics to being able to build real circuits that you design yourself.
This book combines theory, practice, projects, and design patterns in order to enable you to build your own circuits from scratch.
Additionally, this book is designed entirely around safe, low-current DC power.
We stay far away from the wall outlet in this book to be sure that you have a safe and fun experience with electronics.

Note that this book is primarily written as a textbook for electronics classes for high-school and college students.  
It has problems to be worked, activities to do, and reviews at the end of each chapter.
However, it can also be used as a guide for hobbyists (or wannabe hobbyists) to learn on their own.
If you plan on using this book to learn on your own, we suggest that not only do you read the main parts of the chapter, but that you also do the activities and homework as well.  
The goal of the homework is to train your mind to think like a circuit designer.
If you work through the example problems, it will make analyzing and designing circuits simply a matter of habit.

\section{Working the Examples}

In this book, all examples should be worked out using decimals, not fractions.
This is an engineering course, not a math course, so feel free to use a calculator.
However, you will often wind up with very long strings of decimals on some of the answers.
Feel free to round your answers to a single decimal point.
So, for instance, if I divide 5 by 3 on my calculator, it tells me $1.66666667$.
However, I can just give the final answer as $1.7$.
This only applies to the final answer.  
You need to maintain your decimals while you do your computations.

Also, if your answer is a decimal number that \emph{begins} with zeroes, then you should round your answer to include the first 2--3 nonzero digits.
So, if I have an answer of $0.0000033333333$, I can round that to $0.0000033$.

If you have taken physics or chemistry, and you are familiar with significant digits, you can just round your answers to 3--4 significant digits.

\section{Tools You Will Need}

\fixme{Need to write this - breadboard, jumper wire, resistor, etc.}
