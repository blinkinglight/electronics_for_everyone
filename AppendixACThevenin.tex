\chapter{\thev Equivalent AC Circuits}
\label{appThevEquivAC}

% FIXME - just cut and pasted from PartialCircuit.tex.  Needs introduction and finishing

\fixme{Don't read this - it is possibly wrong}

In Chapter~\ref{chapImpedance}, we showed how reactance could not simply be added to resistance, but instead it operates as a ``sideways'' resistance.
As was noted there, reactances have to be added up \emph{separately} from resistances, and we use the letter $j$ to denote reactance values.

To calculate the reactance of capacitors and inductors at a specific frequency, we use Equations~\ref{eqCapReactance} and~\ref{eqReactanceInductor}.
Note that our results will only be valid for that specific frequency, so, when doing calculations like this for a circuit that handles more than one frequency, we would want to use the most likely frequency for the calculation.

To find the \thev Equivalent circuit for an AC circuit, you would first calculate the reactances of the individual components and label them with their reactances, using the $j$ notation.
Then, you would calculate the \thev Equivalent circuit the same way as before.

\simplepdffigure{A Circuit with Resistive and Reactive Components}{ThevACExample}{0.25}

Let's look at an example of how this is done.
Figure~\ref{figThevACExample} shows a partial circuit that is a combination of resistive and reactive elements.
To find out \thev Equivalent circuit, we first need to calculate the impedances of each component in the circuit.  
We've already had a lot of practice with that, so Figure~\ref{figThevACExampleImps}

\simplepdffigure{Circuit with Impedances Shown}{ThevACExampleImps}{0.25}

