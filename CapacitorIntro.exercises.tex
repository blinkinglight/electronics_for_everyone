
\begin{enumerate}
\item 
\question{Convert $23\myf$ to microfarads.}
\solution{$23\myf = 23,000,000\myuf$}
\explanation{$1\my = 1,000,000\myuf$.  Therefore, to convert we multiply by $1,000,000$.}
\item 
\question{Convert $15\myuf$ to farads.}
\solution{$15\myuf = 0.000015\myf$}
\explanation{$1\myuf = 0.000001\myf$.  Therefore, to convert we multiply by $0.000001$.}
\question{Convert $35\mypf$ to farads.}
\solution{$35\mypf = 0.000000035\myf$}
\explanation{$1\mypf = 0.000000001\myf$.  Therefore, to convert we multiply by $ 0.000000001$.}
\item 
\question{Convert $0.0002\myf$ to microfarads.}
\solution{$0.0002\myf = 200\myuf$}
\explanation{$1\myf = 1,000,000\myuf$.  Therefore, to convert we multiply by $1,000,000$.}
\item 
\question{Convert $0.0030\myuf$ to farads.}
\solution{$0.0030\myuf = 0.000000003\myf$}
\explanation{$1 \myuf = 0.000001\mf$. Therefore, to convert we multiply by $0.000001$.}
\item 
\question{If a voltage of $6\myvolt$ is applied to a $55\myuf$ capacitor, how much charge would it store?}
\solution{$0.00033$ coulombs}
\explanation{The equation for charge is $Q = C \cdot V$, where $C$ is the capacitance in farads, and the result is in coulombs.
Therefore, begin by converting the capacitance to farads.
$1\myuf = 0.000001\myf$, so $55\myuf = 0.000055\myf$.
The equation then becomes:
\begin{align*}
Q &= C \cdot V \\
  &= 0.000055 \cdot 6 \\
  &= 0.00033
\end{align*}
Therefore, the capacitor will be holding $0.00033$ coulombs of charge.
}
\item 
\question{If a voltage of $2\myvolt$ is applied to a $13\mypf$ capacitor, how much charge would it store?}
\solution{$0.000000026$ coulombs}
\explanation{The equation for charge is $Q = C \cdot V$, where $C$ is the capacitance in farads, and the result is in coulombs.
Therefore, begin by converting the capacitance to farads.
$1\mypf = 0.000000001\myf$, so $13\mypf = 0.000000013\myf$.
Therefore, the equation becomes:
\begin{align*}
Q &= C \cdot V \\
  &= 0.000000013 \cdot 2 \\
  &= 0.000000026
\end{align*}
Therefore, the capacitor will store $0.000000026$ coulombs worth of charge.
} 
\item 
\question{If a $132\myuf$ capacitor is holding $0.02\textrm{ coulombs}$ of charge, how many volts will it produce when it begins to discharge?}
\solution{$151.5\myvolt$}
\explanation{The equation for charge in a capacitor is $Q = C \cdot V$ where $Q$ is in coulombs and $C$ is in farads.
This can be rearranged to solve for voltage, giving $V = \frac{Q}{C}$.
Then, we convert $132\myuf$ to farads.
Since $1\myuf = 0.000001\myf$, then $132\myuf = 0.000132\myf$.
Therefore, the equation becomes:
\begin{align*}
V &= \frac{Q}{C} \\
  &= \frac{0.02}{0.000132} \\
  &\approx 151.5
\end{align*}
Therefore, when the capacitor begins to discharge, it will discharge at $151.5\myvolt$.
}
\item 
\question{if a $600\mypf$ capacitor is holding $0.03\textrm{ coulombs}$ of charge, how many volts will it produce when it begins to discharge?}
\solution{$50,000\myvolt$}
\explanation{
The equation for the voltage when discharging a capacitor is $V = \frac{Q}{C}$, where the charge is in coulombs and the capacitance is in farads.
We will start by converting $600\mypf$ to farads.
Since $1\mypf = 0.000000001\myf$, then $\600\mypf = 0.0000006\myf$.
Therefore, the equation becomes:
\begin{align*}
V &= \frac{Q}{C} \\
  &= \frac{0.03}{0.0000006} \\
  &= 50000
\end{align*}
Therefore, this would generate $50,000\myvolt$ when it began to discharge. 
Note that this is probably beyond the physical capacity of such a device, and, even if you could charge it up that much, it would discharge almost instantaneously, kind of like a static electricity shock (which has almost the same amount of voltage).
}
\item 
\question{If a $121\myuf$ capacitor is connected to a battery.  After some fluctuation, the capacitor has $0.00089\textrm{ coulombs}$ of charge stored in it and the battery is reading $8.9\myvolt$.  Is the capacitor going to be charging or discharging at this point?}
\solution{The capacitor will be charging.}
\explanation{To know if the capacitor will be charging or discharging, find out first how much charge the capacitory \emph{could} store at the supplied voltage.
If the charge in the capacitor is less than the potential charge at that voltage, then that means that charge would be flowing in.
If the charge in the capacitor is greater than the potential charge at that voltage, then the charge will flow out into the circuit.

At $8.9\myvolt$, the amount of charge for a $121\myuf$ ($0.000121\myf$) capacitor is:
\begin{align*}
Q &= C \cdot V \\
  &= 0.000121 \cdot 8.9 \\
  &= 0.0010769
\end{align*}
Therefore, at $8.9\myvolt$, the charge in the capacitor should go to $0.0010769$ coulombs.
The actual charge in the capacitor is $0.00089$ coulombs, which is \emph{less} than the charge should be at that voltage.
Therefore, the charge is flowing into the capacitor.
}
% FIXME - more problems like this one?
\end{enumerate}
