
%% Math font options
% \usepackage[math]{iwona}
% \usepackage[math]{kurier}
% \usepackage{cmbright}
% \usepackage{lmodern}

%% Font-y stuff



\usepackage{siunitx}
\usepackage{dingbat}
\usepackage[T1]{fontenc}
\usepackage[utf8]{inputenc}
\usepackage{amsfonts}
\usepackage{textcomp}
\usepackage{newtxmath} % Better math lettering (v vs u)
% \usepackage{hyperref} % doesn't work with other packages here


\usepackage{amsmath}

\usepackage{needspace}
%\usepackage{pgfplots}
\usepackage{mdframed}
\usepackage{placeins} % Give float barriers

\usepackage{makeidx}
\usepackage[titles]{tocloft}
\setlength{\cftbeforechapskip}{-0.5ex}

\usepackage{array} % custom column cells
\newcolumntype{M}[1]{>{\centering\arraybackslash}m{#1}}
\setlength{\tabcolsep}{1em}
\renewcommand{\arraystretch}{1.6}



% Better Hyphenation
\usepackage[none]{hyphenat}
\usepackage[english]{babel} % Prevents underscore from causing problems with tex4ht 
% Graphics
\usepackage{graphicx}
\usepackage{xcolor}
\usepackage{maxiplot} % Maxima interface

% Code listings
\usepackage{listings}
\lstset{ %
basicstyle=\ttfamily,
breakatwhitespace=true,
breaklines=true,
tabsize=3
}

\newenvironment{typing}[1]{\begin{figure}[H] \caption{#1} \begin{mdframed}}{\end{mdframed} \end{figure}}
\newenvironment{typingwithlabel}[2]{\begin{figure} \caption{#1} \label{#2} \begin{mdframed}}{\end{mdframed} \end{figure}}

\makeatletter
\@ifpackageloaded{tex4ht}{
\newcommand{\forebook}[1]{#1}
\newcommand{\forprintbook}[1]{}
}{
\newcommand{\forebook}[1]{}
\newcommand{\forprintbook}[1]{#1}
}
\makeatother


\setcounter{tocdepth}{0} % TOC to section level
\setlength{\parindent}{0in} % No paragraph indents
\setlength{\parskip}{10pt} % Between-paragraph skips


\forprintbook{
\def\bookpart#1{%
  \par\newpage\cleardoublepage % Page break
  \thispagestyle{empty}
  \chaptermark{~}
  \sectionmark{~}
  \markboth{~}{~}
  \vspace*{1in} % Vertical shift
  \refstepcounter{part}% Next part
  {\centering\textbf{\Huge Part \thepart}\par}% 
  \addcontentsline{toc}{part}{{\thepart}~~~~~#1}
  \vspace{1cm}% Vertical shift
  {\centering \textbf{\Huge #1}\par}%
  \vspace{2cm}% Vertical shift
  % Some text
}
}

\forebook{
\def\bookpart#1{%
%\section*{~}
%\refstepcounter{part}% Next part
%{\centering\textbf{\Huge Part \thepart}\par}% 
% \part{#1}
% \addcontentsline{toc}{part}{{\thepart}~~~~~#1}
}
}

\forprintbook{
\newenvironment{partintro}{\begin{mdframed}[backgroundcolor=gray!10,skipabove=\baselineskip,skipbelow=\baselineskip]%
~\\ 
}{%
~\\
\end{mdframed}
}
}

\forebook{
\newenvironment{partintro}{}{}
}


% \pgfplotsset{compat=1.10}
\newcommand{\thev}{Th\'{e}venin\ } % Thevenin
\newcommand{\booktitle}[1]{\emph{#1}}
\newcommand{\myamp}{\,\si{\ampere}}
\newcommand{\mymamp}{\,\si{\milli\ampere}}
\newcommand{\myuamp}{\,\si{\micro\ampere}}
\newcommand{\myvolt}{\,\si{\volt}}
\newcommand{\myohm}{\,\si{\ohm}}
\newcommand{\mykohm}{\,\si{\kilo\ohm}}
\newcommand{\myMohm}{\,\si{\mega\ohm}}
\newcommand{\mywatt}{\,\si{\watt}}
\newcommand{\mymwatt}{\,\si{\milli\watt}}
\newcommand{\myuf}{\,\si{\micro\farad}}
\newcommand{\mynf}{\,\si{\nano\farad}}
\newcommand{\mypf}{\,\si{\pico\farad}}
\newcommand{\myf}{\,\si{\farad}}
\newcommand{\myhz}{\,\si{\hertz}}
\newcommand{\mykhz}{\,\si{\kilo\hertz}}
\newcommand{\myhy}{\,\si{\henry}}
\newcommand{\mymhy}{\,\si{\milli\henry}}
\newcommand{\myuhy}{\,\si{\micro\henry}}
\newcommand{\mywb}{\,\si{\weber}}
\newcommand{\myuwb}{\,\si{\micro\weber}}
\newcommand{\mysec}{\textrm{ seconds}}
\newcommand{\deriv}[1]{\mathrm{d}#1}
\newcommand{\pd}[2]{\partial{#1}_{#2}}
\newcommand{\myint}{\int\!}
\renewcommand{\dh}{\deriv{h}}
\newcommand{\dx}{\deriv{x}}
\newcommand{\dy}{\deriv{y}}
\newcommand{\ds}{\deriv{s}}
\newcommand{\dr}{\deriv{r}}
\newcommand{\dt}{\deriv{t}}
\newcommand{\du}{\deriv{u}}
\newcommand{\dv}{\deriv{v}}
\newcommand{\dw}{\deriv{w}}
\newcommand{\diff}{\mathrm{d}}
\newcommand{\dz}{\deriv{z}}
\newcommand{\dq}{\deriv{q}}
\newcommand{\dQ}{\deriv{Q}}
\newcommand{\dV}{\deriv{V}}
\newcommand{\dydx}{\frac{\dy}{\dx}}
\newcommand{\arccot}{\mathrm{arccot}}
\newcommand{\arcsec}{\mathrm{arcsec}}
\newcommand{\arccsc}{\mathrm{arccsc}}
\newcommand{\glossterm}[1]{\textbf{#1}}
\newcommand{\fixme}[1]{FIXME---\textbf{#1}}
\newcommand{\degrees}{^{\circ}}
\renewcommand{\times}{*}

\newcommand{\icode}[1]{\texttt{#1}}


\newcommand\simplepdffigure[3]{
\begin{figure}
\caption{#1}
\label{fig#2}
\centering
\includegraphics[scale=#3]{#2.pdf}
\end{figure}
}

\newcommand\simplegraphicsfigure[3]{
\begin{figure}
\caption{#1}
\label{fig#2}
\centering
\includegraphics[scale=#3]{#2.png}
\end{figure}
}

\newcommand\simplegraph[1]{
	\begin{tikzpicture}
		\begin{axis}[
			xlabel=$x$,
			ylabel=$y$,
			axis equal image
		]
			\addplot+[mark=none,smooth]{#1};
		\end{axis}
	\end{tikzpicture}
}

\newcommand\mxpoutopt[1]{gnuplot_out_file,"./generated_plots\jobname#1.png"}
\newcommand\maxgraphout[1]{\begin{center}\mxpIncludegraphics[scale=0.60]{generated_plots/\jobname#1.png}\end{center}}
\newcommand\maxgraphscale[2]{\begin{center}\mxpIncludegraphics[scale=#2]{generated_plots/\jobname#1.png}\end{center}}
\newcommand\maxdraw[3]{
	\begin{maximacmd}
		draw2d(terminal=eps_color, file_name="generated_plots/\jobname#1", xaxis=true, yaxis=true, yaxis_type=solid, xaxis_type=solid, #2);
	\end{maximacmd}
	\begin{center}
		\mxpIncludegraphics[scale=#3]{generated_plots/\jobname#1.eps}
	\end{center}
}

\newcommand\maxgraph[3]{
	\begin{maximacmd}
		plot2d(#2, #3, [gnuplot_preamble,"set zeroaxis;"], [gnuplot_term, png], [\mxpoutopt{#1}])$
	\end{maximacmd}
	\maxgraphout{#1}
}

\begin{maximacmd}
load(draw);
set_draw_defaults(grid=true, fill_color=grey, xaxis_type=solid, yaxis_type=solid, xlabel="x", ylabel="y");
\end{maximacmd}
% xaxis_color, xaxis_type (solid), xaxis_width, yaxis..., font, font_size, line_width

\newcounter{examplecounter}
\def\theexamplecounter{\thechapter.\arabic{examplecounter}}
\newenvironment{exampleprob}{\begin{quote}%
\refstepcounter{examplecounter}%
\textbf{Example \theexamplecounter}%
\quad
}{%
\end{quote}%
}

\newenvironment{advsidebar}[1]{%
	\begin{sidebar}[Advanced: #1]
}{%
	\end{sidebar}
}

\newenvironment{sidebar}[1][]{%
	\Needspace{10\baselineskip}
	\begin{mdframed}[%
		backgroundcolor=gray!10,
		frametitle={--- \hskip 10pt #1},
		frametitleaboveskip=0pt,
		frametitlebelowskip=10pt,
		innertopmargin=20pt,
		innerbottommargin=10pt,
		shadow=false,
		shadowsize=2pt,
		skipabove=\baselineskip,
		skipbelow=\baselineskip,
		linewidth=0.5pt,
		frametitlerule=true,
		frametitlebackgroundcolor=gray!30
	]%
}{%
	\end{mdframed}
}

\newcommand\reviewsection{\newpage\section*{Review}}
\newcommand\exercisesection{\newpage\section*{Exercises}}
\newcommand{\applysection}{\section*{Apply What You Have Learned}}


